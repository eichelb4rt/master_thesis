\section{Fully Connected Feed-Forward Net}
A single layer of a fully connected Feed-Forward Neural Net with ReLU activations can be expressed as a nested Einsum expression, with the use of multiple semirings.
For this, let
\begin{itemize}
    \item $R_{(+, \cdot)}$ be the standard semiring,
    \item $R_{(\max, +)}$ be the tropical semiring,
    \item $R_{(\min, \max)}$ be the minimax semiring.
\end{itemize}
Let $\nu: \R^n \rightarrow \R^m$ be the function that computes the output of the layer for a given input vector $x \in \R^n$, with weights $A \in \R^{m \times n}$ and biases $b \in \R^m$.
Then
\begin{align*}
    \nu(x) & = \max(Ax + b, 0)                                                                                                                   \\
           & = (i,i\rightarrow i, 0, (i,i \rightarrow i, b, (ij, j \rightarrow i, A, x)_{R_{(+, \cdot)}} )_{R_{(\max, +)}} )_{R_{(\min, \max)}}.
\end{align*}
A multi-layer network can the be achieved by nesting the respective Einsum expressions of each layer.

Because each level of nesting needs a different semiring, we can not use the theorems from \autoref{chap:nested} to compress the expression.
But if we could find a way of compressing the massively nested expressions of deeper neural networks despite of that, then we could benefit from the advantages mentioned in \autoref{sec:methods:comp_advantages} such as the optimisation of contraction paths.
Unfortunately, it is unlikely that this is possible, because we found that expanding the matrix multiplication, that transforms the outputs of another layer, results in an exponentially big term.

To see this, we use the tropical semiring $(\R, \oplus, \odot)$, where $a \oplus b = \max(a,b)$ and $a \odot b = a + b$.
We do this because tropical semiring can naturally express all the operations used in a fully connected neural network.
For this we need to define the tropical power:
$$a^{\odot n} = \underbrace{a \odot a \odot \ldots \odot a}_{n \text{ times}}$$
for $n \in \N$. The following property of the tropical power is also needed:
$$\left(a^{\odot b}\right)^{\odot c} = a^{\odot(b + c)}.$$
The tropical semiring also allows us to use the distributive law of maximization and addition
$$a \odot (b \oplus c) = a \odot b \oplus a \odot c,$$
as well as the distributive law of addition and multiplication
$$(a \odot b)^{\odot n} = a^{\odot n} \odot b^{\odot n},$$
and the distributive law of maximization and multiplication, which is restricted on natural numbers
$$(a \oplus b)^{\odot n} = a^{\odot n} \oplus b^{\odot n}.$$

Let $\nu: \R^n \rightarrow \R^l$ be the function that computes the output of a two-layer fully connected neural network ($n \rightarrow m \rightarrow l$ neurons) with ReLU activations,
which maps inputs $x \in \R^n$ to outputs $\nu(y) \in R^l$, with parameters $A^{(0)} \in \R^{m \times n}, A^{(1)} \in \R^{l \times m}, b^{(0)} \in \R^m, b^{(1)} \in \R^l$.
Then the computation of the neural network is:
$$\nu(x) = \max(A^{(1)}\max(A^{(0)}x + b^{(0)}, 0) + b^{(1)}, 0)$$

In order to reasonably work with matrix multiplication in the tropcial semiring, we can only view matrices with positive integer entries.
% TODO: quote
This is not a limitation, because making the entries integers does not impact the strength of the neural network, because [...] (quote the paper).

In order to only use positive valued matrices, we can rewrite the expression of computing the next layer from a previous layer:
\begin{align*}
    \max(Ax + b, 0) & = \max(A_+ x - A_- x + b, A_- x - A_- x) \\
                    & = \max(A_+ x + b, A_- x) - A_- x
\end{align*}
where $A_+ = \max(A, 0), A_- = \max(-A, 0)$ and therefore $A = A_+ - A_-$.
% TODO: quote
This turns the network output into a tropical rational function (quote):
\begin{align*}
    \nu(x) & = \max(\overbrace{A^{(1)}_+ \max(A^{(0)}_+ x + b^{(0)}, A^{(0)}_- x)}^z + A^{(1)}_- A^{(0)}_+ x + b^{(1)}, \\
           & \phantom{{} =} A^{(1)}_- \max(A^{(0)}_+ x + b^{(0)}, A^{(0)}_- x) + A^{(1)}_+ A^{(0)}_+ x)                 \\
           & \phantom{{} =} -\left[A^{(1)}_- \max(A^{(0)}_+ x + b^{(0)}, A^{(0)}_- x) + A^{(1)}_+ A^{(0)}_+ x\right]
\end{align*}
We focus on the subexpression $z$, which makes the calculation a bit simpler, but keeps the point.

Now, if we want to avoid switching semirings, we need to apply the distributive law a bunch of times.
\begin{align*}
    z   & = A^{(1)}_+ \max(A^{(0)}_+ x + b^{(0)}, A^{(0)}_- x)                                                                                                                                                                                                                                                                                                          \\
    z_i & = \bigodot\limits_{j = 1}^{m}\left(b^{(0)}_j \odot \bigodot\limits_{k = 1}^{n} x_k^{\odot A^{(0)}_{jk+}} \oplus \bigodot\limits_{k = 1}^{n} x_k^{\odot A^{(0)}_{jk-}}\right)^{\odot A^{(1)}_{ij+}}                                                                                                                                                            \\
        & = \bigodot\limits_{j = 1}^{m}\left(\left(b^{(0)}_j\right)^{\odot A^{(1)}_{ij+}} \odot \bigodot\limits_{k = 1}^{n} x_k^{\odot \left(A^{(1)}_{ij+} + A^{(0)}_{jk+}\right)} \oplus \bigodot\limits_{k = 1}^{n} x_k^{\odot \left(A^{(1)}_{ij+} + A^{(0)}_{jk-}\right)}\right)                                                                                     \\
        & = \bigoplus\limits_{J \in 2^{[m]}} \bigodot\limits_{j \in J} \left[\left(b^{(0)}_j\right)^{\odot A^{(1)}_{ij+}} \odot \bigodot\limits_{k = 1}^{n} x_k^{\odot \left(A^{(1)}_{ij+} + A^{(0)}_{jk+}\right)}\right] \odot \bigodot\limits_{j \in [n] \setminus J} \left[\bigodot\limits_{k = 1}^{n} x_k^{\odot \left(A^{(1)}_{ij+} + A^{(0)}_{jk-}\right)}\right]
\end{align*}
Where the second equality is just the first equality written with the operations of the tropical semiring,
the third equality follows from the distributive law of standard addition and multiplication and the distributive law of maximization and multiplication,
and the last equality follows from the distributive law of maximization and addition.

This expression maximizes over a number of subexpressions that grows exponentially in the width of the inner layer.
Which subexpressions can be removed before the evaluation remains an open question.
Note that it depends on the non-linearities of the neural network, which might make it hard to find a general answer to this question.

% TODO: beschreiben, wieso das trotzdem nützlich sein könnte? (Pipelining?)
