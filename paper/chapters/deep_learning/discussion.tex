\section{Discussion}

The techniques that we were able to describe with a single Einsum expression respectively are convolution and max-pooling.
But even here, the mapping is not optimal because we still need design tensors to compute the permutations required for these techniques.
Because these design tensors have a potentially high order, they can be computationally inefficient.
We also showed that they cannot be split up into a product of lower order tensors, which might have been a way to reduce the computational workload with the optimization of contraction paths.
However, this problem might be smaller in practice.
In the case of convolution in inference, the same constant masks are repeatedly used on different input data.
This allows us to precompute the contraction of the mask and reuse the result on different inputs.

We did not manage to describe the other presented examples with a single Einsum expression respectively.
Instead, we had to use mixtures of multiple Einsum expressions over different semirings and element-wise functions.
Such mixtures are technically not in the framework of Einsum, but they might still be helpful,
because they can be computed sequentially.
Because the steps in this sequential computation are all Einsum expressions or element-wise functions,
they could all benefit from specific hardware and efficient Einsum engines individually,
which could result in a solution that is still efficient overall.

Because these mixtures are not supported by Einsum, this result means that Einsum is probably not powerful enough to describe the standard architectures in deep learning.
We do not know this for sure, as there might be semirings we did not explore, that enable a natural translation of these architectures.
However, we believe this to be unlikely, given the high number of operations used both in the combination of tensor entries and the aggregation over axes.
Another possibility which we did not explore is, that there are other architectures which approximate standard architectures, and therefore produce similar outputs, that can naturally be translated to Einsum with the semirings we used.
Architectures like this could be explored in future work.
