\section{Duplications}

The following two theorems revolve around duplicated symbols in index strings and the way these duplications are \textit{broken}.
We speak of a broken duplication in $\bm{s_u}$, if the symbols $s_{ui}$ and $s_{uj}$ at two positions $i,j \in [n_u]$ are the same, meaning $s_{ui} = s_{uj}$,
but symbols at the same positions in $\bm{\hat{s_u}}$, $\hat{s}_{ui}$ and $\hat{s}_{uj}$ are different, meaning $\hat{s}_{ui} \neq \hat{s}_{uj}$.
In the same way, duplications in $\bm{\hat{s}_u}$ can be broken by $\bm{s_u}$.

Because $\bm{s_u}$ and $\bm{\hat{s}_u}$ are easily confused, we can also use a different terminology that does not include these similar variable names.
If $\bm{s_u}$ breaks duplications in $\bm{\hat{s}_u}$, we say that the outer expression \textit{introduces} duplications to the inner expression,
because the outer expression uses duplications in the input string for the inner expression, that were not used in the output string of the inner expression.
If $\bm{\hat{s}_u}$ breaks duplications in $\bm{s_u}$, we say that the outer expression \textit{removes} duplications from the inner expression,
because the outer expression uses different symbols in the input string for the inner expression, where there was orignally a duplication in the output string of the inner expression.

\section{Introducing Duplications}

The following example is another expression, which we cannot compress with the previous theorem:
$$(ij, jjj \rightarrow i, A, (kl, lo \rightarrow kko, B, C))$$
for $A \in \R^{a \times b}, B \in \R^{b \times c}, C \in \R^{c \times b}$.
In the following, we will explore how to compress such expressions.
Note that, for the theorem, we use disjoint sets of symbols for the inner and outer expression.
This helps in the proof, and is not a real constraint in practice,
because we can just rename the symbols in different scopes.
For example, we could also write the above expression as
$$(ij, jjj \rightarrow i, A, (ik, kj \rightarrow iij, B, C)),$$
because the scope of each symbol does not reach into nested expressions,
and therefore the $j$ used in the outer expression and the $j$ used in the inner expression are treated as different symbols.

\begin{theorem}
    \label{thm:nested_einsum:2}

    For $i \in [m + n]$, let $T^{(i)}$ be an $n_i$-th order tensor with index strings $\bm{s_i} \in S^{n_i}$.
    Let $\bm{s_u}$ be an index string for the $n_u$-th order tensor $U$, which is defined as follows:
    $$U := (\bm{s_{m + 1}},\dots,\bm{s_{m + n}} \rightarrow \bm{s_u}, T^{(m + 1)},\dots,T^{(m + n)})$$
    Also let $\bm{\hat{s}_u}$ be alternative index strings for $U$ with $s_{uj} = s_{uj'} \implies \hat{s}_{uj} = \hat{s}_{uj'}$ for all $j, j' \in [n_u]$,
    which means that $\bm{\hat{s}_u}$ can only introduce new symbol duplications, and cannot remove any.
    The index string $\bm{s_u}$ corresponds to the output string of the inner expression,
    and the index string $\bm{\hat{s}_u}$ corresponds to the input string that is used for the input tensor $U$ in the outer expression.

    In our example, $\bm{s_u} = kko$ and $\bm{\hat{s}_u} = jjj$.
    This does not break the symbol duplication of the first and second index,
    and introduces a new duplication on the third index.

    Let $s_v$ be an index string and
    $$V := (\bm{s_1},\dots,\bm{s_m}, \bm{\hat{s}_u} \rightarrow \bm{s_v}, T^{(1)},\dots,T^{(m)}, U)$$
    such that the first and second Einsum expression share no symbols.
    Then these nested Einsum expressions can also be compressed into a single Einsum expression.

    In contrast to \cref{thm:nested_einsum:1}, we cannot just replace the input index string $\bm{\hat{s}_u}$ by all the input index strings in the inner Einsum expression $\bm{s_{m + 1}},\dots,\bm{s_{m + n}}$.
    Instead, we first need to apply a symbol map to the input strings of the inner expression.
    Let $\nu: S \rightarrow S$ such that
    $$\nu(s) := \begin{cases}
            \hat{s}_{uj} & \text{if }\exists j \in [n_u]: s_{uj} = s \\
            s            & \text{else}
        \end{cases}$$
    which maps symbols in $\bm{s_u}$ to the symbol at the same index in $\bm{\hat{s}_u}$ and all other symbols to themselves.

    This symbol map holds information about which symbols will be iterated over at the same time in the outer expression.
    In our example, we have the following symbols on the same positions:
    \begin{itemize}
        \item $s_{u1} = k$ and $\hat{s}_{u1} = j$,
        \item $s_{u2} = k$ and $\hat{s}_{u2} = j$,
        \item $s_{u3} = o$ and $\hat{s}_{u3} = j$.
    \end{itemize}
    Therefore these are the important mappings:
    \begin{align*}
        k & \rightarrow j, \\
        o & \rightarrow j.
    \end{align*}
    This means that $k$ and $o$ will be iterated over at the same time.

    The symbol map $\nu$ can be extended, such that it maps entire index strings instead of just symbols, by setting $\nu(\bm{s_i}) \in S^{n_i}, \nu(\bm{s_i})_j := \nu(s_{ij})$.
    Then we can write the substituted index strings by setting $\bm{\hat{s}_i} := \nu(\bm{s_i})$ for $i \in [m + 1, m + n]$.

    The compressed Einsum expression now becomes the following:
    $$V = (\bm{s_1},\dots,\bm{s_m}, \bm{\hat{s}_{m + 1}}, \dots, \bm{\hat{s}_{m + n}} \rightarrow \bm{s_v}, T^{(1)},\dots,T^{(m + n)})$$
    which helps us to compress the example:
    $$(ij, jjj \rightarrow i, A, (kl, lo \rightarrow kko, B, C)) = (ij, jl, lj \rightarrow i, A, B, C).$$
\end{theorem}

\bigskip
\begin{proof}
    \small
    The fundamental idea behind this theorem is, that by using the index string $\bm{\hat{s}_u}$, we only iterate over a sub-space of the indices that we defined for the computation of $U$.
    The way in which we iterate over this sub-space is determined by the outer expression.
    It could either be the sum over bound indices or the universal quantifier over free indices.
    To formulate this, we need some idea of which multi-indices we iterate over.
    Therefore, let $\mathcal{I}:\bm{s} := \smallset{\bm{i}: \bm{s} \mid \bm{i} \in \mathcal{I}}$ for an index string $\bm{s}$ and a multi-index space $\mathcal{I}$.

    Let $F', B'$ be the free and bound symbols of the inner Einsum expression.
    W.l.o.g. they are both non-empty.
    From them we can derive the multi-index spaces $\mathcal{F}', \mathcal{B}'$ as in the definition.
    Let $\hat{F}' = \sigma(\bm{\hat{s}_u})$ and $\mathcal{\hat{F}}' = \prod_{s \in \hat{F}'} [d_s]$.
    Then $\mathcal{\hat{F}}':\bm{\hat{s}_u} \subseteq \mathcal{F}':\bm{s_u}$.
    This follows from $d_{s_{uj}} = d_{\hat{s}_{uj}}$ for $j \in [n_u]$,
    and the amount of symbols in the projection of $\mathcal{\hat{F}}':\bm{\hat{s}_u}$ being smaller or equal to the amount of symbols in the projection of $\mathcal{F}':\bm{s_u}$.
    The first fact is true per the definition of Einsum.
    The second fact can be rewritten as $\abs{\sigma(\bm{\hat{s}_u})} \leq \abs{\sigma(\bm{s_u})}$ and follows directly from the constraint $s_{uj} = s_{uj'} \implies \hat{s}_{uj} = \hat{s}_{uj'}$ for all $j,j' \in [n_u]$.

    Then
    $$\forall \bm{f'} \in \mathcal{F}': U_{\bm{f'}: \bm{s_u}} = \bigoplus\limits_{\bm{b'} \in \mathcal{B}'}\bigodot\limits_{i = m + 1}^{m + n} T^{(i)}_{(\bm{f'}, \bm{b'}):\bm{s_{i}}}$$
    and therefore
    $$\forall \bm{\hat{f}'} \in \mathcal{\hat{F}}': U_{\bm{\hat{f}}': \bm{\hat{s}_u}} = \bigoplus\limits_{\bm{b'} \in \mathcal{B}'}\bigodot\limits_{i = m + 1}^{m + n} T^{(i)}_{(\bm{\hat{f}}', \bm{b'}):\bm{\hat{s}_{i}}}$$
    because of the previous observation,
    and because the bound symbols of the expression, which are used in $\bm{b'}$, do not occur in $\bm{s_u}$, and are therefore not changed by the symbol map $\nu$.
    Therefore
    $$U = (\bm{\hat{s}_{m + 1}},\dots,\bm{\hat{s}_{m + n}} \rightarrow \bm{\hat{s}_u}, T^{(m + 1)},\dots,T^{(m + n)})$$
    and we can use \cref{thm:nested_einsum:1} for
    $$V = (\bm{s_1}, \dots, \bm{s_m}, \bm{\hat{s}_{m + 1}}, \dots, \bm{\hat{s}_{m + n}} \rightarrow \bm{s_v}, T^{(1)}, \dots, T^{(m + n)})$$
    because the bound symbols of the inner expression have not been mapped to any of the symbols used in the outer expression.
\end{proof}
\bigskip

This theorem suffices to prove a property of the trace in a relatively simple manner, namely that for $A \in \R^{m \times n}, B \in \R^{n \times m}$,
it holds that
$$\text{trace}(A \cdot B) = \text{trace}(B \cdot A).$$

\begin{proof}
    \small
    \begin{align*}
        \text{trace}(A \cdot B) & = (ll \rightarrow , (ik,kj \rightarrow ij, A, B))  \\
                                & = (lk, kl \rightarrow ,A, B)                       \\
                                & = (kl, lk \rightarrow ,B, A)                       \\
                                & = (kk \rightarrow , (il, lj \rightarrow ij, B, A)) \\
                                & = \text{trace}(B \cdot A)
    \end{align*}
    where the second and fourth equality hold because of \cref{thm:nested_einsum:introduce_duplications},
    and the third equality holds because of the commutativity of multiplication in the standard semiring.
\end{proof}
\bigskip

\section{Removing Duplications}

The following example is an expression, which we cannot compress with the previous theorems:
$$(ij, kl, mn, ijklmn \rightarrow ijk, A, B, C, (abc \rightarrow aabbcc, D))$$
for $A \in \R^{x \times x}, B \in \R^{y \times y}, C \in \R^{z \times z}, D \in \R^{x \times y \times z}$.
This is because duplications in $\bm{s_u} = aabbcc$ are removed by the input string $\bm{\hat{s}_u} = ijklmn$.
In the following theorem, we will explore how to compress expressions such as this one.
Again, we use disjoint sets of symbols for the inner and outer expression to help us in the formulation and the proof.

\begin{theorem}
    \label{thm:nested_einsum:3}

    For $i \in [m + n]$, let $T^{(i)}$ be an $n_i$-th order tensor with index strings $\bm{s_i} \in S^{n_i}$.
    Let $\bm{s_u}$ be an index string for the $n_u$-th order tensor $U$, which is defined as follows:
    $$U := (\bm{s_{m + 1}},\dots,\bm{s_{m + n}} \rightarrow \bm{s_u}, T^{(m + 1)},\dots,T^{(m + n)})$$
    Also let $\bm{\hat{s}_u}$ be alternative index strings for $U$ with $s_{uj} \neq s_{uj'} \implies \hat{s}_{uj} \neq \hat{s}_{uj'}$ for all $j, j' \in [n_u]$,
    which means that $\bm{\hat{s}_u}$ can only remove symbol duplications, and cannot introduce any.
    Note that this is the converse of the constraint in \cref{thm:nested_einsum:2}.

    In our example, $\bm{s_u} = oopp$ and $\bm{\hat{s}_u} = jklm$.
    This removes the symbol duplication of the first and second index, as well as the symbol duplication of the third and fourth index.

    Let $s_v$ be an index string and
    $$V := (\bm{s_1},\dots,\bm{s_m}, \bm{\hat{s}_u} \rightarrow \bm{s_v}, T^{(1)},\dots,T^{(m)}, U)$$
    where the first and second Einsum expression share no symbols.
    Then these nested Einsum expressions can also be compressed into a single Einsum expression.

    As in \cref{thm:nested_einsum:2}, we need to apply a symbol map before substituting $\bm{\hat{s}_u}$.
    Interestingly, the symbol map is not applied to the index strings of the inner expression ($\bm{s_{m + 1}},\dots,\bm{s_{m + n}}$),
    but to the index strings of the outer expression ($\bm{s_1},\dots,\bm{s_m}$ and $\bm{s_v}$).
    Similarly, it does not map $\bm{s_u}$ to $\bm{\hat{s}_u}$, but $\bm{\hat{s}_u}$ to $\bm{s_u}$.

    Let $\nu: S \rightarrow S$ such that
    $$\nu(s) := \begin{cases}
            s_{uj} & \text{if }\exists j \in [n_u]: \hat{s}_{uj} = s \\
            s      & \text{else}
        \end{cases},$$
    which can be extended to map entire index strings as in \cref{thm:nested_einsum:2}.
    In our example, these are the important mappings:
    \begin{align*}
        i & \rightarrow a, & k & \rightarrow b, & m & \rightarrow c, \\
        j & \rightarrow a, & l & \rightarrow b, & n & \rightarrow c.
    \end{align*}
    This means that $i$ and $j$ will be iterated over at the same time, $k$ and $l$ will be iterated over at the same time, and $m$ and $l$ will be iterated over at the same time.

    Let $\bm{\hat{s}_i} := \nu(\bm{s_i})$ for $i \in [m]$, $\bm{\hat{s}_v} := \nu(\bm{s_v})$, then the compressed Einsum expression is the following:
    $$V = (\bm{\hat{s}_1},\dots,\bm{\hat{s}_m}, \bm{s_{m + 1}}, \dots, \bm{s_{m + n}} \rightarrow \bm{\hat{s}_v}, T^{(1)},\dots,T^{(m + n)})$$
    which helps us to compress the example:
    \begin{gather*}
        (ij, kl, mn, ijklmn \rightarrow ijk, A, B, C, (abc \rightarrow aabbcc, D))\\
        = (aa, bb, cc, abc \rightarrow aab, A, B, C, D).
    \end{gather*}
    Note how the index string for the output $\bm{s_v}$ was changed into $\bm{\hat{s}_v}$.
    This will become apparent in the proof.
\end{theorem}

\bigskip
\begin{proof}
    \small
    The key idea behind this proof, is that the entries of $U$, which were not defined in the computation, are set to the additive neutral element $\0$.
    This is useful, because in a semiring over some set $M$, the additive neutral element \textit{annihilates} $M$.
    This means, that for any $a \in M$, $a \cdot \0 = \0 \cdot a = \0$.
    Therefore, for any multi-index where $U$ is set to $\0$, $V$ is also set to $\0$.
    This means, that in the computation of $V$, only the indices which respect the duplications in $\bm{s_u}$ are defined.
    % Let us prove this formally.

    Let $F, F', B, B'$ be the free and bound symbols of the outer and inner Einsum expression respectively.
    W.l.o.g. they are all non-empty.
    From them we can derive the multi-index spaces $\mathcal{F}, \mathcal{F}', \mathcal{B}, \mathcal{B}'$ as in the definition.
    Then $U_{(\bm{f}, \bm{b}): \bm{\hat{s}_u}}$ is only non-zero for multi-indices $(\bm{f}, \bm{b}) \in \mathcal{F} \times \mathcal{B}$ with $(\bm{f}, \bm{b}):\hat{s}_{uj} = (\bm{f}, \bm{b}):\hat{s}_{uj'}$, where $j,j' \in [n_u]$ are indices of $\bm{s_u}$ where the symbols are duplicated, i.e. $s_{uj} = s_{uj'}$.
    In our example, this means that $(op \rightarrow oopp, B)$ is only non-zero for $(j,k,l,m) \in [d_j] \times [d_k] \times [d_l] \times [d_m]$ with $j = k$ and $l = m$, because $s_{u1} = s_{u2} = o$ and $s_{u3} = s_{u4} = p$.

    Therefore, when $U$ is multiplied with the other tensors, the resulting entry
    $$\bigodot\limits_{i = 1}^{m} T^{(i)}_{(\bm{f}, \bm{b}): \bm{s_i}} \odot U_{(\bm{f}, \bm{b}): \bm{\hat{s}_u}}$$
    is only non-zero for multi-indices $(\bm{f}, \bm{b}) \in \mathcal{F} \times \mathcal{B}$ that respect the same conditions.
    In our example, this is equivalent to
    $$A_{ij} B_{kl} C_{mn} \odot U_{ijklmn} = \begin{cases}
            A_{ij} B_{kl} C_{mn} \odot U_{ijklmn} & \text{if } i = j, k = l, m = n \\
            \0                                    & \text{else}
        \end{cases}.$$

    Now, this already looks like not all symbols are needed for this computation.
    But to see, in which way we can replace symbols, we need to consider the three ways in which duplications can be broken.
    Either a duplication is broken only by free symbols, only by bound symbols, or by a combination of both.
    In our example, we have all of these cases.
    The duplication $aa$ is broken by $i$ and $j$, which are both free symbols.
    The duplication $cc$ is broken by $m$ and $n$, which are both bound symbols.
    The duplication $bb$ is broken by $k$ and $l$, where $k$ is a free symbol and $l$ is a bound symbol.
    Every one of these cases leads to the same result, but in a slightly different way.

    First let us consider the case where a duplication is broken only by free symbols.
    In this case, the free symbols that break the duplication can be replaced by a single symbol,
    because all entries of $V$, with a multi-index that does not respect the duplication, is $\0$.
    In our example, this is equivalent to replacing $i$ and $j$ by a single symbol $a$:
    \begin{align*}
        \forall i,j,k: V_{ijk}    & = \bigoplus\limits_{l,m,n} A_{ij} B_{kl} C_{mn} \odot U_{ijklmn} \\
        \iff \forall a,k: V_{aak} & = \bigoplus\limits_{l,m,n} A_{aa} B_{kl} C_{mn} \odot U_{aaklmn}
    \end{align*}

    For the next two cases, we need to use that $a \oplus \0 = a$ for any $a \in M$.
    % This helps us to reduce the number of symbols that are used in the summation.
    This means, that only those summands that respect the duplications will be summed over,
    because all summands which do not respect the summation are $\0$.
    This affects the remaining two cases in different ways.
    If a duplication is broken only by bound symbols, then we need a single symbol to sum over all the multi-indices that respect the duplication.
    In our example, this is equivalent to replacing $m$ and $n$ by a single symbol $c$:
    $$\bigoplus\limits_{l,m,n} A_{aa} B_{kl} C_{mn} \odot U_{aaklmn} = \bigoplus\limits_{l,c} A_{aa} B_{kl} C_{cc} \odot U_{aaklcc}$$

    Now, if a duplication is broken by free symbols and by bound symbols,
    then the free symbols can again be replaced by a single symbol, as in the first case.
    In our example, this is useless, because there is already only one symbol $k$ where this can be applied.
    Let us do it anyway, because it might clear up what needs to be done in this step.
    \begin{align*}
        \forall a,k: V_{aak}      & = \bigoplus\limits_{l,c} A_{aa} B_{kl} C_{cc} \odot U_{aaklcc} \\
        \iff \forall a,b: V_{aab} & = \bigoplus\limits_{l,c} A_{aa} B_{bl} C_{cc} \odot U_{aablcc}
    \end{align*}
    Then, the values held by the breaking bound symbols are already defined by the value held by the now only breaking free symbol.
    Therefore, the summation over the values of this symbol is useless, because there is only one combination of indices, which respects the duplication.
    Therefore, the breaking bound symbols need to hold the same exact value as the breaking free symbol,
    and these symbols can be replaced by the same symbol, that was used to replace the free symbols.
    Because of this, the occurence of the symbol also needs to be removed from the sum.
    In our example, this is equivalent to replacing $l$ by $b$ and removing it from the sum:
    \begin{align*}
        \forall a,b: V_{aab} & = \bigoplus\limits_{l,c} A_{aa} B_{bl} C_{cc} \odot U_{aablcc} \\
                             & = \bigoplus\limits_{c} A_{aa} B_{bb} C_{cc} \odot U_{aabbcc}
    \end{align*}

    Therefore, in all three cases, the symbols, that break a duplication, can simply be replaced by a single symbol.
    Conveniently, as a replacing symbol, we can just use the symbol that defined the duplication in the first place,
    because the inner expression shares no symbols with the outer expression.
    This yields exactly the symbol map $\nu$.
    Therefore $\bm{\hat{s}_u}$, which is the index string of $U$ that was used in the outer expression, will be replaced by $\bm{s_u}$, which is the index string of $U$ that was used in the inner expression.
    Just as convenient is, that replacing the breaking symbols that include a combination of free and bound symbols, already removes the breaking bound symbols from the sum,
    because the bound symbols are by definition only those symbols, which are not free.
    In our example, this means that replacing $k$ and $l$ by $b$ already removes $l$ from the sum, because $b$ is now a free symbol as well.

    For the final steps, we need to define new multi-index sets to iterate and sum over.
    For this, let $\hat{F} := \sigma(\bm{\hat{s}_v})$ and $\hat{B} := \left(\bigcup_{i \in [m]} \sigma(\bm{\hat{s}_i}) \cup \bm{s_u}\right) \setminus \sigma(\bm{\hat{s}_v})$.
    Let $\mathcal{\hat{F}} = \prod_{s \in \hat{F}} [d_s]$ and $\mathcal{\hat{B}} = \prod_{s \in \hat{B}} [d_s]$.
    Then
    \begin{align*}
        V                                                                                 & = (\bm{s_1},\dots,\bm{s_m}, \bm{\hat{s}_u} \rightarrow \bm{s_v}, T^{(1)},\dots,T^{(m)}, U)                                                                                                \\
        \iff \forall \bm{f} \in \mathcal{F}: V_{\bm{f}: \bm{s_v}}                         & = \bigoplus\limits_{\bm{b} \in \mathcal{B}} \bigodot\limits_{i = 1}^{m} T^{(i)}_{(\bm{f}, \bm{b}):\bm{s_i}} \odot U_{(\bm{f}, \bm{b}):\bm{\hat{s}_u}}                                     \\
        \iff \forall \bm{\hat{f}} \in \mathcal{\hat{F}}: V_{\bm{\hat{f}}: \bm{\hat{s}_v}} & = \bigoplus\limits_{\bm{\hat{b}} \in \mathcal{\hat{B}}} \bigodot\limits_{i = 1}^{m} T^{(i)}_{(\bm{\hat{f}}, \bm{\hat{b}}):\bm{\hat{s}_i}} \odot U_{(\bm{\hat{f}}, \bm{\hat{b}}):\bm{s_u}} \\
        \iff V                                                                            & = (\bm{\hat{s}_1},\dots,\bm{\hat{s}_m}, \bm{s_u} \rightarrow \bm{\hat{s}_v}, T^{(1)},\dots,T^{(m)}, U)
    \end{align*}
    where the second equivalence holds because of the previously discussed symbol replacements.
    Then we can use \cref{thm:nested_einsum:1} for
    $$V = (\bm{\hat{s}_1}, \dots, \bm{\hat{s}_m}, \bm{s_{m + 1}}, \dots, \bm{s_{m + n}} \rightarrow \bm{\hat{s}_v}, T^{(1)}, \dots, T^{(m + n)})$$
    because the symbols of the outer expression have not been mapped to any of the bound symbols of the inner expression.
\end{proof}
\bigskip
