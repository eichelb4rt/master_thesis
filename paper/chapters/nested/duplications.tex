\section{Duplications}

The following two theorems revolve around duplicated symbols in index strings and the way these duplications are \textit{broken}.
We speak of a broken duplication in $\bm{s_u}$, if the symbols $s_{ui}$ and $s_{uj}$ at two positions $i,j \in [n_u]$ are the same, meaning $s_{ui} = s_{uj}$,
but symbols at the same positions in $\bm{\hat{s}_u}$, $\hat{s}_{ui}$ and $\hat{s}_{uj}$ are different, meaning $\hat{s}_{ui} \neq \hat{s}_{uj}$.
In the same way, duplications in $\bm{\hat{s}_u}$ can be broken by $\bm{s_u}$.

Because $\bm{s_u}$ and $\bm{\hat{s}_u}$ are easily confused, we can also use a different terminology that does not include these similar variable names.
In the following, we only refer to the changes the outer expression applies to the duplications in the inner expression.
\begin{itemize}
    \item If $\bm{s_u}$ breaks duplications in $\bm{\hat{s}_u}$,
          the outer expression uses duplications in the input string for the inner expression, that were not used in the output string of the inner expression.
          Therefore, we say that the outer expression \textit{introduces} duplications to the inner expression.
    \item If $\bm{\hat{s}_u}$ breaks duplications in $\bm{s_u}$,
          the outer expression uses different symbols in the input string for the inner expression, where there was originally a duplication in the output string of the inner expression.
          Therefore, we say that the outer expression \textit{removes} duplications from the inner expression.
\end{itemize}

\subsection{Introducing Duplications}

The first duplication theorem handles all nested expressions, where the outer expression can only introduce duplications and cannot remove any.
This means that if two positions hold the same symbol in $\bm{s_u}$, these positions also have to hold the same symbol in $\bm{\hat{s}_u}$.
This brings the advantage, that the symbol map is much simpler and only has to be applied to the inner expression.

The following example is an expression, which respects this condition:
$$(ij, jjj \rightarrow i, A, (kl, lo \rightarrow kko, B, C))$$
for $A \in \R^{a \times b}, B \in \R^{b \times c}, C \in \R^{c \times b}$.
Again, we use disjoint sets of symbols for the inner and outer expression to help us in the formulation and the proof.

\begin{theorem}
    \label{thm:nested_einsum:introduce_duplications}

    For $i \in [m + n]$, let $T^{(i)}$ be an $n_i$-th order tensor with index strings $\bm{s_i} \in S^{n_i}$.
    Let $\bm{s_u}$ be an index string for the $n_u$-th order tensor $U$, which is defined as follows:
    $$U = (\bm{s_{m + 1}},\dots,\bm{s_{m + n}} \rightarrow \bm{s_u}, T^{(m + 1)},\dots,T^{(m + n)})$$
    Also let $\bm{\hat{s}_u}$ be alternative index strings for $U$ with $s_{uj} = s_{uj'} \implies \hat{s}_{uj} = \hat{s}_{uj'}$ for all $j, j' \in [n_u]$,
    which means that the outer expression can only introduce new symbol duplications, and cannot remove any.

    Let $s_v$ be an index string and
    $$V = (\bm{s_1},\dots,\bm{s_m}, \bm{\hat{s}_u} \rightarrow \bm{s_v}, T^{(1)},\dots,T^{(m)}, U)$$
    such that the first and second Einsum expression share no symbols.
    Then these nested Einsum expressions can also be compressed into a single Einsum expression.

    As in \cref{thm:nested_einsum:general}, we need to apply a symbol map in order to compress the nested expression.
    Let $\nu: S \rightarrow S$ such that
    $$\nu(s) := \begin{cases}
            \hat{s}_{uj} & \text{if $\exists j \in [n_u]: s_{uj} = s$}, \\
            s            & \text{else},
        \end{cases}$$
    which maps symbols in $\bm{s_u}$ to the symbol at the same index in $\bm{\hat{s}_u}$ and all other symbols to themselves.

    In our example, we have the following symbols on the same positions:
    \begin{itemize}
        \item $s_{u1} = k$ and $\hat{s}_{u1} = j$,
        \item $s_{u2} = k$ and $\hat{s}_{u2} = j$,
        \item $s_{u3} = o$ and $\hat{s}_{u3} = j$.
    \end{itemize}
    Therefore these are the important mappings:
    \begin{align*}
        k & \rightarrow j, \\
        o & \rightarrow j.
    \end{align*}

    The symbol map $\nu$ can be extended, such that it maps entire index strings instead of just symbols as in \cref{thm:nested_einsum:general}.
    Then we can write the substituted index strings by setting $\bm{\hat{s}_i} := \nu(\bm{s_i})$ for $i \in [m + 1, m + n]$.
    With these index strings, the compressed Einsum expression is the following:
    $$V = (\bm{s_1},\dots,\bm{s_m}, \bm{\hat{s}_{m + 1}}, \dots, \bm{\hat{s}_{m + n}} \rightarrow \bm{s_v}, T^{(1)},\dots,T^{(m + n)})$$
    which helps us to compress the example:
    $$(ij, jjj \rightarrow i, A, (kl, lo \rightarrow kko, B, C)) = (ij, jl, lj \rightarrow i, A, B, C).$$
\end{theorem}

\begin{proof}
    \small
    Because $s_{uj} = s_{uj'} \implies \hat{s}_{uj} = \hat{s}_{uj'}$ for all $j, j' \in [n_u]$,
    each symbol in $\bm{s_u}$ only has one neighbour in the symbol graph.
    This is illustrated for our example in \cref{fig:nested_expressions:example_symbol_graph_intro}
    Therefore each symbol in $\bm{\hat{s}_u}$ defines its own component,
    and can be used as the symbol in the symbol map, that replaces all symbols in their component.
    Therefore the symbols in $\bm{\hat{s}_u}$ are not changed by applying the symbol map, which are the only symbols in the outer expression that could have been changed by the map.
    Therefore the symbol map only has to be applied to the inner expression.

    \begin{figure}[h]
        \centering
        \begin{tikzpicture}[node distance = 2cm, semithick]

            \node[state] (k)		            {$k$};
            \node[state] (o) [right of=k] 		{$o$};
            \node (middle) at ($(k)!0.5!(o)$)   {};
            \node[state] (j) [above of=middle]	{$j$};

            \path (j) edge (k);
            \path (j) edge (o);

        \end{tikzpicture}
        \caption{Symbol graph for the example}
        \label{fig:nested_expressions:example_symbol_graph_intro}
    \end{figure}
\end{proof}
\bigskip

This theorem suffices to prove a property of the trace in a relatively simple manner, namely that for $A \in \R^{m \times n}, B \in \R^{n \times m}$,
it holds that
$$\text{trace}(A \cdot B) = \text{trace}(B \cdot A).$$

\begin{proof}
    \small
    \begin{align*}
        \text{trace}(A \cdot B) & = (ll \rightarrow , (ik,kj \rightarrow ij, A, B))  \\
                                & = (lk, kl \rightarrow ,A, B)                       \\
                                & = (kl, lk \rightarrow ,B, A)                       \\
                                & = (kk \rightarrow , (il, lj \rightarrow ij, B, A)) \\
                                & = \text{trace}(B \cdot A)
    \end{align*}
    where the second and fourth equality hold because of \cref{thm:nested_einsum:introduce_duplications},
    and the third equality holds because of the commutativity of multiplication in the standard semiring.
\end{proof}
\bigskip

\section{Removing Duplications}

The second duplication theorem handles all nested expressions, where the outer expression can only remove duplications and cannot remove any.
This means that if two positions hold the same symbol in $\bm{\hat{s}_u}$, these positions also have to hold the same symbol in $\bm{s_u}$.
This brings the advantage, that the symbol map is much simpler and only has to be applied to the outer expression.

The following example is an expression, which respects this condition:
$$(ij, kl, mn, ijklmn \rightarrow ijk, A, B, C, (abc \rightarrow aabbcc, D))$$
for $A \in \R^{x \times x}, B \in \R^{y \times y}, C \in \R^{z \times z}, D \in \R^{x \times y \times z}$.
This is because the duplications $aa$, $bb$, and $cc$ in the output string $\bm{s_u}$ are broken by $ij$, $kl$, and $mn$ respectively.
Again, we use disjoint sets of symbols for the inner and outer expression to help us in the formulation and the proof.

\begin{theorem}
    \label{thm:nested_einsum:3}

    For $i \in [m + n]$, let $T^{(i)}$ be an $n_i$-th order tensor with index strings $\bm{s_i} \in S^{n_i}$.
    Let $\bm{s_u}$ be an index string for the $n_u$-th order tensor $U$, which is defined as follows:
    $$U = (\bm{s_{m + 1}},\dots,\bm{s_{m + n}} \rightarrow \bm{s_u}, T^{(m + 1)},\dots,T^{(m + n)}).$$
    Also let $\bm{\hat{s}_u}$ be alternative index strings for $U$ with $s_{uj} \neq s_{uj'} \implies \hat{s}_{uj} \neq \hat{s}_{uj'}$ for all $j, j' \in [n_u]$,
    which means that $\bm{\hat{s}_u}$ can only remove symbol duplications, and cannot introduce any.
    Note that this is the converse of the constraint in \cref{thm:nested_einsum:introduce_duplications}.

    Let $s_v$ be an index string and
    $$V = (\bm{s_1},\dots,\bm{s_m}, \bm{\hat{s}_u} \rightarrow \bm{s_v}, T^{(1)},\dots,T^{(m)}, U)$$
    where the first and second Einsum expression share no symbols.
    Then these nested Einsum expressions can also be compressed into a single Einsum expression.

    As in \cref{thm:nested_einsum:general}, we need to apply a symbol map in order to compress the nested expression.
    Let $\nu: S \rightarrow S$ such that
    $$\nu(s) := \begin{cases}
            s_{uj} & \text{if $\exists j \in [n_u]: \hat{s}_{uj} = s$}, \\
            s      & \text{else},
        \end{cases}$$
    which can be extended to map entire index strings as in \cref{thm:nested_einsum:general}.
    In our example, these are the important mappings:
    \begin{align*}
        i & \rightarrow a, & k & \rightarrow b, & m & \rightarrow c, \\
        j & \rightarrow a, & l & \rightarrow b, & n & \rightarrow c.
    \end{align*}
    This means that $i$ and $j$ will be iterated over at the same time, $k$ and $l$ will be iterated over at the same time, and $m$ and $l$ will be iterated over at the same time.

    We can write the substituted index strings by setting $\bm{\hat{s}_i} := \nu(\bm{s_i})$ for $i \in [m]$, $\bm{\hat{s}_v} := \nu(\bm{s_v})$
    With these index strings, the compressed Einsum expression is the following:
    $$V = (\bm{\hat{s}_1},\dots,\bm{\hat{s}_m}, \bm{s_{m + 1}}, \dots, \bm{s_{m + n}} \rightarrow \bm{\hat{s}_v}, T^{(1)},\dots,T^{(m + n)})$$
    which helps us to compress the example:
    \begin{gather*}
        (ij, kl, mn, ijklmn \rightarrow ijk, A, B, C, (abc \rightarrow aabbcc, D))\\
        = (aa, bb, cc, abc \rightarrow aab, A, B, C, D).
    \end{gather*}
\end{theorem}

\begin{proof}
    \small
    Because $\hat{s}_{uj} = \hat{s}_{uj'} \implies s_{uj} = s_{uj'}$ for all $j, j' \in [n_u]$,
    each symbol in $\bm{\hat{s}_u}$ only has one neighbour in the symbol graph.
    This is illustrated for our example in \cref{fig:nested_expressions:example_symbol_graph_remove}
    Therefore each symbol in $\bm{s_u}$ defines its own component,
    and can be used as the symbol in the symbol map, that replaces all symbols in their component.
    Therefore the symbols in $\bm{s_u}$ are not changed by applying the symbol map, which are the only symbols in the inner expression that could have been changed by the map.
    Therefore the symbol map only has to be applied to the outer expression.

    \begin{figure}[h]
        \centering
        \begin{tikzpicture}[node distance = 2cm, semithick]

            \node[state] (i)					{$i$};
            \node[state] (j) [right of=i]		{$j$};
            \node[state] (k) [right of=j]		{$k$};
            \node[state] (l) [right of=k]		{$l$};
            \node[state] (m) [right of=l]		{$m$};
            \node[state] (n) [right of=m]		{$n$};
            \node (ij) at ($(i)!0.5!(j)$) {};
            \node (kl) at ($(k)!0.5!(l)$) {};
            \node (mn) at ($(m)!0.5!(n)$) {};
            \node[state] (a) [below of=ij] 		{$a$};
            \node[state] (b) [below of=kl] 		{$b$};
            \node[state] (c) [below of=mn] 		{$c$};

            \path (a) edge (i);
            \path (a) edge (j);
            \path (b) edge (k);
            \path (b) edge (l);
            \path (c) edge (m);
            \path (c) edge (n);

        \end{tikzpicture}
        \caption{Symbol graph for the example}
        \label{fig:nested_expressions:example_symbol_graph_remove}
    \end{figure}
\end{proof}
\bigskip
