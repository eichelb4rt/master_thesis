\subsection{Removing Duplications}

The second duplication theorem handles all nested expressions, where the outer expression can only remove duplications and cannot introduce any.
This means that if two positions hold the same symbol in $\bm{\hat{s}_u}$, these positions also have to hold the same symbol in $\bm{s_u}$.
This brings the advantage, that the symbol map is much simpler and only has to be applied to the outer expression.

The following example is an expression, which respects this condition:
$$(ij, kl, mn, ijklmn \rightarrow ijk, A, B, C, (abc \rightarrow aabbcc, D))$$
for $A \in \R^{x \times x}, B \in \R^{y \times y}, C \in \R^{z \times z}, D \in \R^{x \times y \times z}$.
This is because the duplications $aa$, $bb$, and $cc$ in the output string $\bm{s_u}$ are broken by $ij$, $kl$, and $mn$ respectively.
Again, we use disjoint sets of symbols for the inner and outer expression to help us in the formulation and the proof.

\begin{theorem}
    \label{thm:nested_einsum:3}

    For $i \in [m + n]$, let $T^{(i)}$ be an $n_i$-th order tensor with index strings $\bm{s_i} \in S^{n_i}$.
    Let $\bm{s_u}$ be an index string for the $n_u$-th order tensor $U$, which is defined as follows:
    $$U = (\bm{s_{m + 1}},\dots,\bm{s_{m + n}} \rightarrow \bm{s_u}, T^{(m + 1)},\dots,T^{(m + n)}).$$
    Also let $\bm{\hat{s}_u}$ be alternative index strings for $U$ with $s_{uj} \neq s_{uj'} \implies \hat{s}_{uj} \neq \hat{s}_{uj'}$ for all $j, j' \in [n_u]$,
    which means that $\bm{\hat{s}_u}$ can only remove symbol duplications, and cannot introduce any.
    Note that this is the converse of the constraint in \cref{thm:nested_einsum:introduce_duplications}.

    Let $s_v$ be an index string and
    $$V = (\bm{s_1},\dots,\bm{s_m}, \bm{\hat{s}_u} \rightarrow \bm{s_v}, T^{(1)},\dots,T^{(m)}, U)$$
    where the first and second Einsum expression share no symbols.
    Then these nested Einsum expressions can also be compressed into a single Einsum expression.

    As in \cref{thm:nested_einsum:general}, we need to apply a symbol map in order to compress the nested expression.
    Let $\nu: S \rightarrow S$ such that
    $$\nu(s) := \begin{cases}
            s_{uj} & \text{if $\exists j \in [n_u]: \hat{s}_{uj} = s$}, \\
            s      & \text{else},
        \end{cases}$$
    which can be extended to map entire index strings as in \cref{thm:nested_einsum:general}.
    In our example, these are the important mappings:
    \begin{align*}
        i & \rightarrow a, & k & \rightarrow b, & m & \rightarrow c, \\
        j & \rightarrow a, & l & \rightarrow b, & n & \rightarrow c.
    \end{align*}
    This means that $i$ and $j$ will be iterated over at the same time, $k$ and $l$ will be iterated over at the same time, and $m$ and $l$ will be iterated over at the same time.

    We can write the substituted index strings by setting $\bm{\hat{s}_i} := \nu(\bm{s_i})$ for $i \in [m]$, $\bm{\hat{s}_v} := \nu(\bm{s_v})$
    With these index strings, the compressed Einsum expression is the following:
    $$V = (\bm{\hat{s}_1},\dots,\bm{\hat{s}_m}, \bm{s_{m + 1}}, \dots, \bm{s_{m + n}} \rightarrow \bm{\hat{s}_v}, T^{(1)},\dots,T^{(m + n)})$$
    which helps us to compress the example:
    \begin{gather*}
        (ij, kl, mn, ijklmn \rightarrow ijk, A, B, C, (abc \rightarrow aabbcc, D))\\
        = (aa, bb, cc, abc \rightarrow aab, A, B, C, D).
    \end{gather*}
\end{theorem}

\begin{proof}
    \small
    Because $\hat{s}_{uj} = \hat{s}_{uj'} \implies s_{uj} = s_{uj'}$ for all $j, j' \in [n_u]$,
    each symbol in $\bm{\hat{s}_u}$ only has one neighbor in the symbol graph.
    This is illustrated for our example in \cref{fig:nested_expressions:example_symbol_graph_remove}
    Therefore, each symbol in $\bm{s_u}$ defines its own component,
    and can be used as the symbol in the symbol map, that replaces all symbols in their component.
    Therefore, the symbols in $\bm{s_u}$ are not changed by applying the symbol map, which are the only symbols in the inner expression that could have been changed by the map.
    Therefore, the symbol map only has to be applied to the outer expression.

    \begin{figure}[h]
        \centering
        \begin{tikzpicture}[node distance = 2cm, semithick]

            \node[state] (i)					{$i$};
            \node[state] (j) [right of=i]		{$j$};
            \node[state] (k) [right of=j]		{$k$};
            \node[state] (l) [right of=k]		{$l$};
            \node[state] (m) [right of=l]		{$m$};
            \node[state] (n) [right of=m]		{$n$};
            \node (ij) at ($(i)!0.5!(j)$) {};
            \node (kl) at ($(k)!0.5!(l)$) {};
            \node (mn) at ($(m)!0.5!(n)$) {};
            \node[state] (a) [below of=ij] 		{$a$};
            \node[state] (b) [below of=kl] 		{$b$};
            \node[state] (c) [below of=mn] 		{$c$};

            \path (a) edge (i);
            \path (a) edge (j);
            \path (b) edge (k);
            \path (b) edge (l);
            \path (c) edge (m);
            \path (c) edge (n);

        \end{tikzpicture}
        \caption{Symbol graph for the example}
        \label{fig:nested_expressions:example_symbol_graph_remove}
    \end{figure}
\end{proof}
\bigskip