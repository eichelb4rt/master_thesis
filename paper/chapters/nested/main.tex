\chapter{Nested Expressions}
\label{chap:nested}

In practice, concatenations of operations come naturally, e.g. computing the squared norm of a matrix-vector product $\abs{A \cdot v}_2^2$
for $A \in \R^{m \times n}, v \in \R^n$.
This would lead to a nested Einsum expression $\abs{A \cdot v}_2^2 = (i,i \rightarrow , (ij, j \rightarrow i, A, v), (ij, j \rightarrow i, A, v))$.
This expression dictates the order of evaluating the expression.
In the example of the norm, the expression $(ij, j \rightarrow i, A, v)$ has to be evaluated before squaring and summing over the results of this computation.

This is limiting, because the order of evaluation might not yield optimal runtime.
This can be seen with a simple matrix-matrix-vector multiplication, which can be written as follows:
$$(A \cdot B) \cdot v = (ij, j \rightarrow i, (ik, kj \rightarrow ij, A, B), v)$$
which is clearly worse than the optimal contraction order
$$A \cdot (B \cdot v) = (ij, j \rightarrow i, A, (ij, j \rightarrow i, B, v))$$
for $A \in \R^{m \times r}, B \in \R^{r \times n}, v \in \R^n$.

But fortunately, all nested Einsum expressions can be compressed into a single Einsum expression, if they are computed over the same semiring.
This leaves the path of contraction up to the implementation.
In the following theorems, we assume that the computations are all over the same semiring $R = (M, \oplus, \odot)$.

\section{Simple Nested Expressions}

\begin{theorem}
    \label{thm:nested_einsum:1}

    For $i \in [m + n]$, let $T^{(i)}$ be an $n_i$-th order tensor with index strings $\bm{s_i} \in S^{n_i}$.
    Let $\bm{s_u}, \bm{s_v}$ be index strings.
    Let
    $$U := (\bm{s_{m + 1}},\dots,\bm{s_{m + n}} \rightarrow \bm{s_u}, T^{(m + 1)},\dots,T^{(m + n)})$$
    and
    $$V := (\bm{s_1},\dots,\bm{s_m}, \bm{s_u} \rightarrow \bm{s_v}, T^{(1)},\dots,T^{(m)}, U)$$
    where the free symbols of the second Einsum expression share no symbols with the first Einsum expression.
    Then
    $$V = (\bm{s_1}, \dots, \bm{s_{m + n}} \rightarrow \bm{s_v}, T^{(1)}, \dots, T^{(m + n)})$$
\end{theorem}

\begin{proof}
    \small
    Let $F, F', B, B'$ be the free and bound symbols of the second (outer) and first (inner) einsum expression respectively.
    W.l.o.g. they are all non-empty.
    From them we can derive $\mathcal{F}, \mathcal{F}', \mathcal{B}, \mathcal{B}'$ as in the definition.
    Then
    \begin{align*}
        V                                                         & = (\bm{s_1},\dots,\bm{s_m}, \bm{s_u} \rightarrow \bm{s_v}, T^{(1)},\dots,T^{(m)}, U)                                                                                                                                                                \\
        \iff \forall \bm{f} \in \mathcal{F}: V_{\bm{f}: \bm{s_v}} & = \bigoplus\limits_{\bm{b} \in \mathcal{B}} \bigodot\limits_{i = 1}^{m} T^{(i)}_{(\bm{f}, \bm{b}):\bm{s_k}} \odot U_{(\bm{f}, \bm{b}):\bm{s_u}}                                                                                                     \\
                                                                  & = \bigoplus\limits_{\bm{b} \in \mathcal{B}} \bigodot\limits_{i = 1}^{m} T^{(i)}_{(\bm{f}, \bm{b}):\bm{s_i}} \odot \bigoplus\limits_{\bm{b'} \in \mathcal{B}'} \bigodot\limits_{i' = m + 1}^{m + n} T^{(i')}_{(\bm{f}, \bm{b}, \bm{b'}):\bm{s_{i'}}} \\
                                                                  & = \bigoplus\limits_{\bm{b} \in \mathcal{B}} \bigoplus\limits_{\bm{b'} \in \mathcal{B}'} \bigodot\limits_{i = 1}^{m} T^{(i)}_{(\bm{f}, \bm{b}):\bm{s_i}} \odot \bigodot\limits_{i = m + 1}^{m + n} T^{(i)}_{(\bm{f}, \bm{b}, \bm{b'}):\bm{s_{i}}}                                      \\
                                                                  & = \bigoplus\limits_{\bm{b} \in \mathcal{B} \times \mathcal{B}'} \bigodot\limits_{i = 1}^{m + n} T^{(i)}_{(\bm{f}, \bm{b}):\bm{s_i}}                                                                                                                               \\
        \iff V                                                    & = (\bm{s_1}, \dots, \bm{s_{m + n}} \rightarrow \bm{s_v}, T^{(1)}, \dots, T^{(m + n)})
    \end{align*}
    where the third equality follows from
    $$\forall \bm{f'} \in \mathcal{F}': U_{\bm{f'}: \bm{s_u}} = \bigoplus\limits_{\bm{b'} \in \mathcal{B}'} \bigodot\limits_{i' = m + 1}^{m + n} T^{(i')}_{(\bm{f'}, \bm{b'}):\bm{s_{i'}}},$$
    $F' \subseteq B \cup F$, and $(B \cup F) \cap B' = \emptyset$. The last two facts are required so that $(\bm{f}, \bm{b}, \bm{b'}):\bm{s_{i'}}$ is well-defined and projects on the same indices as $(\bm{f'}, \bm{b'}):\bm{s_{i'}}$.
    The fourth equality follows from the distributivity in a semiring.
\end{proof}

This means that we can compress all nested Einsum expressions, where the output string of the inner expression, which is used to compute $U$, is exactly the same as the respective input string in the outer expression, when $U$ is used as an input tensor.
This is already helpful for naturally occuring expressions in linear algebra, e.g.
\begin{align*}
    \abs{A \cdot v}_2^2 & = (i,i\rightarrow,(ij, j \rightarrow i, A, v),(ij, j \rightarrow i, A, v)) \\
                        & = (ij,j,ij,j\rightarrow,A,v,A,v)
\end{align*}
for $A \in \R^{m \times n}, v \in \R^{n}$, or
\begin{align*}
    A \cdot B \cdot v & = (ij, j \rightarrow i, (ik, kj \rightarrow ij, A, B), v) \\
                      & = (ik,kj,j \rightarrow i, A, B, v)
\end{align*}
for $A \in \R^{m \times r}, B \in \R^{r \times n}, v \in \R^n$.
But sometimes, we need to access a different multi-index set than the one we computed, e.g.
$$\text{trace}(A \cdot B) = (ii \rightarrow, (ik, kj \rightarrow ij, A, B))$$
or
$$A \cdot \text{diag}(v) = (ik, kj \rightarrow ij, A, (i \rightarrow ii, v))$$
for $A \in \R^{m \times n}, B \in \R^{n \times m}, v \in \R^{n}$.
For this, we need more general ways of compressing nested Einsum expressions.

\section{Introducing Duplications}

In the following theorem, we explore a way of compressing the expression
$$(ij, jjj \rightarrow i, A, (kl, lo \rightarrow kko, B, C))$$
for $A \in \R^{a \times b}, B \in \R^{b \times c}, C \in \R^{c \times b}$.
Note that, for the theorem, we use disjoined sets of symbols for the inner and outer expression.
This helps in the proof, and is not a real constraint in practice,
because we can just rename the symbols in different scopes.
E.g. we could also write the above expression as
$$(ij, jjj \rightarrow i, A, (ik, kj \rightarrow iij, B, C)).$$

\begin{theorem}
    \label{thm:nested_einsum:2}

    For $i \in [m + n]$, let $T^{(i)}$ be an $n_i$-th order tensor with index strings $\bm{s_i} \in S^{n_i}$.
    Let $\bm{s_u}$ be an index string for the $n_u$-th order tensor $U$, which is defined as follows:
    $$U := (\bm{s_{m + 1}},\dots,\bm{s_{m + n}} \rightarrow \bm{s_u}, T^{(m + 1)},\dots,T^{(m + n)})$$
    Also let $\bm{\hat{s}_u}$ be alternative index strings for $U$ with $s_{uj} = s_{uj'} \implies \hat{s}_{uj} = \hat{s}_{uj'}$ for all $j, j' \in [n_u]$,
    which means that $\bm{\hat{s}_u}$ can only introduce new symbol duplications, and cannot remove any.

    In our example, $\bm{s_u} = kko$ and $\bm{\hat{s}_u} = jjj$.
    This does not break the symbol duplication of the first and second index,
    and introduces a new duplication on the third index.

    Let $s_v$ be an index string and
    $$V := (\bm{s_1},\dots,\bm{s_m}, \bm{\hat{s}_u} \rightarrow \bm{s_v}, T^{(1)},\dots,T^{(m)}, U)$$
    such that the first and second Einsum expression share no symbols.
    Then these nested Einsum expressions can also be compressed into a single Einsum expression.

    In contrast to \autoref{thm:nested_einsum:1}, we cannot just replace the input index string $\bm{\hat{s}_u}$ by all the input index strings in the inner Einsum expression $\bm{s_{m + 1}},\dots,\bm{s_{m + n}}$.
    Instead, we first need to apply a symbol map.
    Let $\nu: S \rightarrow S$ such that
    $$\nu(s) := \begin{cases}
            \hat{s}_{uj} & \text{if }\exists j \in [n_u]: s_{uj} = s \\
            s            & \text{else}
        \end{cases}$$
    which maps symbols in $\bm{s_u}$ to the symbol at the same index in $\bm{\hat{s}_u}$ and all other symbols to themselves.

    This symbol map holds information about which symbols will be iterated over at the same time in the outer expression.
    In our example, the interesting parts of the map are $\nu(k) = j$ and $\nu(o) = j$, which means that $k$ and $j$ will be iterated over at the same time.

    $\nu$ can be extended, such that it maps entire index strings instead of just symbols, by setting $\nu(\bm{s_i}) \in S^{n_i}, \nu(\bm{s_i})_j := \nu(s_{ij})$.
    Then we can write the substituted index strings by setting $\bm{\hat{s}_i} := \nu(\bm{s_i})$ for $i \in [m + 1, m + n]$.

    Then the compressed Einsum expression is the following:
    $$V = (\bm{s_1},\dots,\bm{s_m}, \bm{\hat{s}_{m + 1}}, \dots, \bm{\hat{s}_{m + n}} \rightarrow \bm{s_v}, T^{(1)},\dots,T^{(m + n)})$$
    which helps us to compress the example:
    $$(ij, jjj \rightarrow i, A, (kl, lo \rightarrow kko, B, C)) = (ij, jl, lj \rightarrow i, A, B, C)$$
\end{theorem}

\begin{proof}
    \small
    The fundamental idea behind this theorem is, that by using the index string $\bm{\hat{s}_u}$, we only iterate over a sub-space of the indices that we defined for the computation of $U$.
    To formulate this, we need some idea of which multi-indices we iterate over.
    Therefore, let $\mathcal{M}:\bm{s} := \smallset{M: \bm{s} \mid M \in \mathcal{M}}$ for an index string $\bm{s}$ and a multi-index space $\mathcal{M}$.

    Let $\mathcal{K} = \prod_{s \in \sigma(\hat{s}_u)} [d_s]$ be the multi-index space in the input for the computation of $V$,
    and let $\mathcal{K}' = \prod_{s \in \sigma(s_u)} [d_s]$ be the multi-index space in the output for the computation of $U$.
    Then $\mathcal{K}:\bm{\hat{s}_u} \subseteq \mathcal{K}':\bm{s_u}$, because $d_{s_{uj}} = d_{\hat{s}_{uj}}$ per the definition of einsum,
    and because the amount of axes contributing to $\mathcal{K}$ ($\abs{\sigma(\bm{\hat{s}_u})}$) has to be smaller or equal to the amount of axes contributing to $\mathcal{K}'$ ($\abs{\sigma(\bm{s_u})}$).
    This last fact follows from the constraint $s_{uj} = s_{uj'} \implies \hat{s}_{uj} = \hat{s}_{uj'}$.

    Then
    $$\forall K' \in \mathcal{K}': U_{K': \bm{s_u}} = \bigoplus\limits_{J' \in \mathcal{J}'}\bigodot\limits_{i = m + 1}^{m + n} T^{(i)}_{K'J':\bm{s_{i}}}$$
    and therefore
    $$\forall K \in \mathcal{I}': U_{K: \bm{\hat{s}_u}} = \bigoplus\limits_{J' \in \mathcal{J}'}\bigodot\limits_{i = m + 1}^{m + n} T^{(i)}_{KJ':\bm{\hat{s}_{i}}}$$
    because of the previous observation,
    and because the free symbols of the expression, which are used in $J'$, are not changed by the symbol map $\nu$.

    Therefore
    \begin{align*}
        V                                               & = (\bm{s_1},\dots,\bm{s_m}, \bm{\hat{s}_u} \rightarrow \bm{s_v}, T^{(1)},\dots,T^{(m)}, U)                                                                                                                   \\
        \iff \forall I \in \mathcal{I}: V_{I: \bm{s_v}} & = \bigoplus\limits_{J \in \mathcal{J}} \bigodot\limits_{i = 1}^{m} T^{(i)}_{IJ:\bm{s_i}} \odot U_{IJ:\bm{\hat{s}_u}}                                                                                         \\
                                                        & = \bigoplus\limits_{J \in \mathcal{J}} \bigodot\limits_{i = 1}^{m} T^{(i)}_{IJ:\bm{s_i}} \odot \bigoplus\limits_{J' \in \mathcal{J}'} \bigodot\limits_{i' = m + 1}^{m + n} T^{(i')}_{IJJ':\bm{\hat{s}_{i'}}} \\
                                                        & = \bigoplus\limits_{J \in \mathcal{J} \times \mathcal{J}'} \bigodot\limits_{i = 1}^{m} T^{(i)}_{IJ:\bm{s_i}} \odot \bigodot\limits_{i = m + 1}^{m + n} T^{(i)}_{IJ:\bm{\hat{s}_i}}                           \\
        \iff V                                          & = (\bm{s_1}, \dots, \bm{s_m}, \bm{\hat{s}_{m + 1}}, \dots, \bm{\hat{s}_{m + n}} \rightarrow \bm{s_v}, T^{(1)}, \dots, T^{(m + n)})
    \end{align*}
    where the third equality holds because we only iterate over a sub-space of the indices that we defined for the computation of $U$,
    and because the first and second einsum expression share no symbols.
    The rest of the steps are the same as in \autoref{thm:nested_einsum:1}.
\end{proof}

With this theorem, we can prove a property of the trace in a relatively simple manner, namely that for $A \in \R^{m \times n}, B \in \R^{n \times m}$,
it holds that
$$\text{trace}(A \cdot B) = \text{trace}(B \cdot A).$$

\begin{proof}
    \small
    % \begin{align*}
    %     \text{trace}(A \cdot B) & = (ii \rightarrow , (ik,kj \rightarrow ij, A, B)) \\
    %                             & = (ik, ki \rightarrow ,A, B)                      \\
    %                             & = (ki, ik \rightarrow ,A, B)                      \\
    %                             & = (ik, ki \rightarrow ,B, A)                      \\
    %                             & = (ii \rightarrow , (ik,kj \rightarrow ij, B, A)) \\
    %                             & = \text{trace}(B \cdot A)
    % \end{align*}
    % where the second equality holds because of \autoref{thm:nested_einsum:2},
    % the third equality is just a renaming of the indices,
    % and the fourth equality holds because of the commutivity in the used semiring.
    \begin{align*}
        \text{trace}(A \cdot B) & = (ii \rightarrow , (ik,kj \rightarrow ij, A, B)) \\
                                & = (ik, ki \rightarrow ,A, B)                      \\
                                & = (ki, ik \rightarrow ,B, A)                      \\
                                & = \text{trace}(B \cdot A)
    \end{align*}
    where the second equality holds because of \autoref{thm:nested_einsum:2},
    and the third equality holds because of the commutativity of multiplication in the standard semiring.
\end{proof}
\bigskip

This is already a useful tool for compressing nested expressions, but there are still some naturally occuring expressions we cannot compress with this,
e.g.:
$$A \cdot \text{diag}(v) = (ik, kj \rightarrow ij, A, (i \rightarrow ii, v))$$
for $A \in \R^{m \times n}, v \in \R^{n}$.
This is because the symbol duplication $ii$ is broken by the index string $kj$, and therefore we access more entries than the ones we computed.

\section{Removing Duplications}

The following example is an expression, which we cannot compress with the previous theorems:
$$(ijkl, jklm \rightarrow ij, A, (op \rightarrow oopp, B))$$
for $A \in \R^{a \times b \times b \times c}, B \in \R^{b \times c}$.
In the following theorem, we will explore how to compress expressions such as this one.
Again, we use disjoined sets of symbols for the inner and outer expression to help us in the formulation and the proof.

\begin{theorem}
    \label{thm:nested_einsum:3}

    For $i \in [m + n]$, let $T^{(i)}$ be an $n_i$-th order tensor with index strings $\bm{s_i} \in S^{n_i}$.
    Let $\bm{s_u}$ be an index string for the $n_u$-th order tensor $U$, which is defined as follows:
    $$U := (\bm{s_{m + 1}},\dots,\bm{s_{m + n}} \rightarrow \bm{s_u}, T^{(m + 1)},\dots,T^{(m + n)})$$
    Also let $\bm{\hat{s}_u}$ be alternative index strings for $U$ with $s_{uj} \neq s_{uj'} \implies \hat{s}_{uj} \neq \hat{s}_{uj'}$ for all $j, j' \in [n_u]$,
    which means that $\bm{\hat{s}_u}$ can only remove symbol duplications, and cannot introduce any.
    Note that this is the converse of the constraint in \autoref{thm:nested_einsum:2}.

    In our example, $\bm{s_u} = oopp$ and $\bm{\hat{s}_u} = jklm$.
    This removes the symbol duplication of the first and second index, as well as the symbol duplication of the third and fourth index.

    Let $s_v$ be an index string and
    $$V := (\bm{s_1},\dots,\bm{s_m}, \bm{\hat{s}_u} \rightarrow \bm{s_v}, T^{(1)},\dots,T^{(m)}, U)$$
    where the first and second Einsum expression share no symbols.
    Then these nested Einsum expressions can also be compressed into a single Einsum expression.

    As in \autoref{thm:nested_einsum:2}, we need to apply a symbol map before substituting $\bm{\hat{s}_u}$.
    Interestingly, the symbol map is not applied to the index strings in the computation of $U$ ($\bm{s_{m + 1}},\dots,\bm{s_{m + n}}$),
    but to the index strings in the computation of $V$ ($\bm{s_1},\dots,\bm{s_m}$).
    Similarly, it does not map $\bm{s_u}$ to $\bm{\hat{s}_u}$, but $\bm{\hat{s}_u}$ to $\bm{s_u}$.

    Let $\nu: S \rightarrow S$ such that
    $$\nu(s) := \begin{cases}
            s_{uj} & \text{if }\exists j \in [n_u]: \hat{s}_{uj} = s \\
            s      & \text{else}
        \end{cases},$$
    which can be extended to map entire index strings as in \autoref{thm:nested_einsum:2}.
    In our example, these are the important mappings:
    \begin{align*}
        j & \rightarrow o \\
        k & \rightarrow o \\
        l & \rightarrow p \\
        m & \rightarrow p
    \end{align*}
    This means that $j$ and $k$ will be iterated over at the same time, and $l$ and $m$ will be iterated over at the same time.

    Let $\bm{\hat{s}_i} := \nu(\bm{s_i})$ for $i \in [m]$, $\bm{\hat{s}_v} := \nu(\bm{s_v})$, then the compressed Einsum expression is the following:
    $$V = (\bm{\hat{s}_1},\dots,\bm{\hat{s}_m}, \bm{s_{m + 1}}, \dots, \bm{s_{m + n}} \rightarrow \bm{\hat{s}_v}, T^{(1)},\dots,T^{(m + n)})$$
    which helps us to compress the example:
    $$(ijkl, jklm \rightarrow ij, A, (op \rightarrow oopp, B)) = (ioop, op \rightarrow io, A, B)$$
    Note how the index string for the output $\bm{s_v}$ was changed into $\bm{\hat{s}_v}$.
    This will become apparent in the proof.
\end{theorem}

\begin{proof}
    \small
    The key idea behind this proof, is that the entries of $U$, which were not defined in the computation, are set to the additive neutral element $\0$.
    This is useful, because in a semiring over some set $M$, the additive neutral element \textit{annihilates} $M$.
    This means, that for any $a \in M$, $a \cdot \0 = \0 \cdot a = \0$.
    Therefore, for any multi-index where $U$ is set to $\0$, $V$ is also set to $\0$.
    This means, that in the computation of $V$, only the indices which respect the duplications in $\bm{s_u}$ are defined.
    % Let us prove this formally.

    Let $F, F', B, B'$ be the free and bound symbols of the second (outer) and first (inner) einsum expression respectively.
    W.l.o.g. they are all non-empty.
    From them we can derive $\mathcal{F}, \mathcal{F}', \mathcal{B}, \mathcal{B}'$ as in the definition.
    Then $U_{(\bm{f}, \bm{b}): \bm{\hat{s}_u}}$ is only non-zero for multi-indices $(\bm{f}, \bm{b}) \in \mathcal{F} \times \mathcal{B}$ with $(\bm{f}, \bm{b}):\hat{s}_{uj} = (\bm{f}, \bm{b}):\hat{s}_{uj'}$, where $j,j' \in [n_u]$ are indices of $\bm{s_u}$ where the symbols are duplicated, i.e. $s_{uj} = s_{uj'}$.
    In our example, this means that $(op \rightarrow oopp, B)$ is only non-zero for $(j,k,l,m) \in [d_j] \times [d_k] \times [d_l] \times [d_m]$ with $j = k$ and $l = m$, because $s_{u1} = s_{u2} = o$ and $s_{u3} = s_{u4} = p$.

    Therefore, when $U$ is multiplied with the other tensors, the resulting entry
    $$\bigodot\limits_{i = 1}^{m} T^{(i)}_{(\bm{f}, \bm{b}): \bm{s_i}} \odot U_{(\bm{f}, \bm{b}): \bm{\hat{s}_u}}$$
    is only non-zero for multi-indices $(\bm{f}, \bm{b}) \in \mathcal{F} \times \mathcal{B}$ that respect the same conditions.
    Therefore, and because $a \oplus \0 = a$ for any $a \in M$, the summation is reduced to only those summands, which have multi-indices that respect the duplications.
    Therefore we can iterate over the summands in such a way, that only those multi-indices are considered, that respect the duplications.
    For this, we define a different set of bound symbols $\hat{B} = \left(\bigcup_{i \in [m]} \sigma(\bm{\hat{s}_i}) \cup \sigma(\bm{s_u})\right) \setminus \sigma(\bm{\hat{s}_v})$.
    From these bound symbols, we can derive its corresponding multi-index space $\mathcal{\hat{B}} = \prod_{s \in \hat{B}} [d_s]$.
    In our example, the new set of bound symbols is $\hat{B} = \smallset{p}$, and the original set of bound symbols is $B = \set{k, l, m}$.

    Now, in order to use these multi-indices in a well-defined manner together with the multi-indices of the free symbols, we can define an incomplete symbol map $\mu: S \rightarrow S$.
    This symbol map is incomplete because it does not contain all mappings, that are contained in the complete symbol map $\nu$.
    It is restricted on only those symbols, that will be part of the bound symbols after application of the complete map.
    The symbol map also has to produce index strings that are able to project the multi-indices $(\bm{f},\bm{\hat{b}}) \in \mathcal{F} \times \mathcal{\hat{B}}$ in a well-defined manner, which is possible in the first place because $F \cap \hat{B} = \emptyset$.
    Therefore we define
    $$\mu(s) := \begin{cases}
            s_{uj} & \text{if } \exists j: \hat{s}_{uj} = s \text{ and } s_{uj} \in \hat{B} \\
            s      & \text{else}
        \end{cases}.$$
    To indicate the incompletely mapped index strings, we denote $\mu(\bm{s_i})$ as $\bm{\check{s}_i}$.
    In our example, these are the important mappings:
    \begin{align*}
        l & \rightarrow p \\
        m & \rightarrow p
    \end{align*}

    Then
    % TODO: reformulate to fit example, expand example to demonstrate cases where \bm{f} could not respect duplications
    $$\bigoplus\limits_{\bm{b} \in \mathcal{B}} \bigodot\limits_{i = 1}^{m} T^{(i)}_{(\bm{f}, \bm{b}):\bm{s_i}} \odot U_{(\bm{f}, \bm{b}):\bm{\hat{s}_u}}
        = \begin{cases}
            \bigoplus\limits_{\bm{\hat{b}} \in \mathcal{\hat{B}}} \bigodot\limits_{i = 1}^{m} T^{(i)}_{(\bm{f}, \bm{\hat{b}}):\bm{\check{s}_i}} \odot U_{(\bm{f}, \bm{\hat{b}}):\bm{s_u}} & \text{if } \bm{f} \text{ respects the duplications} \\
            \0                                                                                                                                                                            & \text{else}
        \end{cases}$$
    for all $\bm{f} \in \mathcal{F}$.
    When applied to our example, this is equivalent to:
    $$\bigoplus\limits_{k, l, m} A_{ijkl} \odot U_{jklm}
        = \begin{cases}
            \bigoplus\limits_{p} A_{ijkp} \odot U_{jkpp} & \text{if } j = k \\
            \0                                           & \text{else}
        \end{cases}$$
    for all $(i,j) \in [d_i] \times [d_j]$.

    % For this, we define a new multi-index space $\mathcal{\hat{B}} = \left(\bigcup_{i \in [m]} \sigma(\bm{\hat{s}_i}) \cup \sigma(\bm{s_u})\right) \setminus \sigma(\bm{\hat{s}_v})$,
    % and project these multi-indices with index strings $\bm{\hat{s}_i}$ that respect the duplications.
    % % TODO: the projection is not defined, because the new free variables aren't introduced yet.
    % Then
    % $$\bigoplus\limits_{\bm{b} \in \mathcal{B}} \bigodot\limits_{i = 1}^{m} T^{(i)}_{(\bm{f}, \bm{b}):\bm{s_i}} \odot U_{(\bm{f}, \bm{b}):\bm{\hat{s}_u}}
    %     = \begin{cases}
    %         \bigoplus\limits_{\bm{\hat{b}} \in \mathcal{\hat{B}}} \bigodot\limits_{i = 1}^{m} T^{(i)}_{(\bm{f}, \bm{\hat{b}}):\bm{\hat{s}_i}} \odot U_{(\bm{f}, \bm{\hat{b}}):\bm{s_u}} & \text{if } \bm{f} \text{ respects the duplications} \\
    %         \0                                                                                                                                                                          & \text{else}
    %     \end{cases}$$
    % for all $\bm{f} \in \mathcal{F}$.

    Now, because all $\bm{f} \in \mathcal{F}$ that do not respect the duplications are $\0$, we can also iterate over the free variables in such a way, that only those multi-indices are considered, that respect the duplications.
    For this, we define a new multi-index space $\mathcal{\hat{F}} = \sigma(\bm{\hat{s}_v})$,
    and project these multi-indices with index strings $\bm{\hat{s}_i}$ and $\bm{\hat{s}_v}$ that respect the duplications. Then
    \begin{align*}
        V                                                                                 & = (\bm{s_1},\dots,\bm{s_m}, \bm{\hat{s}_u} \rightarrow \bm{s_v}, T^{(1)},\dots,T^{(m)}, U)                                                                                                                                                                                                                 \\
        \iff \forall \bm{f} \in \mathcal{F}: V_{\bm{f}: \bm{s_v}}                         & = \bigoplus\limits_{\bm{b} \in \mathcal{B}} \bigodot\limits_{i = 1}^{m} T^{(i)}_{(\bm{f}, \bm{b}):\bm{s_i}} \odot U_{(\bm{f}, \bm{b}):\bm{\hat{s}_u}}                                                                                                                                                      \\
        \iff \forall \bm{\hat{f}} \in \mathcal{\hat{F}}: V_{\bm{\hat{f}}: \bm{\hat{s}_v}} & = \bigoplus\limits_{\bm{\hat{b}} \in \mathcal{\hat{B}}} \bigodot\limits_{i = 1}^{m} T^{(i)}_{(\bm{\hat{f}}, \bm{\hat{b}}):\bm{\hat{s}_i}} \odot U_{(\bm{\hat{f}}, \bm{\hat{b}}):\bm{s_u}}                                                                                                                  \\
                                                                                          & = \bigoplus\limits_{\bm{\hat{b}} \in \mathcal{\hat{B}}} \bigodot\limits_{i = 1}^{m} T^{(i)}_{(\bm{\hat{f}}, \bm{\hat{b}}):\bm{\hat{s}_i}} \odot \left[\bigoplus\limits_{\bm{b'} \in \mathcal{B}'} \bigodot\limits_{i' = m + 1}^{m + n} T^{(i')}_{(\bm{\hat{f}}, \bm{\hat{b}}, \bm{b'}):\bm{s_{i'}}}\right] \\
                                                                                          & = \bigoplus\limits_{\bm{\hat{b}} \in \mathcal{\hat{B}} \times \mathcal{B}'} \bigodot\limits_{i = 1}^{m} T^{(i)}_{(\bm{\hat{f}}, \bm{\hat{b}}):\bm{\hat{s}_i}} \odot T^{(i)}_{(\bm{\hat{f}}, \bm{\hat{b}}):\bm{s_i}}                                                                                        \\
        \iff V                                                                            & = (\bm{\hat{s}_1}, \dots, \bm{\hat{s}_m}, \bm{s_{m + 1}}, \dots, \bm{s_{m + n}} \rightarrow \bm{\hat{s}_v}, T^{(1)}, \dots, T^{(m + n)})
    \end{align*}
    % TODO: expand
    TODO: expand?
\end{proof}
\bigskip

% TODO: can i really write "every"?
With these theorems, we can write every naturally occuring complex expression from linear algebra as a single Einsum expression.
The reason for this is, that in linear algebra, only up to two indices are used for a single tensor,
which means that with two index strings, it cannot happen that a duplication is removed and anonther duplication is introduced simultaniously.
Here are some more complex expressions as examples:
\begin{itemize}
    \item squared norm of matrix-vector multiplication: Let $A \in \R^{m \times n}, v \in \R^{n}$. Then
          \begin{align*}
              \abs{A \cdot v}_2^2 & = (i,i\rightarrow,(ij, j \rightarrow i, A, v),(ij, j \rightarrow i, A, v)) \\
                                  & = (ij,j,ij,j\rightarrow,A,v,A,v)
          \end{align*}
    \item trace of matrix-matrix multiplication: Let $A \in \R^{m \times n}, B \in \R^{n \times m}$. Then
          \begin{align*}
              \text{trace}(A \cdot B) & = (ii \rightarrow, (ik, kj \rightarrow ij, A, B)) \\
                                      & = (ik, ki \rightarrow, A, B)
          \end{align*}
    \item matrix multiplication with a diagonal matrix: Let $A \in \R^{m \times n}, v \in \R^{n}$. Then
          \begin{align*}
              A \cdot \text{diag}(v) & = (ik, kj \rightarrow ij, A, (i \rightarrow ii, v)) \\
                                     & = (ij, j \rightarrow ij, A, v)                      \\
          \end{align*}
\end{itemize}

But to write every expression from linear algebra as a single Einsum expression respectively was already possible before, with (TODO: quote Julien).
% There, (beschreib, was Julien gemacht hat).
With these theorems, we just derived a different way of achieving that.
For this, we can state a very simple procedure.
First, every function is translated to their respective Einsum expression, which results in a nested einsum expression.
Then, the nested expressions are compressed from the bottom up.

But we still do not have a way of compressing general nested Einsum expressions, where duplications might be removed and introduced simultaniously.
If we could do that, then we could build a compiler that is able to compress every possible nested Einsum expression, regardless of duplications.

\section{General Nested Expressions}

In the final generalisation of compressing nested Einsum expressions, all duplication breaking is allowed.
This has no application for linear algebra, as all the previous theorems had, because this first comes into play with third order tensors.
This is because, with two or less axes, there is no possibility of simultaniously breaking a duplication and introducing a new one.
Nevertheless, it serves as a useful tool for compressing all kinds of nested Einsum expressions.

In the following theorem, we explore a way of compressing the expression
% $$(ijk, jk \rightarrow ijk, A, (l \rightarrow ll, v))$$
$$(a,b,c,d,e,abbde \rightarrow bc, v^{(1)}, v^{(2)}, v^{(3)}, v^{(4)}, v^{(5)}, (
    i,j,k,l \rightarrow iijkkl, v^{(6)}, v^{(7)}, v^{(8)}, v^{(9)}
    ))$$
for $v^{(i)} \in \R^{d_{vi}}$ with $i \in [9]$, $\bm{d_v} = (f, f, g, g, h, f, f, g, h) \in \N^{9}$.
Again, we use disjoined sets of symbols for the inner and outer expression to help us in the formulation and the proof.

\begin{theorem}
    \label{thm:nested_einsum:4}
    For $i \in [m + n]$, let $T^{(i)}$ be an $n_i$-th order tensor with index strings $\bm{s_i} \in S^{n_i}$.
    Let $\bm{s_u}$ be an index string for the $n_u$-th order tensor $U$, which is defined as follows:
    $$U := (\bm{s_{m + 1}},\dots,\bm{s_{m + n}} \rightarrow \bm{s_u}, T^{(m + 1)},\dots,T^{(m + n)})$$
    Also let $\bm{\hat{s}_u}$ be alternative index strings for $U$.

    Let $s_v$ be an index string and
    $$V := (\bm{s_1},\dots,\bm{s_m}, \bm{\hat{s}_u} \rightarrow \bm{s_v}, T^{(1)},\dots,T^{(m)}, U)$$
    where the first and second Einsum expression share no symbols.
    Then these nested Einsum expressions can also be compressed into a single Einsum expression.

    Once again, a map $\omega: S \rightarrow S$ has to be applied to the index strings before substituting index strings.
    The definition of the map this time is somewhat more complex.
    As in the previous two theorems, this map holds information about which symbols are essentially used together as one index.

    For the definition of the map $\omega$, we first construct an undirected graph $G = (V, E)$ that we call \textit{symbol graph}.
    % $V = \sigma(\bm{s_v}) \cup \sigma(\bm{s_u}) \cup \sigma(\bm{\hat{s}_u}) \cup \bigcup_{i \in [m + n]} \sigma(\bm{s_i})$
    In the symbol graph, the nodes consist of all symbols from both expressions.
    The edges are $E = \smallset{\smallset{s_{uj}, \hat{s}_{uj}} \mid j \in [n_u]}$,
    which connects all symbols from $\bm{s_u}$ and $\bm{\hat{s}_u}$ that share an index.
    The symbol graph for our example is displayed in \autoref{fig:nested_expressions:example_symbol_graph}.

    In the symbol graph, if two symbols are connected, then they need to be iterated over at the same time in the compressed expression, because they are essentially the same index.
    Therefore, it makes sense assigning a symbol $s_C \in S \setminus V$ to each of the graphs components $C$.
    Then we can define $\omega$ as follows:
    $$\omega(s) := \begin{cases}
            s_C & \text{if } s \in C \\
            s   & \text{else}
        \end{cases}$$
    In our example, the components are $\smallset{a,b,i,j}$, $\smallset{c,d,k}$, and $\smallset{e,l}$.
    Therefore we could use
    $$\omega(s) := \begin{cases}
            x & \text{if } s \in \smallset{a,b,i,j} \\
            y & \text{if } s \in \smallset{c,d,k}   \\
            z & \text{if } s \in \smallset{e,l}     \\
            s & \text{else}
        \end{cases}$$

    \begin{figure}[h]
        \centering
        \begin{tikzpicture}[node distance = 3cm, semithick]

            \node[state] (a)					{$a$};
            \node[state] (b) [right of=a]		{$b$};
            \node[state] (c) [right of=b] 		{$c$};
            \node[state] (d) [right of=c] 		{$d$};
            \node[state] (e) [right of=d] 		{$e$};
            \node (middle) at ($(a)!0.5!(b)$) {};
            \node[state] (i) [below of=middle] 		{$i$};
            \node[state] (j) [right of=i] 		{$j$};
            \node[state] (k) [right of=j] 		{$k$};
            \node[state] (l) [right of=k] 		{$l$};

            \node (x_center) at ($(a)!0.5!(j)$) {};
            \node (y_center) at ($(c)!0.5!(d|-k)$) {};
            \node (z_center) at ($(e)!0.5!(l)$) {};
            \node (x) [below of=x_center] {$x$};
            \node (y) [below of=y_center] {$y$};
            \node (z) [below of=z_center] {$z$};

            \path (a) edge (i);
            \path (i) edge (b);
            \path (b) edge (j);
            \path (c) edge (k);
            \path (k) edge (d);
            \path (e) edge (l);

            \begin{pgfonlayer}{background}
                \draw[gray!15, fill=gray!15, line width=12mm, line cap=round, line join=round] (a.center) -- (b.center) -- (j.center) -- (i.center) -- cycle;
                \draw[gray!15, fill=gray!15, line width=12mm, line cap=round, line join=round] (c.center) -- (d.center) -- (k.center) -- cycle;
                \draw[gray!15, fill=gray!15, line width=12mm, line cap=round, line join=round] (e.center) -- (l.center) -- cycle;
            \end{pgfonlayer}

        \end{tikzpicture}
        \caption{Symbol graph for the example}
        \label{fig:nested_expressions:example_symbol_graph}
    \end{figure}

    $\omega$ can be extended in a similar way as $\mu, \nu$ to map entire index strings.
    In this general form, the this map is applied to all index strings from both expressions before the substitution.
    Let $\bm{\hat{s}_i} := \omega(\bm{s_i})$ for $i \in [m + n]$, $\bm{\hat{s}_v} := \omega(\bm{s_v})$, then the compressed Einsum expression is the following:
    $$V = (\bm{\hat{s}_1}, \dots, \bm{\hat{s}_{m + n}} \rightarrow \bm{\hat{s}_v}, T^{(1)},\dots,T^{(m + n)})$$
    which helps us to compress the example:
    \begin{gather*}
        (a,b,c,d,e,abbde \rightarrow bc, v^{(1)}, v^{(2)}, v^{(3)}, v^{(4)}, v^{(5)}, (
        i,j,k,l \rightarrow iijkkl, v^{(6)}, v^{(7)}, v^{(8)}, v^{(9)}
        ))\\
        =(x,x,y,y,z,x,x,y,z \rightarrow xy, v^{(1)}, v^{(2)}, v^{(3)}, v^{(4)}, v^{(5)}, v^{(6)}, v^{(7)}, v^{(8)}, v^{(9)})
    \end{gather*}
\end{theorem}

\begin{proof}
    \small
    Wird spannend.
\end{proof}
