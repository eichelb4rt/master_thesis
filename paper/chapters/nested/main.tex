\chapter{Nested Expressions}
\label{chap:nested}

In practice, concatenations of operations arise naturally, e.g. computing the squared norm of a matrix-vector product $\abs{A \cdot v}_2^2$
for $A \in \R^{m \times n}, v \in \R^n$.
This would lead to a nested Einsum expression $\abs{A \cdot v}_2^2 = (i,i \rightarrow , (ij, j \rightarrow i, A, v), (ij, j \rightarrow i, A, v))$.
This expression dictates the order of evaluating the expression.
In the example of the norm, the expression $(ij, j \rightarrow i, A, v)$ has to be evaluated before squaring and summing over the results of this computation.

This is limiting, because the order of evaluation might not yield optimal runtime.
This can be seen with a simple matrix-matrix-vector multiplication, which can be written as follows:
$$(A \cdot B) \cdot v = (ij, j \rightarrow i, (ik, kj \rightarrow ij, A, B), v)$$
which is clearly worse than the optimal contraction order
$$A \cdot (B \cdot v) = (ij, j \rightarrow i, A, (ij, j \rightarrow i, B, v))$$
for $A \in \R^{m \times r}, B \in \R^{r \times n}, v \in \R^n$.
Another limitation of nested Einsum expressions is that we can not fully benefit from the computational advantages that come with the optimization of single Einsum expressions.

But fortunately, all nested Einsum expressions can be compressed into a single Einsum expression, if they are computed over the same semiring.
For instance,
\begin{align*}
    \abs{A \cdot v}_2^2 & = (i,i \rightarrow , (ij, j \rightarrow i, A, v), (ij, j \rightarrow i, A, v)) \\
                        & = (ij, j, ij, j \rightarrow, A, v, A, v)
\end{align*}
and
\begin{align*}
    (A \cdot B) \cdot v & = (ij, j \rightarrow i, (ik, kj \rightarrow ij, A, B), v) \\
                        & = (ik, kj, j \rightarrow i, A, B, v).
\end{align*}
This leaves the path of contraction up to the implementation, and lets us benefit from all the computational advantages mentioned in \cref{chap:einsum}.
For the following theorems, we assume that the computations are all over the same semiring $R = (M, \oplus, \odot)$.

\section{Simple Nested Expressions}

In the following, we will explore how to compress such expressions:
$$\underbrace{(ij, j \rightarrow i, \overbrace{(ik, kj \rightarrow ij, A, B)}^{\text{inner expression}}, v)}_{\text{outer expression}}$$
for $A \in \R^{m \times r}, B \in \R^{r \times n}, v \in \R^n$.

\begin{theorem}
    \label{thm:nested_einsum:simple}

    For $i \in [m + n]$, let $T^{(i)}$ be an $n_i$-th order tensor with index string $\bm{s_i} \in S^{n_i}$.
    Let $\bm{s_u}, \bm{s_v}$ be index strings.
    Let
    $$U = (\bm{s_{m + 1}},\dots,\bm{s_{m + n}} \rightarrow \bm{s_u}, T^{(m + 1)},\dots,T^{(m + n)})$$
    and
    $$V = (\bm{s_1},\dots,\bm{s_m}, \bm{s_u} \rightarrow \bm{s_v}, T^{(1)},\dots,T^{(m)}, U)$$
    where the bound symbols of the second Einsum expression share no symbols with the first Einsum expression,
    then
    $$V = (\bm{s_1}, \dots, \bm{s_{m + n}} \rightarrow \bm{s_v}, T^{(1)}, \dots, T^{(m + n)})$$
    is the compressed Einsum expression for $V$ that includes the computation of $U$.
\end{theorem}

\bigskip
\begin{proof}
    \small
    Let $F, F', B, B'$ be the free and bound symbols of the outer and inner Einsum expression respectively.
    W.l.o.g. they are all non-empty.
    From them we can derive the multi-index spaces $\mathcal{F}, \mathcal{F}', \mathcal{B}, \mathcal{B}'$ as in the definition.
    Then
    \begin{align*}
         & V                                                         &  & = (\bm{s_1},\dots,\bm{s_m}, \bm{s_u} \rightarrow \bm{s_v}, T^{(1)},\dots,T^{(m)}, U)                                                                                                                                                                             \\
         & \iff \forall \bm{f} \in \mathcal{F}: V_{\bm{f}: \bm{s_v}} &  & = \bigoplus\limits_{\bm{b} \in \mathcal{B}} \bigodot\limits_{i = 1}^{m} T^{(i)}_{(\bm{f}, \bm{b}):\bm{s_i}} \odot U_{(\bm{f}, \bm{b}):\bm{s_u}}                                                                                                                  \\
         &                                                           &  & = \bigoplus\limits_{\bm{b} \in \mathcal{B}} \bigodot\limits_{i = 1}^{m} T^{(i)}_{(\bm{f}, \bm{b}):\bm{s_i}} \odot \left[\bigoplus\limits_{\bm{b'} \in \mathcal{B}'} \bigodot\limits_{i' = m + 1}^{m + n} T^{(i')}_{(\bm{f}, \bm{b}, \bm{b'}):\bm{s_{i'}}}\right] \\
         &                                                           &  & = \bigoplus\limits_{\bm{b} \in \mathcal{B}} \bigoplus\limits_{\bm{b'} \in \mathcal{B}'} \bigodot\limits_{i = 1}^{m} T^{(i)}_{(\bm{f}, \bm{b}):\bm{s_i}} \odot \bigodot\limits_{i = m + 1}^{m + n} T^{(i)}_{(\bm{f}, \bm{b}, \bm{b'}):\bm{s_{i}}}                 \\
         &                                                           &  & = \bigoplus\limits_{\bm{b} \in \mathcal{B} \times \mathcal{B}'} \bigodot\limits_{i = 1}^{m + n} T^{(i)}_{(\bm{f}, \bm{b}):\bm{s_i}}                                                                                                                              \\
         & \iff V                                                    &  & = (\bm{s_1}, \dots, \bm{s_{m + n}} \rightarrow \bm{s_v}, T^{(1)}, \dots, T^{(m + n)})
    \end{align*}
    where the third equality follows from the definition of $U$:
    $$\forall \bm{f'} \in \mathcal{F}': U_{\bm{f'}: \bm{s_u}} = \bigoplus\limits_{\bm{b'} \in \mathcal{B}'} \bigodot\limits_{i' = m + 1}^{m + n} T^{(i')}_{(\bm{f'}, \bm{b'}):\bm{s_{i'}}}$$
    and from the fact, that the symbols in $\bm{s_u}$ are used in the outer expression as an input string, and in the inner expression as the output string, and therefore $F' \subseteq B \cup F$.
    Additionally, because of the stated requirement $(B \cup F) \cap B' = \emptyset$, the symbols representing $\bm{b'}$ do not clash with the symbols representing $(\bm{f}, \bm{b})$, and therefore $(\bm{f}, \bm{b}, \bm{b'}):\bm{s_{i'}}$ is well-defined and projects on the same indices as $(\bm{f'}, \bm{b'}):\bm{s_{i'}}$.
    The fourth equality follows from the distributivity of a semiring.
\end{proof}
\bigskip

This means that we can compress all nested Einsum expressions, where the output string of the inner expression, which is used to compute $U$, is exactly the same as the respective input string in the outer expression, where $U$ is used as an input tensor.
This is already helpful for some naturally occuring expressions in linear algebra, e.g.
\begin{align*}
    \abs{A \cdot v}_2^2 & = (i,i\rightarrow,(ij, j \rightarrow i, A, v),(ij, j \rightarrow i, A, v)) \\
                        & = (ij,j,ij,j\rightarrow,A,v,A,v)
\end{align*}
for $A \in \R^{m \times n}, v \in \R^{n}$, or
\begin{align*}
    A \cdot B \cdot v & = (ij, j \rightarrow i, (ik, kj \rightarrow ij, A, B), v) \\
                      & = (ik,kj,j \rightarrow i, A, B, v)
\end{align*}
for $A \in \R^{m \times r}, B \in \R^{r \times n}, v \in \R^n$.
However, sometimes we need to access a different multi-index set than the one we computed, e.g.
$$\text{trace}(A \cdot B) = (ii \rightarrow, (ik, kj \rightarrow ij, A, B))$$
or
$$A \cdot \text{diag}(v) = (ik, kj \rightarrow ij, A, (i \rightarrow ii, v))$$
for $A \in \R^{m \times n}, B \in \R^{n \times m}, v \in \R^{n}$.
For this, we need more general ways of compressing nested Einsum expressions.

\section{General Nested Expressions}

The following is an example of an expression, which we cannot compress with the previous theorem:
$$(a,b,c,d,e,abbcde \rightarrow bc, v^{(1)}, v^{(2)}, v^{(3)}, v^{(4)}, v^{(5)}, (
    i,j,k,l \rightarrow iijkkl, v^{(6)}, v^{(7)}, v^{(8)}, v^{(9)}
    ))$$
for $v^{(i)} \in \R^{d_{vi}}$ with $i \in [9]$, where $d_{vi} \in \N$ are appropriate dimensions.
This is because the output string $\bm{s_u} = iijkkl$ and the input string $\bm{\hat{s}_u} = abbde$ are not the same.
In the following, we will explore how to compress such expressions.
Note that, for the theorem, we use disjoint sets of symbols for the inner and outer expression.
This helps in the proof, and is not a real constraint in practice,
because we can just rename the symbols in different scopes.
For example, we could also write the above expression as
$$(ij, jjj \rightarrow i, A, (ik, kj \rightarrow iij, B, C)),$$
because the scope of each symbol does not reach into nested expressions,
and therefore the $j$ used in the outer expression and the $j$ used in the inner expression are treated as different symbols.

\begin{theorem}
    \label{thm:nested_einsum:general}
    For $i \in [m + n]$, let $T^{(i)}$ be an $n_i$-th order tensor with index string $\bm{s_i} \in S^{n_i}$.
    Let $\bm{s_u}$ be an index string for the $n_u$-th order tensor $U$, which is defined as follows:
    $$U = (\bm{s_{m + 1}},\dots,\bm{s_{m + n}} \rightarrow \bm{s_u}, T^{(m + 1)},\dots,T^{(m + n)})$$
    Also let $\bm{\hat{s}_u}$ be alternative index strings for $U$.

    Let $s_v$ be an index string and
    $$V = (\bm{s_1},\dots,\bm{s_m}, \bm{\hat{s}_u} \rightarrow \bm{s_v}, T^{(1)},\dots,T^{(m)}, U)$$
    where the first and second Einsum expression share no symbols.
    Then these nested Einsum expressions can also be compressed into a single Einsum expression.

    Let us clarify that the index string $\bm{s_u}$ corresponds to the output string of the inner expression,
    and the index string $\bm{\hat{s}_u}$ corresponds to the input string that is used for the input tensor $U$ in the outer expression.
    In our example, these are $\bm{s_u}$ and $\bm{\hat{s}_u}$:
    $$(a,b,c,d,e,\overbrace{abbcde}^{\bm{\hat{s}_u}} \rightarrow bc, v^{(1)}, v^{(2)}, v^{(3)}, v^{(4)}, v^{(5)}, (
        i,j,k,l \rightarrow \overbrace{iijkkl}^{\bm{s_u}}, v^{(6)}, v^{(7)}, v^{(8)}, v^{(9)}
        )).$$

    In contrast to \cref{thm:nested_einsum:simple}, we cannot just replace the input index string $\bm{\hat{s}_u}$ by all the input index strings in the inner Einsum expression $\bm{s_{m + 1}},\dots,\bm{s_{m + n}}$.
    Instead, we first need to apply a symbol map $\nu: S \rightarrow S$ to each of the index strings.
    This symbol map holds information about which symbols are effectively the used for the same index.

    For the definition of the map $\nu$, we first construct an undirected graph $G = (V, E)$ that we call \textit{symbol graph}.
    % $V = \sigma(\bm{s_v}) \cup \sigma(\bm{s_u}) \cup \sigma(\bm{\hat{s}_u}) \cup \bigcup_{i \in [m + n]} \sigma(\bm{s_i})$
    In the symbol graph, the nodes consist of all symbols from $\bm{s_u}$ and $\bm{\hat{s}_u}$.
    The edges are $E = \smallset{\smallset{s_{uj}, \hat{s}_{uj}} \mid j \in [n_u]}$,
    which connects all symbols from $\bm{s_u}$ and $\bm{\hat{s}_u}$ that share an index.
    The symbol graph for our example is illustrated in \cref{fig:nested_expressions:example_symbol_graph}.

    \begin{figure}[h]
        \centering
        \begin{tikzpicture}[node distance = 2cm, semithick]

            \node[state] (a)					{$a$};
            \node[state] (b) [right of=a]		{$b$};
            \node[state] (c) [right of=b] 		{$c$};
            \node[state] (d) [right of=c] 		{$d$};
            \node[state] (e) [right of=d] 		{$e$};
            \node (middle) at ($(a)!0.5!(b)$) {};
            \node[state] (i) [below of=middle] 		{$i$};
            \node[state] (j) [right of=i] 		{$j$};
            \node[state] (k) [right of=j] 		{$k$};
            \node[state] (l) [right of=k] 		{$l$};

            \node (x_center) at ($(a)!0.5!(j)$) {};
            \node (y_center) at ($(c)!0.5!(d|-k)$) {};
            \node (z_center) at ($(e)!0.5!(l)$) {};
            \node (x) [below of=x_center] {$x$};
            \node (y) [below of=y_center] {$y$};
            \node (z) [below of=z_center] {$z$};

            \path (a) edge (i);
            \path (i) edge (b);
            \path (b) edge (j);
            \path (c) edge (k);
            \path (k) edge (d);
            \path (e) edge (l);

            \begin{pgfonlayer}{background}
                \draw[gray!15, fill=gray!15, line width=12mm, line cap=round, line join=round] (a.center) -- (b.center) -- (j.center) -- (i.center) -- cycle;
                \draw[gray!15, fill=gray!15, line width=12mm, line cap=round, line join=round] (c.center) -- (d.center) -- (k.center) -- cycle;
                \draw[gray!15, fill=gray!15, line width=12mm, line cap=round, line join=round] (e.center) -- (l.center) -- cycle;
            \end{pgfonlayer}

        \end{tikzpicture}
        \caption{Symbol graph for the example}
        \label{fig:nested_expressions:example_symbol_graph}
    \end{figure}

    In the symbol graph, if two symbols are connected, then they will both be mapped to the same symbol.
    Therefore, it makes sense assigning a symbol $s_C \in S \setminus V$ to each of the graphs components $C$.
    Then we can define $\nu$ as follows:
    $$\nu(s) := \begin{cases}
            s_C & \text{if $\exists C: \text{$C$ is a component of $G$ and $s \in C$}$}, \\
            s   & \text{else}.
        \end{cases}$$
    In our example, the components are $\smallset{a,b,i,j}$, $\smallset{c,d,k}$, and $\smallset{e,l}$.
    Therefore we could use
    $$\nu(s) := \begin{cases}
            x & \text{if $s \in \smallset{a,b,i,j}$}, \\
            y & \text{if $s \in \smallset{c,d,k}$},   \\
            z & \text{if $s \in \smallset{e,l}$},     \\
            s & \text{else}.
        \end{cases}$$

    The symbol map $\nu$ can be extended, such that it maps entire index strings instead of just symbols, by setting $\nu(\bm{s_i}) \in S^{n_i}, \nu(\bm{s_i})_j := \nu(s_{ij})$.
    Then we can write the substituted index strings by setting $\bm{\hat{s}_i} := \nu(\bm{s_i})$ for $i \in [m + n]$ and $\bm{\hat{s}_t} = \nu(\bm{s_t})$.
    With these index strings, the compressed Einsum expression is the following:
    $$V = (\bm{\hat{s}_1}, \dots, \bm{\hat{s}_{m + n}} \rightarrow \bm{\hat{s}_v}, T^{(1)},\dots,T^{(m + n)})$$
    which helps us to compress the example:
    \begin{gather*}
        (a,b,c,d,e,abbde \rightarrow bc, v^{(1)}, v^{(2)}, v^{(3)}, v^{(4)}, v^{(5)}, (
        i,j,k,l \rightarrow iijkkl, v^{(6)}, v^{(7)}, v^{(8)}, v^{(9)}
        ))\\
        =(x,x,y,y,z,x,x,y,z \rightarrow xy, v^{(1)}, v^{(2)}, v^{(3)}, v^{(4)}, v^{(5)}, v^{(6)}, v^{(7)}, v^{(8)}, v^{(9)}).
    \end{gather*}
\end{theorem}

\bigskip
For the proof of this theorem, we first need three lemmata, which essentially boil down to one intuitive thought:
The effective equality of two symbols can be expressed by multiplication with the unity matrix $\1_d$:
$$\left(\1_d\right)_{ij} := \begin{cases}
        \1 & \text{if $i = j$}, \\
        \0 & \text{else}
    \end{cases}$$
for $i,j \in [d]$, where $\0$ and $\1$ indicate the neutral element of addition and multiplication in the given semiring respectively.

\begin{lemma}
    \label{lemma:nested_einsum:1}
    For $i \in [n]$, let $T^{(i)}$ be an $n_i$-th order tensor with index string $\bm{s_i} \in S^{n_i}$.
    Let $\bm{s_t}$ be the index string for $T$ with
    $$T = (\bm{s_1}, \dots, \bm{s_n} \rightarrow \bm{s_t}, T^{(1)}, \dots, T^{(n)}).$$
    Let $F$ and $B$ be the free and bound symbols of this expression.
    Let $k \in [n]$ and $j \in [n_k]$, then we can replace the $j$-th symbol of the $k$-th index string with a new symbol $s_{\text{new}} \in S \setminus (F \cup B)$ by adding the unity matrix $\1_{d_{kj}}$ as an input tensor in the following way:

    Let $\bm{s'_k}$ be a new index string such that
    $$s'_{ki} := \begin{cases}
            s_{\text{new}} & \text{if $i = j$}, \\
            s_{ki}         & \text{else}
        \end{cases}$$
    for $i \in [n_k]$.
    Let $\bm{s_\1} = (s_{kj}, s_{\text{new}})$.
    Then
    $$T = (\bm{s_1}, \dots, \bm{s'_k}, \dots, \bm{s_n}, \bm{s_\1} \rightarrow \bm{s_t}, T^{(1)}, \dots, T^{(n)}, \1_{d_{kj}}).$$
\end{lemma}

\begin{proof}
    \small
    Let $\mathcal{F}$ and $\mathcal{B}$ be the induces multi-index spaces for the free and bound symbols of the Einsum expression.
    Then
    \begin{align*}
        T                                                        & = (\bm{s_1}, \dots, \bm{s_n} \rightarrow \bm{s_t}, T^{(1)}, \dots, T^{(n)})                                                                                                                                                                        \\
        \iff \forall \bm{f} \in \mathcal{F}: T_{\bm{f}:\bm{s_t}} & = \bigoplus\limits_{\bm{b} \in \mathcal{B}} \bigodot\limits_{i \in [n]} T^{(i)}_{(\bm{f}, \bm{b}):\bm{s_i}}                                                                                                                                        \\
                                                                 & = \bigoplus\limits_{\bm{b} \in \mathcal{B} \times [d_{kj}]} \bigodot\limits_{1 \leq i < k} T^{(i)}_{(\bm{f}, \bm{b}):\bm{s_i}} \odot T^{(k)}_{(\bm{f}, \bm{b}):\bm{s'_k}} \odot \bigodot\limits_{k < i \leq n} T^{(i)}_{(\bm{f}, \bm{b}):\bm{s_i}} \\
                                                                 & \phantom{{}=\bigoplus\limits_{\bm{b} \in \mathcal{B} \times [d_{kj}]}} \odot \begin{cases}
            \1 & \text{if $(\bm{f}, \bm{b}): s_{kj} = (\bm{f}, \bm{b}): s_{\text{new}}$}, \\
            \0 & \text{else}
        \end{cases}                                                                                                                                             \\
                                                                 & = \bigoplus\limits_{\bm{b} \in \mathcal{B} \times [d_{kj}]} \bigodot\limits_{1 \leq i < k} T^{(i)}_{(\bm{f}, \bm{b}):\bm{s_i}} \odot T^{(k)}_{(\bm{f}, \bm{b}):\bm{s'_k}} \odot \bigodot\limits_{k < i \leq n} T^{(i)}_{(\bm{f}, \bm{b}):\bm{s_i}} \\
                                                                 & \phantom{{}=\bigoplus\limits_{\bm{b} \in \mathcal{B} \times [d_{kj}]}} \odot \left(\1_{d_{kj}}\right)_{(\bm{f}, \bm{b}):\bm{s_\1}}                                                                                                                 \\
        \iff T                                                   & = (\bm{s_1}, \dots, \bm{s'_k}, \dots, \bm{s_n}, \bm{s_\1} \rightarrow \bm{s_t}, T^{(1)}, \dots, T^{(n)}, \1_{d_{kj}})
    \end{align*}
    where the third equality holds because in the summation over $\mathcal{B} \times [d_{kj}]$, exactly those summands get selected by the condition, which are also valid summands in the previous summation over $\mathcal{B}$.
    All other summands are disregarded because they are multiplied by $\0$, which is the additive neutral element in the semiring and \textit{annihilates} every element, which means $a \odot \0 = \0$ for every $a \in M$.
\end{proof}
\bigskip

This lemma intuitively means that we can replace any symbol in an index string of an input tensor of our choice with a new symbol by introducing the unity matrix with an appropriate index string as a factor.
Now the same holds for the index string of the output tensor $\bm{s_t}$, which will be the content of the next lemma.

\begin{lemma}
    \label{lemma:nested_einsum:2}
    For $i \in [n]$, let $T^{(i)}$ be an $n_i$-th order tensor with index string $\bm{s_i} \in S^{n_i}$.
    Let $\bm{s_t}$ be the index string for $T$ with
    $$T = (\bm{s_1}, \dots, \bm{s_n} \rightarrow \bm{s_t}, T^{(1)}, \dots, T^{(n)}).$$
    Let $F$ and $B$ be the free and bound symbols of this expression.
    Let $n_t := \abs{\bm{s_t}}$, $j \in [n_t]$, and $d_{tj} := d_{s_{tj}}$, then we can replace the $j$-th symbol of the output string with a new symbol $s_{\text{new}} \in S \setminus (F \cup B)$ by adding the unity matrix $\1_{d_{tj}}$ as an input tensor in the following way:

    Let $\bm{s'_t}$ be a new index string such that
    $$s'_{ti} := \begin{cases}
            s_{\text{new}} & \text{if $i = j$}, \\
            s_{ti}         & \text{else}
        \end{cases}$$
    for $i \in [n_t]$.
    Let $\bm{s_\1} = (s_{tj}, s_{\text{new}})$.
    Then
    $$T = (\bm{s_1}, \dots, \bm{s_n}, \bm{s_\1} \rightarrow \bm{s'_t}, T^{(1)}, \dots, T^{(n)}, \1_{d_{kj}}).$$
\end{lemma}

\begin{proof}
    \small
    Let $\mathcal{F}$ and $\mathcal{B}$ be the induces multi-index spaces for the free and bound symbols of the Einsum expression.
    % TODO: Fallunterscheidung s_{tj} wird bound symbol / wird kein bound symbol
    If $s_{tj}$ occurs in $\bm{s_t}$ even after replacing it with $s_{\text{new}}$, then
    \begin{align*}
        T                                                                         & = (\bm{s_1}, \dots, \bm{s_n} \rightarrow \bm{s_t}, T^{(1)}, \dots, T^{(n)})                                                                                             \\
        \iff \forall \bm{f} \in \mathcal{F}: T_{\bm{f}:\bm{s_t}}                  & = \bigoplus\limits_{\bm{b} \in \mathcal{B}} \bigodot\limits_{i \in [n]} T^{(i)}_{(\bm{f}, \bm{b}):\bm{s_i}}                                                             \\
        \iff \forall \bm{f} \in \mathcal{F} \times [d_{tj}]: T_{\bm{f}:\bm{s'_t}} & = \bigoplus\limits_{\bm{b} \in \mathcal{B}} \bigodot\limits_{i \in [n]} T^{(i)}_{(\bm{f}, \bm{b}):\bm{s_i}} \odot \begin{cases}
            \1 & \text{if $\bm{f}: s_{tj} = \bm{f}: s_{\text{new}}$}, \\
            \0 & \text{else}
        \end{cases}                             \\
                                                                                  & = \bigoplus\limits_{\bm{b} \in \mathcal{B}} \bigodot\limits_{i \in [n]} T^{(i)}_{(\bm{f}, \bm{b}):\bm{s_i}} \odot \left(\1_{d_{tj}}\right)_{(\bm{f}, \bm{b}):\bm{s_\1}} \\
        \iff T                                                                    & = (\bm{s_1}, \dots, \bm{s_n}, \bm{s_\1} \rightarrow \bm{s'_t}, T^{(1)}, \dots, T^{(n)}, \1_{d_{tj}})
    \end{align*}
    where the third equality holds because exactly those indices get selected by the condition, where $T$ was originally defined.
    If $s_{tj}$ no longer occurs in $\bm{s_t}$ after replacing it with $s_{\text{new}}$, then $s_{tj}$ turns into a bound symbol.
    Therefore we have to define $\mathcal{F}' = \prod_{s \in \bm{s'_t}} [d_s]$.
    Then
    \begin{align*}
        T                                                          & = (\bm{s_1}, \dots, \bm{s_n} \rightarrow \bm{s_t}, T^{(1)}, \dots, T^{(n)})                                                                                                             \\
        \iff \forall \bm{f} \in \mathcal{F}: T_{\bm{f}:\bm{s_t}}   & = \bigoplus\limits_{\bm{b} \in \mathcal{B}} \bigodot\limits_{i \in [n]} T^{(i)}_{(\bm{f}, \bm{b}):\bm{s_i}}                                                                             \\
        \iff \forall \bm{f} \in \mathcal{F}': T_{\bm{f}:\bm{s'_t}} & = \bigoplus\limits_{\bm{b} \in \mathcal{B} \times [d_{tj}]} \bigodot\limits_{i \in [n]} T^{(i)}_{(\bm{f}, \bm{b}):\bm{s_i}} \odot \begin{cases}
            \1 & \text{if $(\bm{f}, \bm{b}): s_{tj} = (\bm{f}, \bm{b}): s_{\text{new}}$}, \\
            \0 & \text{else}
        \end{cases}                             \\
                                                                   & = \bigoplus\limits_{\bm{b} \in \mathcal{B} \times [d_{tj}]} \bigodot\limits_{i \in [n]} T^{(i)}_{(\bm{f}, \bm{b}):\bm{s_i}} \odot \left(\1_{d_{tj}}\right)_{(\bm{f}, \bm{b}):\bm{s_\1}} \\
        \iff T                                                     & = (\bm{s_1}, \dots, \bm{s_n}, \bm{s_\1} \rightarrow \bm{s'_t}, T^{(1)}, \dots, T^{(n)}, \1_{d_{tj}})
    \end{align*}
    where the third equality holds because exactly those summands get selected by the condition, where $(\bm{f}, \bm{b}):\bm{s_t}$ could also get used as an index for $T$.
    All others are annihilated.
\end{proof}
\bigskip

Now with these two lemmata, we can replace any symbol in any index string, regardless whether it is an input string or the output string, by introducing the unity matrix with an appropriate index string as a factor.
In the following lemma, we will show that any unity matrix factors can be removed again, by renaming certain symbols in all other index strings in the Einsum expression.

\begin{lemma}
    \label{lemma:nested_einsum:3}
    For $i \in [n]$, let $T^{(i)}$ be an $n_i$-th order tensor with index string $\bm{s_i} \in S^{n_i}$,
    where $T^{(n)} = \1_m$ for some $m \in \N$.
    Let $\bm{s_t}$ be the index string for $T$ with
    $$T = (\bm{s_1}, \dots, \bm{s_n} \rightarrow \bm{s_t}, T^{(1)}, \dots, T^{(n)}).$$

    Let $F$ and $B$ be the free and bound symbols of this expression.
    Then we can introduce a symbol map $\mu: S \rightarrow S$, which maps both symbols in $\bm{s_n} = s_{n1}s_{n2}$ to the same symbol $s_{\text{new}} \in S \setminus (F \cup B)$:
    $$\mu(s) := \begin{cases}
            s_{\text{new}} & \text{if $s \in \smallset{s_{n1}, s_{n2}}$}, \\
            s              & \text{else}.
        \end{cases}$$
    The symbol map $\mu$ can be extended, such that it maps entire index strings instead of just symbols, by setting $\mu(\bm{s_i}) \in S^{n_i}, \mu(\bm{s_i})_j := \mu(s_{ij})$.
    Then we can write the substituted index strings by setting $\bm{s'_i} := \mu(\bm{s_i})$ for $i \in [n]$ and $\bm{s'_t} = \mu(\bm{s_t})$.
    With these index strings, the following holds:
    $$T = (\bm{s'_1}, \dots, \bm{s'_{n - 1}} \rightarrow \bm{s'_t}, T^{(1)}, \dots, T^{(n - 1)}).$$
\end{lemma}

\begin{proof}
    \small
    For this proof, we provide the following example of an Einsum expression on which we demonstrate the given arguments for better understanding:
    $$(ij, kl, mn, ij, kl, mn \rightarrow imn, A, B, C, \1_a, \1_b, \1_c)$$
    for $A \in \R^{a \times a}, B \in \R^{b \times b}, C \in \R^{c \times c}$ and some $a,b,c \in \N$.

    We need to consider three cases for the symbols used in the index string $\bm{s_n} = (s_{n1}, s_{n2})$:
    \begin{itemize}
        \item $s_{n1}$ and $s_{n2}$ are both free symbols,
        \item $s_{n1}$ and $s_{n2}$ are both bound symbols,
        \item one symbol of $s_{n1}$ and $s_{n2}$ is a free symbol, the other is a bound symbol.
    \end{itemize}
    Every one of these cases leads to the same result, but in a slightly different way.

    First let us consider the case where both symbols are free.
    In this case, both symbols can be replaced by a single symbol,
    because $T$ is $\0$ for all entries with a multi-index,
    where the indices projected by the symbols are not equal.

    In our example, this is equivalent to the following:
    \begin{align*}
        \forall i,m,n: T_{imn}    & = \bigoplus\limits_{j,k,l} A_{ij} B_{kl} C_{mn} \left(\1_a\right)_{ij} \left(\1_b\right)_{kl} \left(\1_c\right)_{mn} \\
                                  & = \begin{cases}
            \bigoplus\limits_{j,k,l} A_{ij} B_{kl} C_{mn} \left(\1_a\right)_{ij} \left(\1_b\right)_{kl} & \text{if $m = n$}, \\
            \0                                                                                          & \text{else}
        \end{cases}                                                                                          \\
        \iff \forall i,z: T_{izz} & = \bigoplus\limits_{j,k,l} A_{ij} B_{kl} C_{zz} \left(\1_a\right)_{ij} \left(\1_b\right)_{kl}.
    \end{align*}

    Next let us consider the case where both symbols are bound.
    In this case, those summands are multiplied with $\0$,
    which have a multi-index where the projected indices are not equal.
    Therefore, those summands are annihilated and left out from the summation.
    This means that both symbols can be replaced by a single symbol.

    In our example, this is equivalent to the following:
    \begin{align*}
        \forall i,z: T_{izz} & = \bigoplus\limits_{j,k,l} A_{ij} B_{kl} C_{zz} \left(\1_a\right)_{ij} \left(\1_b\right)_{kl}          \\
                             & = \bigoplus\limits_{j,k,l} A_{ij} B_{kl} C_{zz} \left(\1_a\right)_{ij} \odot \begin{cases}
            \1 & \text{if $k = l$}, \\
            \0 & \text{else}
        \end{cases} \\
                             & = \bigoplus\limits_{j,y} A_{ij} B_{yy} C_{zz} \left(\1_a\right)_{ij}.
    \end{align*}

    Next let us consider the case where one symbol is free and one symbol is bound.
    W.l.o.g. we consider the case where $s_{n1}$ is free and $s_{n2}$ is bound.
    In this case, those summands are multiplied with $\0$,
    which have a multi-index where the index projected by the bound symbol $s_{n2}$ is not the same as the index projected by the free symbol $s_{n1}$.
    Therefore those summands are annihilated and left out from the summation, and the symbol $s_{n2}$ can be replaced by the symbol $s_{n1}$.
    Additionally, we can rename the $s_{n1}$ to some new symbol.

    In our example, this is equivalent to the following:
    \begin{align*}
        \forall i,z: T_{izz}      & = \bigoplus\limits_{j,y} A_{ij} B_{yy} C_{zz} \left(\1_a\right)_{ij}          \\
                                  & = \bigoplus\limits_{j,y} A_{ij} B_{yy} C_{zz} \odot \begin{cases}
            \1 & \text{if $i = j$}, \\
            \0 & \text{else}
        \end{cases} \\
                                  & = \bigoplus\limits_{y} A_{ii} B_{yy} C_{zz}                                   \\
        \iff \forall x,z: T_{xzz} & = \bigoplus\limits_{y} A_{xx} B_{yy} C_{zz}.
    \end{align*}

    Therefore, in all three cases, the symbols, that are used in an index string for a unity matrix, can simply be replaced by a single symbol.
\end{proof}

From these three lemmata, \cref{thm:nested_einsum:general} follows with the following procedure:
\begin{enumerate}[label={Step \arabic*:}, align=left]
    \item Apply \cref{lemma:nested_einsum:1} to all of the symbols in the input string $\bm{\hat{s}_u}$ and therefore replace it with a new index string $\bm{s'_u} \in S^{n_u}$ where $\bm{s'_u}$ contains no duplicate symbols.
    \item Apply \cref{lemma:nested_einsum:2} to all of the symbols in the output string $\bm{s_u}$ and therefore replace it with the same new index string $\bm{s'_u}$.
    \item Apply \cref{thm:nested_einsum:simple} to the nested expression and therefore compress the nested expression into a single expression with lots of unity matrices.
          This is possible because the input string and output string for $U$ are both $\bm{s'_u}$ now.
    \item Apply \cref{lemma:nested_einsum:3} to remove unity matrices from the compressed expression until there are no more of those unity matrices left, which were introduced in Step 1 and Step 2.
\end{enumerate}

\begin{proof}
    \small
    For the proof, we again demonstrate the arguments on the example used in \cref{thm:nested_einsum:general} for better understanding:
    $$(a,b,c,d,e,abbcde \rightarrow bc,v^{(1)}, v^{(2)}, v^{(3)}, v^{(4)}, v^{(5)}, (
        i,j,k,l \rightarrow iijkkl, v^{(6)}, v^{(7)}, v^{(8)}, v^{(9)}
        ))$$
    Applying Step 1 and Step 2 to this example results in the following Einsum expression if $\bm{s'_u} = s_1 s_2 s_3 s_4 s_5 s_6$:
    \begin{gather*}
        (a,b,c,d,e,s_1 s_2 s_3 s_4 s_5 s_6, a s_1, b s_2, b s_3, c s_4, d s_5, e s_6 \rightarrow bc,\\
        v^{(1)}, v^{(2)}, v^{(3)}, v^{(4)}, v^{(5)}, \1, \1, \1, \1, \1, \1,\\
        (i,j,k,l, i s_1, i s_2, j s_3, k s_4, k s_5, l s_6 \rightarrow s_1 s_2 s_3 s_4 s_5 s_6,\\
        v^{(6)}, v^{(7)}, v^{(8)}, v^{(9)}, \1, \1, \1, \1, \1, \1)),
    \end{gather*}
    where we used $\1$ to indicate unity matrices of different sizes, because the sizes can be derived from the context and are not important for better understanding.
    Applying Step 3 results in the following compressed expression:
    \begin{gather*}
        (a,b,c,d,e,i,j,k,l, i s_1, i s_2, j s_3, k s_4, k s_5, l s_6, a s_1, b s_2, b s_3, c s_4, d s_5, e s_6 \rightarrow bc,\\
        v^{(1)}, v^{(2)}, v^{(3)}, v^{(4)}, v^{(5)}, v^{(6)}, v^{(7)}, v^{(8)}, v^{(9)}, \1, \1, \1, \1, \1, \1, \1, \1, \1, \1, \1, \1)
    \end{gather*}

    The only fact that remains to be understood is why the removal of the unity matrices in Step 4 leads to the transformation described in \cref{thm:nested_einsum:general}.
    For this, we construct another undirected graph $G' = (V', E')$ that we call \textit{extended symbol graph}.
    In the extended symbol graph, the nodes consist of all symbols from $\bm{s_u}$, $\bm{s'_u}$, and $\bm{\hat{s}_u}$.
    An edge $\smallset{u,v}$ exists precisely when the compressed expression contains a unity matrix with index string $uv$, that was introduced in Step 1 or Step 2.
    The extended symbol graph for our example is illustrated in \cref{fig:nested_expressions:example_extended_symbol_graph}.

    \begin{figure}[h]
        \centering
        \begin{tikzpicture}[semithick, scale=0.6]

            \node[state] (a) at (0, 0)		{$a$};
            \node[state] (b) at (3, 0)		{$b$};
            \node[state] (c) at (6, 0) 		{$c$};
            \node[state] (d) at (9, 0) 		{$d$};
            \node[state] (e) at (12, 0) 	{$e$};

            \node[state] (s1) at (-0.25, -2)        {$s_1$};
            \node[state] (s2) at (2.25, -2)         {$s_2$};
            \node[state] (s3) at (4.75, -2)         {$s_3$};
            \node[state] (s4) at (7.25, -2)         {$s_4$};
            \node[state] (s5) at (9.75, -2)         {$s_5$};
            \node[state] (s6) at (12.25, -2)        {$s_6$};

            \node[state] (i) at (1.5, -4) 		{$i$};
            \node[state] (j) at (4.5, -4) 		{$j$};
            \node[state] (k) at (7.5, -4) 		{$k$};
            \node[state] (l) at (10.5, -4) 		{$l$};

            \path (a) edge (s1);
            \path (b) edge (s2);
            \path (b) edge (s3);
            \path (c) edge (s4);
            \path (d) edge (s5);
            \path (e) edge (s6);

            \path (i) edge (s1);
            \path (i) edge (s2);
            \path (j) edge (s3);
            \path (k) edge (s4);
            \path (k) edge (s5);
            \path (l) edge (s6);

            \begin{pgfonlayer}{background}
                \draw[gray!15, fill=gray!15, line width=12mm, line cap=round, line join=round] (a.center) -- (s1.center) -- (i.center) -- (j.center) -- (s3.center) -- (b.center) -- cycle;
            \end{pgfonlayer}

        \end{tikzpicture}
        \caption{Extended symbol graph for the example with the first component highlighted}
        \label{fig:nested_expressions:example_extended_symbol_graph}
    \end{figure}

    Now, every newly introduced unity matrix is represented by an edge in the extended symbol graph.
    If a unity matrix is removed with \cref{lemma:nested_einsum:3}, the symbols connected by the representing edge collapse into one symbol,
    and the rest of the extended symbol graph stays the same.
    An example of this collapse is illustrated in \cref{fig:nested_expressions:collapsed_extended_symbol_graph}.

    \begin{figure}[h]
        \centering
        \begin{tikzpicture}[semithick, scale=0.6]

            % graph where the unity matrix was not removed

            \node[state] (a) at (0, 0)		{$a$};
            \node[state] (b) at (3, 0)		{$b$};

            \node[state] (s1) at (-0.25, -2)        {$s_1$};
            \node[state] (s2) at (2.25, -2)         {$s_2$};
            \node[state] (s3) at (4.75, -2)         {$s_3$};

            \node[state] (i) at (1.5, -4) 		{$i$};
            \node[state] (j) at (4.5, -4) 		{$j$};

            \path (a) edge (s1);
            \path[dashed] (b) edge (s2);
            \path (b) edge (s3);

            \path (i) edge (s1);
            \path (i) edge (s2);
            \path (j) edge (s3);

            % graph where the unity matrix was removed

            \node[state] (removed_a) at (9, 0)		{$a$};
            \node (removed_b) at (12, 0)		{};

            \node[state] (removed_s1) at (8.75, -2)    {$s_1$};
            \node[state, dashed] (removed_x) at (11.25, -2)         {$x$};
            \node[state] (removed_s3) at (13.75, -2)         {$s_3$};

            \node[state] (removed_i) at (10.5, -4) 		{$i$};
            \node[state] (removed_j) at (13.5, -4) 		{$j$};

            \path (removed_a) edge (removed_s1);
            \path (removed_x) edge (removed_s3);

            \path (removed_i) edge (removed_s1);
            \path (removed_i) edge (removed_x);
            \path (removed_j) edge (removed_s3);

            \begin{pgfonlayer}{background}
                \draw[gray!15, fill=gray!15, line width=12mm, line cap=round, line join=round] (a.center) -- (s1.center) -- (i.center) -- (j.center) -- (s3.center) -- (b.center) -- cycle;
                \draw[gray!15, fill=gray!15, line width=12mm, line cap=round, line join=round] (removed_a.center) -- (removed_s1.center) -- (removed_i.center) -- (removed_j.center) -- (removed_s3.center) -- (removed_b.center) -- cycle;
            \end{pgfonlayer}

            % connect them
            \draw[->, thick] (6, -2) -- (7.5, -2);

        \end{tikzpicture}
        \caption{First component of the extended symbol graph after removing the unity matrix represented by the edge $\smallset{b, s_2}$}
        \label{fig:nested_expressions:collapsed_extended_symbol_graph}
    \end{figure}

    Therefore, after repeatedly applying \cref{lemma:nested_einsum:3},
    all nodes that are part of the same component in the extended symbol graph will collapse into one symbol.
    Now the only thing left to show is that two nodes are connected in $G$ exactly if they are also connected in $G'$.
    This can be seen by understanding the edges of $G$ and $G'$.
    In $G$, two symbols $s_{ui}$ and $\hat{s}_{uj}$ share an edge if they share a position $i = j$.
    In $G'$, two symbols $s_{ui}$ and $\hat{s}_{uj}$ share a neighbour $s'_{uk}$ if both were replaced by $s'_{uk}$ in Step one and Step two,
    which happens exactly if they share a position $i = j = k$.
    Therefore every edge in $G$ is represented by a shared neighbour in $G'$.
    Now, because every $s'_{uk}$ has exactly two neighbours, and because there are no direct edges between any symbols of $\bm{s_u}$ and $\bm{\hat{s}_u}$, there are no more edges in $G'$ other than the ones that contribute to a shared neighbour $s'_{uk}$.
    Then, because sharing an edge in $G$ is the same as sharing a neighbour in $G'$, two symbols are in the same component in $G$ precisely when they are also in the same component in $G'$.

    Therefore collapsing every edge in the extended symbol graph leads to the symbol map defined in \cref{thm:nested_einsum:general}.
\end{proof}

From this general theorem we can derive two more specific theorems,
which make the process of compressing simpler for a subset of nested expressions.

\section{Duplications}

The following two theorems revolve around duplicated symbols in index strings and the way these duplications are \textit{broken}.
We speak of a broken duplication in $\bm{s_u}$, if the symbols $s_{ui}$ and $s_{uj}$ at two positions $i,j \in [n_u]$ are the same, meaning $s_{ui} = s_{uj}$,
but symbols at the same positions in $\bm{\hat{s}_u}$, $\hat{s}_{ui}$ and $\hat{s}_{uj}$ are different, meaning $\hat{s}_{ui} \neq \hat{s}_{uj}$.
In the same way, duplications in $\bm{\hat{s}_u}$ can be broken by $\bm{s_u}$.

Because $\bm{s_u}$ and $\bm{\hat{s}_u}$ are easily confused, we can also use a different terminology that does not include these similar variable names.
In the following, we only refer to the changes the outer expression applies to the duplications in the inner expression.
\begin{itemize}
    \item If $\bm{s_u}$ breaks duplications in $\bm{\hat{s}_u}$,
          the outer expression uses duplications in the input string for the inner expression, that were not used in the output string of the inner expression.
          Therefore, we say that the outer expression \textit{introduces} duplications to the inner expression.
    \item If $\bm{\hat{s}_u}$ breaks duplications in $\bm{s_u}$,
          the outer expression uses different symbols in the input string for the inner expression, where there was originally a duplication in the output string of the inner expression.
          Therefore, we say that the outer expression \textit{removes} duplications from the inner expression.
\end{itemize}

\section{Introducing Duplications}

The following example is another expression, which we cannot compress with the previous theorem:
$$(ij, jjj \rightarrow i, A, (kl, lo \rightarrow kko, B, C))$$
for $A \in \R^{a \times b}, B \in \R^{b \times c}, C \in \R^{c \times b}$.
In the following, we will explore how to compress such expressions.
Note that, for the theorem, we use disjoint sets of symbols for the inner and outer expression.
This helps in the proof, and is not a real constraint in practice,
because we can just rename the symbols in different scopes.
For example, we could also write the above expression as
$$(ij, jjj \rightarrow i, A, (ik, kj \rightarrow iij, B, C)),$$
because the scope of each symbol does not reach into nested expressions,
and therefore the $j$ used in the outer expression and the $j$ used in the inner expression are treated as different symbols.

\begin{theorem}
    \label{thm:nested_einsum:2}

    For $i \in [m + n]$, let $T^{(i)}$ be an $n_i$-th order tensor with index strings $\bm{s_i} \in S^{n_i}$.
    Let $\bm{s_u}$ be an index string for the $n_u$-th order tensor $U$, which is defined as follows:
    $$U := (\bm{s_{m + 1}},\dots,\bm{s_{m + n}} \rightarrow \bm{s_u}, T^{(m + 1)},\dots,T^{(m + n)})$$
    Also let $\bm{\hat{s}_u}$ be alternative index strings for $U$ with $s_{uj} = s_{uj'} \implies \hat{s}_{uj} = \hat{s}_{uj'}$ for all $j, j' \in [n_u]$,
    which means that $\bm{\hat{s}_u}$ can only introduce new symbol duplications, and cannot remove any.
    The index string $\bm{s_u}$ corresponds to the output string of the inner expression,
    and the index string $\bm{\hat{s}_u}$ corresponds to the input string that is used for the input tensor $U$ in the outer expression.

    In our example, $\bm{s_u} = kko$ and $\bm{\hat{s}_u} = jjj$.
    This does not break the symbol duplication of the first and second index,
    and introduces a new duplication on the third index.

    Let $s_v$ be an index string and
    $$V := (\bm{s_1},\dots,\bm{s_m}, \bm{\hat{s}_u} \rightarrow \bm{s_v}, T^{(1)},\dots,T^{(m)}, U)$$
    such that the first and second Einsum expression share no symbols.
    Then these nested Einsum expressions can also be compressed into a single Einsum expression.

    In contrast to \cref{thm:nested_einsum:1}, we cannot just replace the input index string $\bm{\hat{s}_u}$ by all the input index strings in the inner Einsum expression $\bm{s_{m + 1}},\dots,\bm{s_{m + n}}$.
    Instead, we first need to apply a symbol map to the input strings of the inner expression.
    Let $\nu: S \rightarrow S$ such that
    $$\nu(s) := \begin{cases}
            \hat{s}_{uj} & \text{if }\exists j \in [n_u]: s_{uj} = s \\
            s            & \text{else}
        \end{cases}$$
    which maps symbols in $\bm{s_u}$ to the symbol at the same index in $\bm{\hat{s}_u}$ and all other symbols to themselves.

    This symbol map holds information about which symbols will be iterated over at the same time in the outer expression.
    In our example, we have the following symbols on the same positions:
    \begin{itemize}
        \item $s_{u1} = k$ and $\hat{s}_{u1} = j$,
        \item $s_{u2} = k$ and $\hat{s}_{u2} = j$,
        \item $s_{u3} = o$ and $\hat{s}_{u3} = j$.
    \end{itemize}
    Therefore these are the important mappings:
    \begin{align*}
        k & \rightarrow j, \\
        o & \rightarrow j.
    \end{align*}
    This means that $k$ and $o$ will be iterated over at the same time.

    The symbol map $\nu$ can be extended, such that it maps entire index strings instead of just symbols, by setting $\nu(\bm{s_i}) \in S^{n_i}, \nu(\bm{s_i})_j := \nu(s_{ij})$.
    Then we can write the substituted index strings by setting $\bm{\hat{s}_i} := \nu(\bm{s_i})$ for $i \in [m + 1, m + n]$.

    The compressed Einsum expression now becomes the following:
    $$V = (\bm{s_1},\dots,\bm{s_m}, \bm{\hat{s}_{m + 1}}, \dots, \bm{\hat{s}_{m + n}} \rightarrow \bm{s_v}, T^{(1)},\dots,T^{(m + n)})$$
    which helps us to compress the example:
    $$(ij, jjj \rightarrow i, A, (kl, lo \rightarrow kko, B, C)) = (ij, jl, lj \rightarrow i, A, B, C).$$
\end{theorem}

\bigskip
\begin{proof}
    \small
    The fundamental idea behind this theorem is, that by using the index string $\bm{\hat{s}_u}$, we only iterate over a sub-space of the indices that we defined for the computation of $U$.
    The way in which we iterate over this sub-space is determined by the outer expression.
    It could either be the sum over bound indices or the universal quantifier over free indices.
    To formulate this, we need some idea of which multi-indices we iterate over.
    Therefore, let $\mathcal{I}:\bm{s} := \smallset{\bm{i}: \bm{s} \mid \bm{i} \in \mathcal{I}}$ for an index string $\bm{s}$ and a multi-index space $\mathcal{I}$.

    Let $F', B'$ be the free and bound symbols of the inner Einsum expression.
    W.l.o.g. they are both non-empty.
    From them we can derive the multi-index spaces $\mathcal{F}', \mathcal{B}'$ as in the definition.
    Let $\hat{F}' = \sigma(\bm{\hat{s}_u})$ and $\mathcal{\hat{F}}' = \prod_{s \in \hat{F}'} [d_s]$.
    Then $\mathcal{\hat{F}}':\bm{\hat{s}_u} \subseteq \mathcal{F}':\bm{s_u}$.
    This follows from $d_{s_{uj}} = d_{\hat{s}_{uj}}$ for $j \in [n_u]$,
    and the amount of symbols in the projection of $\mathcal{\hat{F}}':\bm{\hat{s}_u}$ being smaller or equal to the amount of symbols in the projection of $\mathcal{F}':\bm{s_u}$.
    The first fact is true per the definition of Einsum.
    The second fact can be rewritten as $\abs{\sigma(\bm{\hat{s}_u})} \leq \abs{\sigma(\bm{s_u})}$ and follows directly from the constraint $s_{uj} = s_{uj'} \implies \hat{s}_{uj} = \hat{s}_{uj'}$ for all $j,j' \in [n_u]$.

    Then
    $$\forall \bm{f'} \in \mathcal{F}': U_{\bm{f'}: \bm{s_u}} = \bigoplus\limits_{\bm{b'} \in \mathcal{B}'}\bigodot\limits_{i = m + 1}^{m + n} T^{(i)}_{(\bm{f'}, \bm{b'}):\bm{s_{i}}}$$
    and therefore
    $$\forall \bm{\hat{f}'} \in \mathcal{\hat{F}}': U_{\bm{\hat{f}}': \bm{\hat{s}_u}} = \bigoplus\limits_{\bm{b'} \in \mathcal{B}'}\bigodot\limits_{i = m + 1}^{m + n} T^{(i)}_{(\bm{\hat{f}}', \bm{b'}):\bm{\hat{s}_{i}}}$$
    because of the previous observation,
    and because the bound symbols of the expression, which are used in $\bm{b'}$, do not occur in $\bm{s_u}$, and are therefore not changed by the symbol map $\nu$.
    Therefore
    $$U = (\bm{\hat{s}_{m + 1}},\dots,\bm{\hat{s}_{m + n}} \rightarrow \bm{\hat{s}_u}, T^{(m + 1)},\dots,T^{(m + n)})$$
    and we can use \cref{thm:nested_einsum:1} for
    $$V = (\bm{s_1}, \dots, \bm{s_m}, \bm{\hat{s}_{m + 1}}, \dots, \bm{\hat{s}_{m + n}} \rightarrow \bm{s_v}, T^{(1)}, \dots, T^{(m + n)})$$
    because the bound symbols of the inner expression have not been mapped to any of the symbols used in the outer expression.
\end{proof}
\bigskip

This theorem suffices to prove a property of the trace in a relatively simple manner, namely that for $A \in \R^{m \times n}, B \in \R^{n \times m}$,
it holds that
$$\text{trace}(A \cdot B) = \text{trace}(B \cdot A).$$

\begin{proof}
    \small
    \begin{align*}
        \text{trace}(A \cdot B) & = (ll \rightarrow , (ik,kj \rightarrow ij, A, B))  \\
                                & = (lk, kl \rightarrow ,A, B)                       \\
                                & = (kl, lk \rightarrow ,B, A)                       \\
                                & = (kk \rightarrow , (il, lj \rightarrow ij, B, A)) \\
                                & = \text{trace}(B \cdot A)
    \end{align*}
    where the second and fourth equality hold because of \cref{thm:nested_einsum:introduce_duplications},
    and the third equality holds because of the commutativity of multiplication in the standard semiring.
\end{proof}
\bigskip

\section{Removing Duplications}

The following example is an expression, which we cannot compress with the previous theorems:
$$(ij, kl, mn, ijklmn \rightarrow ijk, A, B, C, (abc \rightarrow aabbcc, D))$$
for $A \in \R^{x \times x}, B \in \R^{y \times y}, C \in \R^{z \times z}, D \in \R^{x \times y \times z}$.
This is because duplications in $\bm{s_u} = aabbcc$ are removed by the input string $\bm{\hat{s}_u} = ijklmn$.
In the following theorem, we will explore how to compress expressions such as this one.
Again, we use disjoint sets of symbols for the inner and outer expression to help us in the formulation and the proof.

\begin{theorem}
    \label{thm:nested_einsum:3}

    For $i \in [m + n]$, let $T^{(i)}$ be an $n_i$-th order tensor with index strings $\bm{s_i} \in S^{n_i}$.
    Let $\bm{s_u}$ be an index string for the $n_u$-th order tensor $U$, which is defined as follows:
    $$U := (\bm{s_{m + 1}},\dots,\bm{s_{m + n}} \rightarrow \bm{s_u}, T^{(m + 1)},\dots,T^{(m + n)})$$
    Also let $\bm{\hat{s}_u}$ be alternative index strings for $U$ with $s_{uj} \neq s_{uj'} \implies \hat{s}_{uj} \neq \hat{s}_{uj'}$ for all $j, j' \in [n_u]$,
    which means that $\bm{\hat{s}_u}$ can only remove symbol duplications, and cannot introduce any.
    Note that this is the converse of the constraint in \cref{thm:nested_einsum:2}.

    In our example, $\bm{s_u} = oopp$ and $\bm{\hat{s}_u} = jklm$.
    This removes the symbol duplication of the first and second index, as well as the symbol duplication of the third and fourth index.

    Let $s_v$ be an index string and
    $$V := (\bm{s_1},\dots,\bm{s_m}, \bm{\hat{s}_u} \rightarrow \bm{s_v}, T^{(1)},\dots,T^{(m)}, U)$$
    where the first and second Einsum expression share no symbols.
    Then these nested Einsum expressions can also be compressed into a single Einsum expression.

    As in \cref{thm:nested_einsum:2}, we need to apply a symbol map before substituting $\bm{\hat{s}_u}$.
    Interestingly, the symbol map is not applied to the index strings of the inner expression ($\bm{s_{m + 1}},\dots,\bm{s_{m + n}}$),
    but to the index strings of the outer expression ($\bm{s_1},\dots,\bm{s_m}$ and $\bm{s_v}$).
    Similarly, it does not map $\bm{s_u}$ to $\bm{\hat{s}_u}$, but $\bm{\hat{s}_u}$ to $\bm{s_u}$.

    Let $\nu: S \rightarrow S$ such that
    $$\nu(s) := \begin{cases}
            s_{uj} & \text{if }\exists j \in [n_u]: \hat{s}_{uj} = s \\
            s      & \text{else}
        \end{cases},$$
    which can be extended to map entire index strings as in \cref{thm:nested_einsum:2}.
    In our example, these are the important mappings:
    \begin{align*}
        i & \rightarrow a, & k & \rightarrow b, & m & \rightarrow c, \\
        j & \rightarrow a, & l & \rightarrow b, & n & \rightarrow c.
    \end{align*}
    This means that $i$ and $j$ will be iterated over at the same time, $k$ and $l$ will be iterated over at the same time, and $m$ and $l$ will be iterated over at the same time.

    Let $\bm{\hat{s}_i} := \nu(\bm{s_i})$ for $i \in [m]$, $\bm{\hat{s}_v} := \nu(\bm{s_v})$, then the compressed Einsum expression is the following:
    $$V = (\bm{\hat{s}_1},\dots,\bm{\hat{s}_m}, \bm{s_{m + 1}}, \dots, \bm{s_{m + n}} \rightarrow \bm{\hat{s}_v}, T^{(1)},\dots,T^{(m + n)})$$
    which helps us to compress the example:
    \begin{gather*}
        (ij, kl, mn, ijklmn \rightarrow ijk, A, B, C, (abc \rightarrow aabbcc, D))\\
        = (aa, bb, cc, abc \rightarrow aab, A, B, C, D).
    \end{gather*}
    Note how the index string for the output $\bm{s_v}$ was changed into $\bm{\hat{s}_v}$.
    This will become apparent in the proof.
\end{theorem}

\bigskip
\begin{proof}
    \small
    The key idea behind this proof, is that the entries of $U$, which were not defined in the computation, are set to the additive neutral element $\0$.
    This is useful, because in a semiring over some set $M$, the additive neutral element \textit{annihilates} $M$.
    This means, that for any $a \in M$, $a \cdot \0 = \0 \cdot a = \0$.
    Therefore, for any multi-index where $U$ is set to $\0$, $V$ is also set to $\0$.
    This means, that in the computation of $V$, only the indices which respect the duplications in $\bm{s_u}$ are defined.
    % Let us prove this formally.

    Let $F, F', B, B'$ be the free and bound symbols of the outer and inner Einsum expression respectively.
    W.l.o.g. they are all non-empty.
    From them we can derive the multi-index spaces $\mathcal{F}, \mathcal{F}', \mathcal{B}, \mathcal{B}'$ as in the definition.
    Then $U_{(\bm{f}, \bm{b}): \bm{\hat{s}_u}}$ is only non-zero for multi-indices $(\bm{f}, \bm{b}) \in \mathcal{F} \times \mathcal{B}$ with $(\bm{f}, \bm{b}):\hat{s}_{uj} = (\bm{f}, \bm{b}):\hat{s}_{uj'}$, where $j,j' \in [n_u]$ are indices of $\bm{s_u}$ where the symbols are duplicated, i.e. $s_{uj} = s_{uj'}$.
    In our example, this means that $(op \rightarrow oopp, B)$ is only non-zero for $(j,k,l,m) \in [d_j] \times [d_k] \times [d_l] \times [d_m]$ with $j = k$ and $l = m$, because $s_{u1} = s_{u2} = o$ and $s_{u3} = s_{u4} = p$.

    Therefore, when $U$ is multiplied with the other tensors, the resulting entry
    $$\bigodot\limits_{i = 1}^{m} T^{(i)}_{(\bm{f}, \bm{b}): \bm{s_i}} \odot U_{(\bm{f}, \bm{b}): \bm{\hat{s}_u}}$$
    is only non-zero for multi-indices $(\bm{f}, \bm{b}) \in \mathcal{F} \times \mathcal{B}$ that respect the same conditions.
    In our example, this is equivalent to
    $$A_{ij} B_{kl} C_{mn} \odot U_{ijklmn} = \begin{cases}
            A_{ij} B_{kl} C_{mn} \odot U_{ijklmn} & \text{if } i = j, k = l, m = n \\
            \0                                    & \text{else}
        \end{cases}.$$

    Now, this already looks like not all symbols are needed for this computation.
    But to see, in which way we can replace symbols, we need to consider the three ways in which duplications can be broken.
    Either a duplication is broken only by free symbols, only by bound symbols, or by a combination of both.
    In our example, we have all of these cases.
    The duplication $aa$ is broken by $i$ and $j$, which are both free symbols.
    The duplication $cc$ is broken by $m$ and $n$, which are both bound symbols.
    The duplication $bb$ is broken by $k$ and $l$, where $k$ is a free symbol and $l$ is a bound symbol.
    Every one of these cases leads to the same result, but in a slightly different way.

    First let us consider the case where a duplication is broken only by free symbols.
    In this case, the free symbols that break the duplication can be replaced by a single symbol,
    because all entries of $V$, with a multi-index that does not respect the duplication, is $\0$.
    In our example, this is equivalent to replacing $i$ and $j$ by a single symbol $a$:
    \begin{align*}
        \forall i,j,k: V_{ijk}    & = \bigoplus\limits_{l,m,n} A_{ij} B_{kl} C_{mn} \odot U_{ijklmn} \\
        \iff \forall a,k: V_{aak} & = \bigoplus\limits_{l,m,n} A_{aa} B_{kl} C_{mn} \odot U_{aaklmn}
    \end{align*}

    For the next two cases, we need to use that $a \oplus \0 = a$ for any $a \in M$.
    % This helps us to reduce the number of symbols that are used in the summation.
    This means, that only those summands that respect the duplications will be summed over,
    because all summands which do not respect the summation are $\0$.
    This affects the remaining two cases in different ways.
    If a duplication is broken only by bound symbols, then we need a single symbol to sum over all the multi-indices that respect the duplication.
    In our example, this is equivalent to replacing $m$ and $n$ by a single symbol $c$:
    $$\bigoplus\limits_{l,m,n} A_{aa} B_{kl} C_{mn} \odot U_{aaklmn} = \bigoplus\limits_{l,c} A_{aa} B_{kl} C_{cc} \odot U_{aaklcc}$$

    Now, if a duplication is broken by free symbols and by bound symbols,
    then the free symbols can again be replaced by a single symbol, as in the first case.
    In our example, this is useless, because there is already only one symbol $k$ where this can be applied.
    Let us do it anyway, because it might clear up what needs to be done in this step.
    \begin{align*}
        \forall a,k: V_{aak}      & = \bigoplus\limits_{l,c} A_{aa} B_{kl} C_{cc} \odot U_{aaklcc} \\
        \iff \forall a,b: V_{aab} & = \bigoplus\limits_{l,c} A_{aa} B_{bl} C_{cc} \odot U_{aablcc}
    \end{align*}
    Then, the values held by the breaking bound symbols are already defined by the value held by the now only breaking free symbol.
    Therefore, the summation over the values of this symbol is useless, because there is only one combination of indices, which respects the duplication.
    Therefore, the breaking bound symbols need to hold the same exact value as the breaking free symbol,
    and these symbols can be replaced by the same symbol, that was used to replace the free symbols.
    Because of this, the occurence of the symbol also needs to be removed from the sum.
    In our example, this is equivalent to replacing $l$ by $b$ and removing it from the sum:
    \begin{align*}
        \forall a,b: V_{aab} & = \bigoplus\limits_{l,c} A_{aa} B_{bl} C_{cc} \odot U_{aablcc} \\
                             & = \bigoplus\limits_{c} A_{aa} B_{bb} C_{cc} \odot U_{aabbcc}
    \end{align*}

    Therefore, in all three cases, the symbols, that break a duplication, can simply be replaced by a single symbol.
    Conveniently, as a replacing symbol, we can just use the symbol that defined the duplication in the first place,
    because the inner expression shares no symbols with the outer expression.
    This yields exactly the symbol map $\nu$.
    Therefore $\bm{\hat{s}_u}$, which is the index string of $U$ that was used in the outer expression, will be replaced by $\bm{s_u}$, which is the index string of $U$ that was used in the inner expression.
    Just as convenient is, that replacing the breaking symbols that include a combination of free and bound symbols, already removes the breaking bound symbols from the sum,
    because the bound symbols are by definition only those symbols, which are not free.
    In our example, this means that replacing $k$ and $l$ by $b$ already removes $l$ from the sum, because $b$ is now a free symbol as well.

    For the final steps, we need to define new multi-index sets to iterate and sum over.
    For this, let $\hat{F} := \sigma(\bm{\hat{s}_v})$ and $\hat{B} := \left(\bigcup_{i \in [m]} \sigma(\bm{\hat{s}_i}) \cup \bm{s_u}\right) \setminus \sigma(\bm{\hat{s}_v})$.
    Let $\mathcal{\hat{F}} = \prod_{s \in \hat{F}} [d_s]$ and $\mathcal{\hat{B}} = \prod_{s \in \hat{B}} [d_s]$.
    Then
    \begin{align*}
        V                                                                                 & = (\bm{s_1},\dots,\bm{s_m}, \bm{\hat{s}_u} \rightarrow \bm{s_v}, T^{(1)},\dots,T^{(m)}, U)                                                                                                \\
        \iff \forall \bm{f} \in \mathcal{F}: V_{\bm{f}: \bm{s_v}}                         & = \bigoplus\limits_{\bm{b} \in \mathcal{B}} \bigodot\limits_{i = 1}^{m} T^{(i)}_{(\bm{f}, \bm{b}):\bm{s_i}} \odot U_{(\bm{f}, \bm{b}):\bm{\hat{s}_u}}                                     \\
        \iff \forall \bm{\hat{f}} \in \mathcal{\hat{F}}: V_{\bm{\hat{f}}: \bm{\hat{s}_v}} & = \bigoplus\limits_{\bm{\hat{b}} \in \mathcal{\hat{B}}} \bigodot\limits_{i = 1}^{m} T^{(i)}_{(\bm{\hat{f}}, \bm{\hat{b}}):\bm{\hat{s}_i}} \odot U_{(\bm{\hat{f}}, \bm{\hat{b}}):\bm{s_u}} \\
        \iff V                                                                            & = (\bm{\hat{s}_1},\dots,\bm{\hat{s}_m}, \bm{s_u} \rightarrow \bm{\hat{s}_v}, T^{(1)},\dots,T^{(m)}, U)
    \end{align*}
    where the second equivalence holds because of the previously discussed symbol replacements.
    Then we can use \cref{thm:nested_einsum:1} for
    $$V = (\bm{\hat{s}_1}, \dots, \bm{\hat{s}_m}, \bm{s_{m + 1}}, \dots, \bm{s_{m + n}} \rightarrow \bm{\hat{s}_v}, T^{(1)}, \dots, T^{(m + n)})$$
    because the symbols of the outer expression have not been mapped to any of the bound symbols of the inner expression.
\end{proof}
\bigskip


% TODO: can i really write "every"?
With these two more specific theorems, we can write every naturally occuring complex expression from linear algebra as a single Einsum expression.
The reason for this is, that in linear algebra, only up to two indices are used for a single tensor,
which means that with two index strings, it cannot happen that a duplication is removed and anonther duplication is introduced simultaniously.
Here are some more complex examples of expressions, which we can compress with the two more specific theorems:
\begin{itemize}
    \item Squared norm of matrix-vector multiplication: Let $A \in \R^{m \times n}, v \in \R^{n}$. Then
          \begin{align*}
              \abs{A \cdot v}_2^2 & = (i,i\rightarrow,(ij, j \rightarrow i, A, v),(ij, j \rightarrow i, A, v)) \\
                                  & = (ij,j,ij,j\rightarrow,A,v,A,v).
          \end{align*}
    \item Trace of matrix-matrix multiplication: Let $A \in \R^{m \times n}, B \in \R^{n \times m}$. Then
          \begin{align*}
              \text{trace}(A \cdot B) & = (ii \rightarrow, (ik, kj \rightarrow ij, A, B)) \\
                                      & = (ik, ki \rightarrow, A, B).
          \end{align*}
    \item Matrix multiplication with a diagonal matrix: Let $A \in \R^{m \times n}, v \in \R^{n}$. Then
          \begin{align*}
              A \cdot \text{diag}(v) & = (ik, kj \rightarrow ij, A, (i \rightarrow ii, v)) \\
                                     & = (ij, j \rightarrow ij, A, v).
          \end{align*}
\end{itemize}

But writing every expression from linear algebra as a single Einsum expression respectively was already possible before \cite{Klaus2023}.
With these theorems, we just derived a different way of achieving that.
For this, we can state a simple procedure.
First, every function is translated to their respective Einsum expression, which results in a nested Einsum expression.
Then, the nested expressions are compressed from the bottom up.

But there are naturally occuring examples of nested Einsum expressions, which need more than the two duplication theorems to be compressed.
For instance, when symbolically deriving Einsum expressions to find out if they compute a convex function,
nested statements can arise, that break duplications in both directions.
An example of a function, where this happens in the second derivative, is the trace of a matrix where all entries have been squared:
$$(ii,ii \rightarrow, A, A)$$
for some $A \in \R^{n \times n}$.
Now, with \cref{thm:nested_einsum:general}, this procedure allows us to compress even these expressions.
Generally, it allows us to write every computation, where the single steps can be translated to Einsum over a single semiring, as one big Einsum expression that does not blow up in size with levels of nesting.