\subsection{Introducing Duplications}

The first duplication theorem handles all nested expressions, where the outer expression can only introduce duplications and cannot remove any.
This means that if two positions hold the same symbol in $\bm{s_u}$, these positions also have to hold the same symbol in $\bm{\hat{s}_u}$.
This brings the advantage, that the symbol map is much simpler and only has to be applied to the inner expression.

The following example is an expression, which respects this condition:
$$(ij, jjj \rightarrow i, A, (kl, lo \rightarrow kko, B, C))$$
for $A \in \R^{a \times b}, B \in \R^{b \times c}, C \in \R^{c \times b}$.
Again, we use disjoint sets of symbols for the inner and outer expression to help us in the formulation and the proof.

\begin{theorem}
    \label{thm:nested_einsum:introduce_duplications}

    For $i \in [m + n]$, let $T^{(i)}$ be an $n_i$-th order tensor with index strings $\bm{s_i} \in S^{n_i}$.
    Let $\bm{s_u}$ be an index string for the $n_u$-th order tensor $U$, which is defined as follows:
    $$U := (\bm{s_{m + 1}},\dots,\bm{s_{m + n}} \rightarrow \bm{s_u}, T^{(m + 1)},\dots,T^{(m + n)})$$
    Also let $\bm{\hat{s}_u}$ be alternative index strings for $U$ with $s_{uj} = s_{uj'} \implies \hat{s}_{uj} = \hat{s}_{uj'}$ for all $j, j' \in [n_u]$,
    which means that the outer expression can only introduce new symbol duplications, and cannot remove any.

    Let $s_v$ be an index string and
    $$V := (\bm{s_1},\dots,\bm{s_m}, \bm{\hat{s}_u} \rightarrow \bm{s_v}, T^{(1)},\dots,T^{(m)}, U)$$
    such that the first and second Einsum expression share no symbols.
    Then these nested Einsum expressions can also be compressed into a single Einsum expression.

    As in \cref{thm:nested_einsum:general}, we need to apply a symbol map in order to compress the nested expression.
    Let $\nu: S \rightarrow S$ such that
    $$\nu(s) := \begin{cases}
            \hat{s}_{uj} & \text{if $\exists j \in [n_u]: s_{uj} = s$} \\
            s            & \text{else}
        \end{cases}$$
    which maps symbols in $\bm{s_u}$ to the symbol at the same index in $\bm{\hat{s}_u}$ and all other symbols to themselves.

    In our example, we have the following symbols on the same positions:
    \begin{itemize}
        \item $s_{u1} = k$ and $\hat{s}_{u1} = j$,
        \item $s_{u2} = k$ and $\hat{s}_{u2} = j$,
        \item $s_{u3} = o$ and $\hat{s}_{u3} = j$.
    \end{itemize}
    Therefore these are the important mappings:
    \begin{align*}
        k & \rightarrow j, \\
        o & \rightarrow j.
    \end{align*}

    The symbol map $\nu$ can be extended, such that it maps entire index strings instead of just symbols as in \cref{thm:nested_einsum:general}.
    Then we can write the substituted index strings by setting $\bm{\hat{s}_i} := \nu(\bm{s_i})$ for $i \in [m + 1, m + n]$.
    With these index strings, the compressed Einsum expression is the following:
    $$V = (\bm{s_1},\dots,\bm{s_m}, \bm{\hat{s}_{m + 1}}, \dots, \bm{\hat{s}_{m + n}} \rightarrow \bm{s_v}, T^{(1)},\dots,T^{(m + n)})$$
    which helps us to compress the example:
    $$(ij, jjj \rightarrow i, A, (kl, lo \rightarrow kko, B, C)) = (ij, jl, lj \rightarrow i, A, B, C).$$
\end{theorem}

\begin{proof}
    \small
    Because $s_{uj} = s_{uj'} \implies \hat{s}_{uj} = \hat{s}_{uj'}$ for all $j, j' \in [n_u]$,
    each symbol in $\bm{s_u}$ only has one neighbour in the symbol graph.
    This is illustrated for our example in \cref{fig:nested_expressions:example_symbol_graph_intro}
    Therefore each symbol in $\bm{\hat{s}_u}$ defines its own component,
    and can be used as the symbol in the symbol map, that replaces all symbols in their component.
    Therefore the symbols in $\bm{\hat{s}_u}$ are not changed by applying the symbol map, which are the only symbols in the outer expression that could have been changed by the map.
    Therefore the symbol map only has to be applied to the inner expression.

    \begin{figure}[h]
        \centering
        \begin{tikzpicture}[node distance = 3cm, semithick]

            \node[state] (k)		            {$k$};
            \node[state] (o) [right of=k] 		{$o$};
            \node (middle) at ($(k)!0.5!(o)$)   {};
            \node[state] (j) [above of=middle]	{$j$};

            \path (j) edge (k);
            \path (j) edge (o);

        \end{tikzpicture}
        \caption{Symbol graph for the example}
        \label{fig:nested_expressions:example_symbol_graph_intro}
    \end{figure}
\end{proof}
\bigskip