\section{Introducing Duplications}

The following example is another expression, which we cannot compress with the previous theorem:
$$(ij, jjj \rightarrow i, A, (kl, lo \rightarrow kko, B, C))$$
for $A \in \R^{a \times b}, B \in \R^{b \times c}, C \in \R^{c \times b}$.
In the following, we will explore how to compress such expressions.
Note that, for the theorem, we use disjoint sets of symbols for the inner and outer expression.
This helps in the proof, and is not a real constraint in practice,
because we can just rename the symbols in different scopes.
For example, we could also write the above expression as
$$(ij, jjj \rightarrow i, A, (ik, kj \rightarrow iij, B, C)),$$
because the scope of each symbol does not reach into nested expressions,
and therefore the $j$ used in the outer expression and the $j$ used in the inner expression are treated as different symbols.

\begin{theorem}
    \label{thm:nested_einsum:2}

    For $i \in [m + n]$, let $T^{(i)}$ be an $n_i$-th order tensor with index strings $\bm{s_i} \in S^{n_i}$.
    Let $\bm{s_u}$ be an index string for the $n_u$-th order tensor $U$, which is defined as follows:
    $$U := (\bm{s_{m + 1}},\dots,\bm{s_{m + n}} \rightarrow \bm{s_u}, T^{(m + 1)},\dots,T^{(m + n)})$$
    Also let $\bm{\hat{s}_u}$ be alternative index strings for $U$ with $s_{uj} = s_{uj'} \implies \hat{s}_{uj} = \hat{s}_{uj'}$ for all $j, j' \in [n_u]$,
    which means that $\bm{\hat{s}_u}$ can only introduce new symbol duplications, and cannot remove any.
    The index string $\bm{s_u}$ corresponds to the output string of the inner expression,
    and the index string $\bm{\hat{s}_u}$ corresponds to the input string that is used for the input tensor $U$ in the outer expression.

    In our example, $\bm{s_u} = kko$ and $\bm{\hat{s}_u} = jjj$.
    This does not break the symbol duplication of the first and second index,
    and introduces a new duplication on the third index.

    Let $s_v$ be an index string and
    $$V := (\bm{s_1},\dots,\bm{s_m}, \bm{\hat{s}_u} \rightarrow \bm{s_v}, T^{(1)},\dots,T^{(m)}, U)$$
    such that the first and second Einsum expression share no symbols.
    Then these nested Einsum expressions can also be compressed into a single Einsum expression.

    In contrast to \cref{thm:nested_einsum:1}, we cannot just replace the input index string $\bm{\hat{s}_u}$ by all the input index strings in the inner Einsum expression $\bm{s_{m + 1}},\dots,\bm{s_{m + n}}$.
    Instead, we first need to apply a symbol map to the input strings of the inner expression.
    Let $\nu: S \rightarrow S$ such that
    $$\nu(s) := \begin{cases}
            \hat{s}_{uj} & \text{if }\exists j \in [n_u]: s_{uj} = s \\
            s            & \text{else}
        \end{cases}$$
    which maps symbols in $\bm{s_u}$ to the symbol at the same index in $\bm{\hat{s}_u}$ and all other symbols to themselves.

    This symbol map holds information about which symbols will be iterated over at the same time in the outer expression.
    In our example, we have the following symbols on the same positions:
    \begin{itemize}
        \item $s_{u1} = k$ and $\hat{s}_{u1} = j$,
        \item $s_{u2} = k$ and $\hat{s}_{u2} = j$,
        \item $s_{u3} = o$ and $\hat{s}_{u3} = j$.
    \end{itemize}
    Therefore these are the important mappings:
    \begin{align*}
        k & \rightarrow j, \\
        o & \rightarrow j.
    \end{align*}
    This means that $k$ and $o$ will be iterated over at the same time.

    The symbol map $\nu$ can be extended, such that it maps entire index strings instead of just symbols, by setting $\nu(\bm{s_i}) \in S^{n_i}, \nu(\bm{s_i})_j := \nu(s_{ij})$.
    Then we can write the substituted index strings by setting $\bm{\hat{s}_i} := \nu(\bm{s_i})$ for $i \in [m + 1, m + n]$.

    The compressed Einsum expression now becomes the following:
    $$V = (\bm{s_1},\dots,\bm{s_m}, \bm{\hat{s}_{m + 1}}, \dots, \bm{\hat{s}_{m + n}} \rightarrow \bm{s_v}, T^{(1)},\dots,T^{(m + n)})$$
    which helps us to compress the example:
    $$(ij, jjj \rightarrow i, A, (kl, lo \rightarrow kko, B, C)) = (ij, jl, lj \rightarrow i, A, B, C).$$
\end{theorem}

\bigskip
\begin{proof}
    \small
    The fundamental idea behind this theorem is, that by using the index string $\bm{\hat{s}_u}$, we only iterate over a sub-space of the indices that we defined for the computation of $U$.
    The way in which we iterate over this sub-space is determined by the outer expression.
    It could either be the sum over bound indices or the universal quantifier over free indices.
    To formulate this, we need some idea of which multi-indices we iterate over.
    Therefore, let $\mathcal{I}:\bm{s} := \smallset{\bm{i}: \bm{s} \mid \bm{i} \in \mathcal{I}}$ for an index string $\bm{s}$ and a multi-index space $\mathcal{I}$.

    Let $F', B'$ be the free and bound symbols of the inner Einsum expression.
    W.l.o.g. they are both non-empty.
    From them we can derive the multi-index spaces $\mathcal{F}', \mathcal{B}'$ as in the definition.
    Let $\hat{F}' = \sigma(\bm{\hat{s}_u})$ and $\mathcal{\hat{F}}' = \prod_{s \in \hat{F}'} [d_s]$.
    Then $\mathcal{\hat{F}}':\bm{\hat{s}_u} \subseteq \mathcal{F}':\bm{s_u}$.
    This follows from $d_{s_{uj}} = d_{\hat{s}_{uj}}$ for $j \in [n_u]$,
    and the amount of symbols in the projection of $\mathcal{\hat{F}}':\bm{\hat{s}_u}$ being smaller or equal to the amount of symbols in the projection of $\mathcal{F}':\bm{s_u}$.
    The first fact is true per the definition of Einsum.
    The second fact can be rewritten as $\abs{\sigma(\bm{\hat{s}_u})} \leq \abs{\sigma(\bm{s_u})}$ and follows directly from the constraint $s_{uj} = s_{uj'} \implies \hat{s}_{uj} = \hat{s}_{uj'}$ for all $j,j' \in [n_u]$.

    Then
    $$\forall \bm{f'} \in \mathcal{F}': U_{\bm{f'}: \bm{s_u}} = \bigoplus\limits_{\bm{b'} \in \mathcal{B}'}\bigodot\limits_{i = m + 1}^{m + n} T^{(i)}_{(\bm{f'}, \bm{b'}):\bm{s_{i}}}$$
    and therefore
    $$\forall \bm{\hat{f}'} \in \mathcal{\hat{F}}': U_{\bm{\hat{f}}': \bm{\hat{s}_u}} = \bigoplus\limits_{\bm{b'} \in \mathcal{B}'}\bigodot\limits_{i = m + 1}^{m + n} T^{(i)}_{(\bm{\hat{f}}', \bm{b'}):\bm{\hat{s}_{i}}}$$
    because of the previous observation,
    and because the bound symbols of the expression, which are used in $\bm{b'}$, do not occur in $\bm{s_u}$, and are therefore not changed by the symbol map $\nu$.
    Therefore
    $$U = (\bm{\hat{s}_{m + 1}},\dots,\bm{\hat{s}_{m + n}} \rightarrow \bm{\hat{s}_u}, T^{(m + 1)},\dots,T^{(m + n)})$$
    and we can use \cref{thm:nested_einsum:1} for
    $$V = (\bm{s_1}, \dots, \bm{s_m}, \bm{\hat{s}_{m + 1}}, \dots, \bm{\hat{s}_{m + n}} \rightarrow \bm{s_v}, T^{(1)}, \dots, T^{(m + n)})$$
    because the bound symbols of the inner expression have not been mapped to any of the symbols used in the outer expression.
\end{proof}
\bigskip