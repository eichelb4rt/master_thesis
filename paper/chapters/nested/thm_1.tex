\section{Simple Nested Expressions}

\begin{theorem}
    \label{thm:nested_einsum:1}

    For $i \in [m + n]$, let $T^{(i)}$ be an $n_i$-th order tensor with index strings $\bm{s_i} \in S^{n_i}$.
    Let $\bm{s_u}, \bm{s_v}$ be index strings.
    Let
    $$U := (\bm{s_{m + 1}},\dots,\bm{s_{m + n}} \rightarrow \bm{s_u}, T^{(m + 1)},\dots,T^{(m + n)})$$
    and
    $$V := (\bm{s_1},\dots,\bm{s_m}, \bm{s_u} \rightarrow \bm{s_v}, T^{(1)},\dots,T^{(m)}, U)$$
    where the free symbols of the second Einsum expression share no symbols with the first Einsum expression.
    Then
    $$V = (\bm{s_1}, \dots, \bm{s_{m + n}} \rightarrow \bm{s_v}, T^{(1)}, \dots, T^{(m + n)})$$
\end{theorem}

\begin{proof}
    \small
    Let $F, F', B, B'$ be the free and bound symbols of the second (outer) and first (inner) einsum expression respectively.
    W.l.o.g. they are all non-empty.
    From them we can derive $\mathcal{F}, \mathcal{F}', \mathcal{B}, \mathcal{B}'$ as in the definition.
    Then
    \begin{align*}
        V                                                         & = (\bm{s_1},\dots,\bm{s_m}, \bm{s_u} \rightarrow \bm{s_v}, T^{(1)},\dots,T^{(m)}, U)                                                                                                                                                                \\
        \iff \forall \bm{f} \in \mathcal{F}: V_{\bm{f}: \bm{s_v}} & = \bigoplus\limits_{\bm{b} \in \mathcal{B}} \bigodot\limits_{i = 1}^{m} T^{(i)}_{(\bm{f}, \bm{b}):\bm{s_k}} \odot U_{(\bm{f}, \bm{b}):\bm{s_u}}                                                                                                     \\
                                                                  & = \bigoplus\limits_{\bm{b} \in \mathcal{B}} \bigodot\limits_{i = 1}^{m} T^{(i)}_{(\bm{f}, \bm{b}):\bm{s_i}} \odot \bigoplus\limits_{\bm{b'} \in \mathcal{B}'} \bigodot\limits_{i' = m + 1}^{m + n} T^{(i')}_{(\bm{f}, \bm{b}, \bm{b'}):\bm{s_{i'}}} \\
                                                                  & = \bigoplus\limits_{\bm{b} \in \mathcal{B}} \bigoplus\limits_{\bm{b'} \in \mathcal{B}'} \bigodot\limits_{i = 1}^{m} T^{(i)}_{(\bm{f}, \bm{b}):\bm{s_i}} \odot \bigodot\limits_{i = m + 1}^{m + n} T^{(i)}_{(\bm{f}, \bm{b}, \bm{b'}):\bm{s_{i}}}                                      \\
                                                                  & = \bigoplus\limits_{\bm{b} \in \mathcal{B} \times \mathcal{B}'} \bigodot\limits_{i = 1}^{m + n} T^{(i)}_{(\bm{f}, \bm{b}):\bm{s_i}}                                                                                                                               \\
        \iff V                                                    & = (\bm{s_1}, \dots, \bm{s_{m + n}} \rightarrow \bm{s_v}, T^{(1)}, \dots, T^{(m + n)})
    \end{align*}
    where the third equality follows from
    $$\forall \bm{f'} \in \mathcal{F}': U_{\bm{f'}: \bm{s_u}} = \bigoplus\limits_{\bm{b'} \in \mathcal{B}'} \bigodot\limits_{i' = m + 1}^{m + n} T^{(i')}_{(\bm{f'}, \bm{b'}):\bm{s_{i'}}},$$
    $F' \subseteq B \cup F$, and $(B \cup F) \cap B' = \emptyset$. The last two facts are required so that $(\bm{f}, \bm{b}, \bm{b'}):\bm{s_{i'}}$ is well-defined and projects on the same indices as $(\bm{f'}, \bm{b'}):\bm{s_{i'}}$.
    The fourth equality follows from the distributivity in a semiring.
\end{proof}
\bigskip