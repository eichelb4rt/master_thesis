\begin{lemma}
    \label{lemma:nested_einsum:3}
    For $i \in [n]$, let $T^{(i)}$ be an $n_i$-th order tensor with index string $\bm{s_i} \in S^{n_i}$,
    where $T^{(n)} = \1_m$ for some $m \in \N$.
    Let $\bm{s_t}$ be the index string for $T$ with
    $$T = (\bm{s_1}, \dots, \bm{s_n} \rightarrow \bm{s_t}, T^{(1)}, \dots, T^{(n)}).$$

    Let $F$ and $B$ be the free and bound symbols of this expression.
    Then we can introduce a symbol map $\mu: S \rightarrow S$, which maps both symbols in $\bm{s_n}$ to the same symbol $s_{\text{new}} \in S \setminus (F \cup B)$:
    $$\mu(s) := \begin{cases}
            s_{\text{new}} & \text{if $s \in \smallset{s_{n1}, s_{n2}}$} \\
            s              & \text{else}
        \end{cases}$$
    The symbol map $\mu$ can be extended, such that it maps entire index strings instead of just symbols, by setting $\mu(\bm{s_i}) \in S^{n_i}, \mu(\bm{s_i})_j := \mu(s_{ij})$.
    Then we can write the substituted index strings by setting $\bm{s'_i} := \mu(\bm{s_i})$ for $i \in [n]$ and $\bm{s'_t} = \mu(\bm{s_t})$.
    With these index strings, the following holds:
    $$T = (\bm{s'_1}, \dots, \bm{s'_{n - 1}} \rightarrow \bm{s'_t}, T^{(1)}, \dots, T^{(n - 1)}).$$
\end{lemma}

\begin{proof}
    \small
    For this proof, we provide the following example of an Einsum expression on which we demonstrate the given arguments for better understanding:
    $$(ij, kl, mn, ij, kl, mn \rightarrow imn, A, B, C, \1_a, \1_b, \1_c)$$
    for $A \in \R^{a \times a}, B \in \R^{b \times b}, C \in \R^{c \times c}$ and some $a,b,c \in \N$.

    We need to consider three cases for the symbols used in the index string $\bm{s_n} = (s_{n1}, s_{n2})$:
    \begin{itemize}
        \item $s_{n1}$ and $s_{n2}$ are both free symbols,
        \item $s_{n1}$ and $s_{n2}$ are both bound symbols,
        \item one symbol of $s_{n1}$ and $s_{n2}$ is a free symbol, the other is a bound symbol.
    \end{itemize}
    Every one of these cases leads to the same result, but in a slightly different way.

    First let us consider the case where both symbols are free.
    In this case, both symbols can be replaced by a single symbol,
    because $T$ is $\0$ for all entries with a multi-index,
    where the indices projected by the symbols are not equal.

    In our example, this is equivalent to the following:
    \begin{align*}
        \forall i,m,n: T_{imn}    & = \bigoplus\limits_{j,k,l} A_{ij} B_{kl} C_{mn} \left(\1_a\right)_{ij} \left(\1_b\right)_{kl} \left(\1_c\right)_{mn} \\
                                  & = \begin{cases}
            \bigoplus\limits_{j,k,l} A_{ij} B_{kl} C_{mn} \left(\1_a\right)_{ij} \left(\1_b\right)_{kl} & \text{if $m = n$} \\
            \0                                                                                          & \text{else}
        \end{cases}                                                                                          \\
        \iff \forall i,z: T_{izz} & = \bigoplus\limits_{j,k,l} A_{ij} B_{kl} C_{zz} \left(\1_a\right)_{ij} \left(\1_b\right)_{kl}.
    \end{align*}

    Next let us consider the case where both symbols are bound.
    In this case, those summands are multiplied with $\0$,
    which have a multi-index where the projected indices are not equal.
    Therefore, those summands are annihilated and left out from the summation.
    This means that both symbols can be replaced by a single symbol.

    In our example, this is equivalent to the following:
    \begin{align*}
        \forall i,z: T_{izz} & = \bigoplus\limits_{j,k,l} A_{ij} B_{kl} C_{zz} \left(\1_a\right)_{ij} \left(\1_b\right)_{kl}          \\
                             & = \bigoplus\limits_{j,k,l} A_{ij} B_{kl} C_{zz} \left(\1_a\right)_{ij} \odot \begin{cases}
            \1 & \text{if $k = l$} \\
            \0 & \text{else}
        \end{cases} \\
                             & = \bigoplus\limits_{j,y} A_{ij} B_{yy} C_{zz} \left(\1_a\right)_{ij}.
    \end{align*}

    Next let us consider the case where one symbol is free and one symbol is bound.
    W.l.o.g. we consider the case where $s_{n1}$ is free and $s_{n2}$ is bound.
    In this case, those summands are multiplied with $\0$,
    which have a multi-index where the index projected by the bound symbol $s_{n2}$ is not the same as the index projected by the free symbol $s_{n1}$.
    Therefore those summands are annihilated and left out from the summation, and the symbol $s_{n2}$ can be replaced by the symbol $s_{n1}$.
    Additionally, we can rename the $s_{n1}$ to some new symbol.

    In our example, this is equivalent to the following:
    \begin{align*}
        \forall i,z: T_{izz}      & = \bigoplus\limits_{j,y} A_{ij} B_{yy} C_{zz} \left(\1_a\right)_{ij}          \\
                                  & = \bigoplus\limits_{j,y} A_{ij} B_{yy} C_{zz} \odot \begin{cases}
            \1 & \text{if $i = j$} \\
            \0 & \text{else}
        \end{cases} \\
                                  & = \bigoplus\limits_{y} A_{ii} B_{yy} C_{zz}                                   \\
        \iff \forall x,z: T_{xzz} & = \bigoplus\limits_{y} A_{xx} B_{yy} C_{zz}.
    \end{align*}

    Therefore, in all three cases, the symbols, that are used in an index string for a unity matrix, can simply be replaced by a single symbol.
\end{proof}