\section{Removing Duplications}

In the following theorem, we explore a way of compressing the expression
$$(ijk, jk \rightarrow ijk, A, (l \rightarrow ll, v))$$
for $A \in \R^{a \times b}, v \in \R^b$.
Again, we use disjoined sets of symbols for the inner and outer expression to help us in the formulation and the proof.

\begin{theorem}
    \label{thm:nested_einsum:3}

    For $i \in [m + n]$, let $T^{(i)}$ be an $n_i$-th order tensor with index strings $\bm{s_i} \in S^{n_i}$.
    Let $\bm{s_u}$ be an index string for the $n_u$-th order tensor $U$, which is defined as follows:
    $$U := (\bm{s_{m + 1}},\dots,\bm{s_{m + n}} \rightarrow \bm{s_u}, T^{(m + 1)},\dots,T^{(m + n)})$$
    Also let $\bm{\hat{s}_u}$ be alternative index strings for $U$ with $s_{uj} \neq s_{uj'} \implies \hat{s}_{uj} \neq \hat{s}_{uj'}$ for all $j, j' \in [n_u]$,
    which means that $\bm{\hat{s}_u}$ can only remove symbol duplications, and cannot introduce any.
    Note that this is the converse of the constraint in \autoref{thm:nested_einsum:2}.

    In our example, $\bm{s_u} = ll$ and $\bm{\hat{s}_u} = jk$.
    This removes the symbol duplication of the first and second index.

    Let $s_v$ be an index string and
    $$V := (\bm{s_1},\dots,\bm{s_m}, \bm{\hat{s}_u} \rightarrow \bm{s_v}, T^{(1)},\dots,T^{(m)}, U)$$
    where the first and second Einsum expression share no symbols.
    Then these nested Einsum expressions can also be compressed into a single Einsum expression.

    As in \autoref{thm:nested_einsum:3}, we need to apply a symbol map before substituting $\bm{\hat{s}_u}$.
    Interestingly, the symbol map is not applied to the index strings in the computation of $U$ ($\bm{s_{m + 1}},\dots,\bm{s_{m + n}}$),
    but to the index strings in the computation of $V$ ($\bm{s_1},\dots,\bm{s_m}$).
    Similarly, it does not map $\bm{s_u}$ to $\bm{\hat{s}_u}$, but $\bm{\hat{s}_u}$ to $\bm{s_u}$.

    Let $\nu: S \rightarrow S$ such that
    $$\nu(s) := \begin{cases}
            s_{uj} & \text{if }\exists j \in [n_u]: \hat{s}_{uj} = s \\
            s      & \text{else}
        \end{cases},$$
    which can be extended to map entire index strings as in \autoref{thm:nested_einsum:2}.

    Let $\bm{\hat{s}_i} := \nu(\bm{s_i})$ for $i \in [m]$, $\bm{\hat{s}_v} := \nu(\bm{s_v})$, then the compressed Einsum expression is the following:
    $$V = (\bm{\hat{s}_1},\dots,\bm{\hat{s}_m}, \bm{s_{m + 1}}, \dots, \bm{s_{m + n}} \rightarrow \bm{\hat{s}_v}, T^{(1)},\dots,T^{(m + n)})$$
    which helps us to compress the example:
    $$(ijk, jk \rightarrow ijk, A, (l \rightarrow ll, v)) = (ill, l \rightarrow ill, A, v)$$
    Note how even the index string for the output $\bm{s_v}$ was changed into $\bm{\hat{s}_v}$.
    This will become apparent in the proof.
\end{theorem}

\begin{proof}
    \small
    The key idea behind this proof, is that the entries of $U$, which were not defined in the computation, are set to the additive neutral element $\0$.
    This is useful, because in a semiring over some set $M$, the additive neutral element \textit{annihilates} $M$.
    This means, that for any $a \in M$, $a \cdot \0 = \0 \cdot a = \0$.
    Therefore, for any multiindex where $U$ is set to $\0$, $V$ is also set to $\0$.
    This means, that in the computation of $V$, only the indices which follow the duplications in $\bm{s_u}$ are defined.
    % Let us prove this formally.

    Let $F, F', B, B'$ be the free and bound symbols of the second (outer) and first (inner) einsum expression respectively.
    W.l.o.g. they are all non-empty.
    From them we can derive $\mathcal{F}, \mathcal{F}', \mathcal{B}, \mathcal{B}'$ as in the definition.
    Then
    $$U_{(\bm{f}, \bm{b}): \bm{\hat{s}_u}} = \begin{cases}
            U_{(\bm{f}, \bm{b}): \bm{s_u}} & \text{if } (\bm{f}, \bm{b}):s_{uj} = (\bm{f}, \bm{b}):s_{uj'} \text{ for } j,j' \in [n_u] \text{ with } s_{uj} = s_{uj'} \\
            \0                             & \text{else}
        \end{cases}$$
    for all $(\bm{f}, \bm{b}) \in \mathcal{F} \times \mathcal{B}$.
    Therefore,
    $$T^{(i)}_{(\bm{f}, \bm{b}): \bm{s_i}} \odot U_{(\bm{f}, \bm{b}): \bm{\hat{s}_u}} = \begin{cases}
            T^{(i)}_{(\bm{f}, \bm{b}): \bm{\hat{s}_i}} \odot U_{(\bm{f}, \bm{b}): \bm{s_u}} & \text{if } (\bm{f}, \bm{b}):s_{uj} = (\bm{f}, \bm{b}):s_{uj'} \\
                                                                                            & \text{for } j,j' \in [n_u] \text{ with } s_{uj} = s_{uj'}     \\
            \0                                                                              & \text{else}
        \end{cases}$$
    for all $(\bm{f}, \bm{b}) \in \mathcal{F} \times \mathcal{B}$ and $i \in [m + n]$,
    because $(\bm{f}, \bm{b}): \bm{s_i} = (\bm{f}, \bm{b}): \bm{\hat{s}_i}$ if $(\bm{f}, \bm{b}):s_{uj} = (\bm{f}, \bm{b}):s_{uj'}$ for $j,j' \in [n_u]$ with $s_{uj} = s_{uj'}$,
    because the equality of $(\bm{f}, \bm{b}):s_{uj}$ and $(\bm{f}, \bm{b}):s_{uj'}$ reproduces the duplications in $\bm{s_u}$ that were removed in $\bm{\hat{s}_u}$.
    (THINK ABOUT FORMULATION!)

    For the same reasons,
    $$\bigodot\limits_{i = 1}^{m} T^{(i)}_{(\bm{f}, \bm{b}): \bm{s_i}} \odot U_{(\bm{f}, \bm{b}): \bm{\hat{s}_u}} = \begin{cases}
            \bigodot\limits_{i = 1}^{m} T^{(i)}_{(\bm{f}, \bm{b}): \bm{\hat{s}_i}} \odot U_{(\bm{f}, \bm{b}): \bm{s_u}} & \text{if } (\bm{f}, \bm{b}):s_{uj} = (\bm{f}, \bm{b}):s_{uj'} \\
                                                                                                                        & \text{for } j,j' \in [n_u] \text{ with } s_{uj} = s_{uj'}     \\
            \0                                                                                                          & \text{else}
        \end{cases}$$
    for all $(\bm{f}, \bm{b}) \in \mathcal{F} \times \mathcal{B}$, and
    $$V_{\bm{f}: \bm{s_u}} = \begin{cases}
            V_{\bm{f}: \bm{\hat{s}_u}} & \text{if } \bm{f}:s_{uj} = \bm{f}:s_{uj'} \text{ for } j,j' \in [n_u] \\
                                       & \text{with } s_{uj} = s_{uj'} \text{ and } s_{uj} \in F               \\
            \0                         & \text{else}
        \end{cases}$$
    for all $\bm{f} \in \mathcal{F}$.

    Then
    \begin{align*}
        V                                                               & = (\bm{s_1},\dots,\bm{s_m}, \bm{\hat{s}_u} \rightarrow \bm{s_v}, T^{(1)},\dots,T^{(m)}, U)                                                                                                                          \\
        \iff \forall \bm{f} \in \mathcal{F}: V_{\bm{f}: \bm{s_v}}       & = \bigoplus\limits_{\bm{b} \in \mathcal{B}} \bigodot\limits_{i = 1}^{m} T^{(i)}_{(\bm{f}, \bm{b}):\bm{s_i}} \odot U_{(\bm{f}, \bm{b}):\bm{\hat{s}_u}}                                                               \\
        \iff \forall \bm{f} \in \mathcal{F}: V_{\bm{f}: \bm{\hat{s}_v}} & = \bigoplus\limits_{\bm{b} \in \mathcal{B}} \bigodot\limits_{i = 1}^{m} T^{(i)}_{(\bm{f}, \bm{b}):\bm{\hat{s}_i}} \odot U_{(\bm{f}, \bm{b}):\bm{s_u}}                                                               \\
                                                                        & = \bigoplus\limits_{\bm{b} \in \mathcal{B} \times \mathcal{B}'} \bigodot\limits_{i = 1}^{m} T^{(i)}_{(\bm{f}, \bm{b}):\bm{\hat{s}_i}} \odot \bigodot\limits_{i = m + 1}^{m + n} T^{(i)}_{(\bm{f}, \bm{b}):\bm{s_i}} \\
        \iff V                                                          & = (\bm{s_1}, \dots, \bm{s_m}, \bm{\hat{s}_{m + 1}}, \dots, \bm{\hat{s}_{m + n}} \rightarrow \bm{s_v}, T^{(1)}, \dots, T^{(m + n)})
    \end{align*}
    where the second equivalence and third equality hold because of the facts shown above.
    The fourth equality just skips the steps already shown in \autoref{thm:nested_einsum:1}.
\end{proof}
\bigskip