\section{General Nested Expressions}

The following example is an expression, which we cannot compress with the previous theorem:
$$(a,b,c,d,e,abbcde \rightarrow bc, v^{(1)}, v^{(2)}, v^{(3)}, v^{(4)}, v^{(5)}, (
    i,j,k,l \rightarrow iijkkl, v^{(6)}, v^{(7)}, v^{(8)}, v^{(9)}
    ))$$
for $v^{(i)} \in \R^{d_{vi}}$ with $i \in [9]$, where $d_{vi} \in \N$ are appropriate dimensions.
This is because the output string $\bm{s_u} = iijkkl$ and the input string $\bm{\hat{s}_u} = abbde$ are not the same.
In the following, we will explore how to compress such expressions.
Note that, for the theorem, we use disjoint sets of symbols for the inner and outer expression.
This helps in the proof, and is not a real constraint in practice,
because we can just rename the symbols in different scopes.
For example, we could also write the above expression as
$$(ij, jjj \rightarrow i, A, (ik, kj \rightarrow iij, B, C)),$$
because the scope of each symbol does not reach into nested expressions,
and therefore the $j$ used in the outer expression and the $j$ used in the inner expression are treated as different symbols.

\begin{theorem}
    \label{thm:nested_einsum:4}
    For $i \in [m + n]$, let $T^{(i)}$ be an $n_i$-th order tensor with index string $\bm{s_i} \in S^{n_i}$.
    Let $\bm{s_u}$ be an index string for the $n_u$-th order tensor $U$, which is defined as follows:
    $$U := (\bm{s_{m + 1}},\dots,\bm{s_{m + n}} \rightarrow \bm{s_u}, T^{(m + 1)},\dots,T^{(m + n)})$$
    Also let $\bm{\hat{s}_u}$ be alternative index strings for $U$.

    Let $s_v$ be an index string and
    $$V := (\bm{s_1},\dots,\bm{s_m}, \bm{\hat{s}_u} \rightarrow \bm{s_v}, T^{(1)},\dots,T^{(m)}, U)$$
    where the first and second Einsum expression share no symbols.
    Then these nested Einsum expressions can also be compressed into a single Einsum expression.

    In contrast to \autoref{thm:nested_einsum:1}, we cannot just replace the input index string $\bm{\hat{s}_u}$ by all the input index strings in the inner Einsum expression $\bm{s_{m + 1}},\dots,\bm{s_{m + n}}$.
    Instead, we first need to apply a symbol map $\nu: S \rightarrow S$ to each of the index strings.
    This symbol map holds information about which symbols are effectively the used for the same index.

    For the definition of the map $\nu$, we first construct an undirected graph $G = (V, E)$ that we call \textit{symbol graph}.
    % $V = \sigma(\bm{s_v}) \cup \sigma(\bm{s_u}) \cup \sigma(\bm{\hat{s}_u}) \cup \bigcup_{i \in [m + n]} \sigma(\bm{s_i})$
    In the symbol graph, the nodes consist of all symbols from both expressions.
    The edges are $E = \smallset{\smallset{s_{uj}, \hat{s}_{uj}} \mid j \in [n_u]}$,
    which connects all symbols from $\bm{s_u}$ and $\bm{\hat{s}_u}$ that share an index.
    The symbol graph for our example is illustrated in \autoref{fig:nested_expressions:example_symbol_graph}.

    \begin{figure}[h]
        \centering
        \begin{tikzpicture}[node distance = 3cm, semithick]

            \node[state] (a)					{$a$};
            \node[state] (b) [right of=a]		{$b$};
            \node[state] (c) [right of=b] 		{$c$};
            \node[state] (d) [right of=c] 		{$d$};
            \node[state] (e) [right of=d] 		{$e$};
            \node (middle) at ($(a)!0.5!(b)$) {};
            \node[state] (i) [below of=middle] 		{$i$};
            \node[state] (j) [right of=i] 		{$j$};
            \node[state] (k) [right of=j] 		{$k$};
            \node[state] (l) [right of=k] 		{$l$};

            \node (x_center) at ($(a)!0.5!(j)$) {};
            \node (y_center) at ($(c)!0.5!(d|-k)$) {};
            \node (z_center) at ($(e)!0.5!(l)$) {};
            \node (x) [below of=x_center] {$x$};
            \node (y) [below of=y_center] {$y$};
            \node (z) [below of=z_center] {$z$};

            \path (a) edge (i);
            \path (i) edge (b);
            \path (b) edge (j);
            \path (c) edge (k);
            \path (k) edge (d);
            \path (e) edge (l);

            \begin{pgfonlayer}{background}
                \draw[gray!15, fill=gray!15, line width=12mm, line cap=round, line join=round] (a.center) -- (b.center) -- (j.center) -- (i.center) -- cycle;
                \draw[gray!15, fill=gray!15, line width=12mm, line cap=round, line join=round] (c.center) -- (d.center) -- (k.center) -- cycle;
                \draw[gray!15, fill=gray!15, line width=12mm, line cap=round, line join=round] (e.center) -- (l.center) -- cycle;
            \end{pgfonlayer}

        \end{tikzpicture}
        \caption{Symbol graph for the example}
        \label{fig:nested_expressions:example_symbol_graph}
    \end{figure}

    In the symbol graph, if two symbols are connected, then they are effectively the same index.
    Therefore, it makes sense assigning a symbol $s_C \in S \setminus V$ to each of the graphs components $C$.
    Then we can define $\nu$ as follows:
    $$\nu(s) := \begin{cases}
            s_C & \text{if } s \in C \\
            s   & \text{else}
        \end{cases}.$$
    In our example, the components are $\smallset{a,b,i,j}$, $\smallset{c,d,k}$, and $\smallset{e,l}$.
    Therefore we could use
    $$\nu(s) := \begin{cases}
            x & \text{if } s \in \smallset{a,b,i,j} \\
            y & \text{if } s \in \smallset{c,d,k}   \\
            z & \text{if } s \in \smallset{e,l}     \\
            s & \text{else}
        \end{cases}.$$

    The symbol map $\nu$ can be extended, such that it maps entire index strings instead of just symbols, by setting $\nu(\bm{s_i}) \in S^{n_i}, \nu(\bm{s_i})_j := \nu(s_{ij})$.
    Then we can write the substituted index strings by setting $\bm{\hat{s}_i} := \nu(\bm{s_i})$ for $i \in [m + n]$ and $\bm{\hat{s}_t} = \nu(\bm{s_t})$.
    With these index strings, the compressed Einsum expression is the following:
    $$V = (\bm{\hat{s}_1}, \dots, \bm{\hat{s}_{m + n}} \rightarrow \bm{\hat{s}_v}, T^{(1)},\dots,T^{(m + n)})$$
    which helps us to compress the example:
    \begin{gather*}
        (a,b,c,d,e,abbde \rightarrow bc, v^{(1)}, v^{(2)}, v^{(3)}, v^{(4)}, v^{(5)}, (
        i,j,k,l \rightarrow iijkkl, v^{(6)}, v^{(7)}, v^{(8)}, v^{(9)}
        ))\\
        =(x,x,y,y,z,x,x,y,z \rightarrow xy, v^{(1)}, v^{(2)}, v^{(3)}, v^{(4)}, v^{(5)}, v^{(6)}, v^{(7)}, v^{(8)}, v^{(9)}).
    \end{gather*}
\end{theorem}

\bigskip
For the proof of this theorem, we first need three lemmata, which essentially boil down to one intuitive thought:
The effective equality of two symbols can be expressed by multiplication with the unity matrix $\1_d$:
$$\left(\1_d\right)_{ij} := \begin{cases}
        \1 & \text{if $i = j$} \\
        \0 & \text{else}
    \end{cases},$$
for $i,j \in [d]$, where $\0$ and $\1$ indicate the neutral element of addition and multiplication in the given semiring respectively.

\begin{lemma}
    \label{lemma:nested_einsum:1}
    For $i \in [n]$, let $T^{(i)}$ be an $n_i$-th order tensor with index string $\bm{s_i} \in S^{n_i}$.
    Let $\bm{s_t}$ be the index string for $T$ with
    $$T = (\bm{s_1}, \dots, \bm{s_n} \rightarrow \bm{s_t}, T^{(1)}, \dots, T^{(n)}).$$
    Let $F$ and $B$ be the free and bound symbols of this expression.
    Let $k \in [n]$ and $j \in [n_k]$, then we can replace the $j$-th symbol of the $k$-th index string with a new symbol $s_{\text{new}} \in S \setminus (F \cup B)$ by adding the unity matrix $\1_{d_{kj}}$ as an input tensor in the following way:

    Let $\bm{s'_k}$ be a new index string such that
    $$s'_{ki} := \begin{cases}
            s_{\text{new}} & \text{if $i = j$}, \\
            s_{ki}         & \text{else}
        \end{cases}$$
    for $i \in [n_k]$.
    Let $\bm{s_\1} = (s_{kj}, s_{\text{new}})$.
    Then
    $$T = (\bm{s_1}, \dots, \bm{s'_k}, \dots, \bm{s_n}, \bm{s_\1} \rightarrow \bm{s_t}, T^{(1)}, \dots, T^{(n)}, \1_{d_{kj}}).$$
\end{lemma}

\begin{proof}
    \small
    Let $\mathcal{F}$ and $\mathcal{B}$ be the induces multi-index spaces for the free and bound symbols of the Einsum expression.
    Then
    \begin{align*}
        T                                                        & = (\bm{s_1}, \dots, \bm{s_n} \rightarrow \bm{s_t}, T^{(1)}, \dots, T^{(n)})                                                                                                                                                                        \\
        \iff \forall \bm{f} \in \mathcal{F}: T_{\bm{f}:\bm{s_t}} & = \bigoplus\limits_{\bm{b} \in \mathcal{B}} \bigodot\limits_{i \in [n]} T^{(i)}_{(\bm{f}, \bm{b}):\bm{s_i}}                                                                                                                                        \\
                                                                 & = \bigoplus\limits_{\bm{b} \in \mathcal{B} \times [d_{kj}]} \bigodot\limits_{1 \leq i < k} T^{(i)}_{(\bm{f}, \bm{b}):\bm{s_i}} \odot T^{(k)}_{(\bm{f}, \bm{b}):\bm{s'_k}} \odot \bigodot\limits_{k < i \leq n} T^{(i)}_{(\bm{f}, \bm{b}):\bm{s_i}} \\
                                                                 & \phantom{{}=\bigoplus\limits_{\bm{b} \in \mathcal{B} \times [d_{kj}]}} \odot \begin{cases}
            \1 & \text{if $(\bm{f}, \bm{b}): s_{kj} = (\bm{f}, \bm{b}): s_{\text{new}}$}, \\
            \0 & \text{else}
        \end{cases}                                                                                                                                             \\
                                                                 & = \bigoplus\limits_{\bm{b} \in \mathcal{B} \times [d_{kj}]} \bigodot\limits_{1 \leq i < k} T^{(i)}_{(\bm{f}, \bm{b}):\bm{s_i}} \odot T^{(k)}_{(\bm{f}, \bm{b}):\bm{s'_k}} \odot \bigodot\limits_{k < i \leq n} T^{(i)}_{(\bm{f}, \bm{b}):\bm{s_i}} \\
                                                                 & \phantom{{}=\bigoplus\limits_{\bm{b} \in \mathcal{B} \times [d_{kj}]}} \odot \left(\1_{d_{kj}}\right)_{(\bm{f}, \bm{b}):\bm{s_\1}}                                                                                                                 \\
        \iff T                                                   & = (\bm{s_1}, \dots, \bm{s'_k}, \dots, \bm{s_n}, \bm{s_\1} \rightarrow \bm{s_t}, T^{(1)}, \dots, T^{(n)}, \1_{d_{kj}})
    \end{align*}
    where the third equality holds because in the summation over $\mathcal{B} \times [d_{kj}]$, exactly those summands get selected by the condition, which are also valid summands in the previous summation over $\mathcal{B}$.
    All other summands are disregarded because they are multiplied by $\0$, which is the additive neutral element in the semiring and \textit{annihilates} every element, which means $a \odot \0 = \0$ for every $a \in M$.
\end{proof}
\bigskip

This lemma intuitively means that we can replace any symbol in an index string of an input tensor of our choice with a new symbol by introducing the unity matrix with an appropriate index string as a factor.
Now the same holds for the index string of the output tensor $\bm{s_t}$, which will be the content of the next lemma.

\begin{lemma}
    \label{lemma:nested_einsum:2}
    For $i \in [n]$, let $T^{(i)}$ be an $n_i$-th order tensor with index string $\bm{s_i} \in S^{n_i}$.
    Let $\bm{s_t}$ be the index string for $T$ with
    $$T = (\bm{s_1}, \dots, \bm{s_n} \rightarrow \bm{s_t}, T^{(1)}, \dots, T^{(n)}).$$
    Let $F$ and $B$ be the free and bound symbols of this expression.
    Let $n_t := \abs{\bm{s_t}}$, $j \in [n_t]$, and $d_{tj} := d_{s_{tj}}$, then we can replace the $j$-th symbol of the output string with a new symbol $s_{\text{new}} \in S \setminus (F \cup B)$ by adding the unity matrix $\1_{d_{tj}}$ as an input tensor in the following way:

    Let $\bm{s'_t}$ be a new index string such that
    $$s'_{ti} := \begin{cases}
            s_{\text{new}} & \text{if $i = j$}, \\
            s_{ti}         & \text{else}
        \end{cases}$$
    for $i \in [n_t]$.
    Let $\bm{s_\1} = (s_{tj}, s_{\text{new}})$.
    Then
    $$T = (\bm{s_1}, \dots, \bm{s_n}, \bm{s_\1} \rightarrow \bm{s'_t}, T^{(1)}, \dots, T^{(n)}, \1_{d_{kj}}).$$
\end{lemma}

\begin{proof}
    \small
    Let $\mathcal{F}$ and $\mathcal{B}$ be the induces multi-index spaces for the free and bound symbols of the Einsum expression.
    If $s_{tj}$ occurs in $\bm{s_t}$ even after replacing it with $s_{\text{new}}$, then
    \begin{align*}
        T                                                                         & = (\bm{s_1}, \dots, \bm{s_n} \rightarrow \bm{s_t}, T^{(1)}, \dots, T^{(n)})                                                                                             \\
        \iff \forall \bm{f} \in \mathcal{F}: T_{\bm{f}:\bm{s_t}}                  & = \bigoplus\limits_{\bm{b} \in \mathcal{B}} \bigodot\limits_{i \in [n]} T^{(i)}_{(\bm{f}, \bm{b}):\bm{s_i}}                                                             \\
        \iff \forall \bm{f} \in \mathcal{F} \times [d_{tj}]: T_{\bm{f}:\bm{s'_t}} & = \bigoplus\limits_{\bm{b} \in \mathcal{B}} \bigodot\limits_{i \in [n]} T^{(i)}_{(\bm{f}, \bm{b}):\bm{s_i}} \odot \begin{cases}
            \1 & \text{if $\bm{f}: s_{tj} = \bm{f}: s_{\text{new}}$}, \\
            \0 & \text{else}
        \end{cases}                             \\
                                                                                  & = \bigoplus\limits_{\bm{b} \in \mathcal{B}} \bigodot\limits_{i \in [n]} T^{(i)}_{(\bm{f}, \bm{b}):\bm{s_i}} \odot \left(\1_{d_{tj}}\right)_{(\bm{f}, \bm{b}):\bm{s_\1}} \\
        \iff T                                                                    & = (\bm{s_1}, \dots, \bm{s_n}, \bm{s_\1} \rightarrow \bm{s'_t}, T^{(1)}, \dots, T^{(n)}, \1_{d_{tj}})
    \end{align*}
    where the third equality holds because exactly those indices get selected by the condition, where $T$ was originally defined.
    If $s_{tj}$ no longer occurs in $\bm{s_t}$ after replacing it with $s_{\text{new}}$, then $s_{tj}$ turns into a bound symbol.
    Therefore, we have to define $\mathcal{F}' = \prod_{s \in \bm{s'_t}} [d_s]$.
    Then
    \begin{align*}
        T                                                          & = (\bm{s_1}, \dots, \bm{s_n} \rightarrow \bm{s_t}, T^{(1)}, \dots, T^{(n)})                                                                                                             \\
        \iff \forall \bm{f} \in \mathcal{F}: T_{\bm{f}:\bm{s_t}}   & = \bigoplus\limits_{\bm{b} \in \mathcal{B}} \bigodot\limits_{i \in [n]} T^{(i)}_{(\bm{f}, \bm{b}):\bm{s_i}}                                                                             \\
        \iff \forall \bm{f} \in \mathcal{F}': T_{\bm{f}:\bm{s'_t}} & = \bigoplus\limits_{\bm{b} \in \mathcal{B} \times [d_{tj}]} \bigodot\limits_{i \in [n]} T^{(i)}_{(\bm{f}, \bm{b}):\bm{s_i}} \odot \begin{cases}
            \1 & \text{if $(\bm{f}, \bm{b}): s_{tj} = (\bm{f}, \bm{b}): s_{\text{new}}$}, \\
            \0 & \text{else}
        \end{cases}                             \\
                                                                   & = \bigoplus\limits_{\bm{b} \in \mathcal{B} \times [d_{tj}]} \bigodot\limits_{i \in [n]} T^{(i)}_{(\bm{f}, \bm{b}):\bm{s_i}} \odot \left(\1_{d_{tj}}\right)_{(\bm{f}, \bm{b}):\bm{s_\1}} \\
        \iff T                                                     & = (\bm{s_1}, \dots, \bm{s_n}, \bm{s_\1} \rightarrow \bm{s'_t}, T^{(1)}, \dots, T^{(n)}, \1_{d_{tj}})
    \end{align*}
    where the third equality holds because exactly those summands get selected by the condition, where $(\bm{f}, \bm{b}):\bm{s_t}$ could also get used as an index for $T$.
    All others are annihilated.
\end{proof}
\bigskip

Now with these two lemmata, we can replace any symbol in any index string, regardless if it is and input string or the output string by introducing the unity matrix with an appropriate index string as a factor.
In the following lemma, we will show that any unity matrix factors can be removed again, by renaming certain symbols in all other index strings in the Einsum expression.

\begin{lemma}
    For $i \in [n]$, let $T^{(i)}$ be an $n_i$-th order tensor with index string $\bm{s_i} \in S^{n_i}$,
    where $T^{(n)} = \1_m$ for some $m \in \N$.
    Let $\bm{s_t}$ be the index string for $T$ with
    $$T = (\bm{s_1}, \dots, \bm{s_n} \rightarrow \bm{s_t}, T^{(1)}, \dots, T^{(n)}).$$

    Let $F$ and $B$ be the free and bound symbols of this expression.
    Then we can introduce a symbol map $\mu: S \rightarrow S$, which maps both symbols in $\bm{s_n}$ to the same symbol $s_{\text{new}} \in S \setminus (F \cup B)$:
    $$\mu(s) := \begin{cases}
            s_{\text{new}} & \text{if $s \in \smallset{s_{n1}, s_{n2}}$} \\
            s              & \text{else}
        \end{cases}$$
    The symbol map $\mu$ can be extended, such that it maps entire index strings instead of just symbols, by setting $\mu(\bm{s_i}) \in S^{n_i}, \mu(\bm{s_i})_j := \mu(s_{ij})$.
    Then we can write the substituted index strings by setting $\bm{s'_i} := \mu(\bm{s_i})$ for $i \in [n]$ and $\bm{s'_t} = \mu(\bm{s_t})$.
    With these index strings, the following holds:
    $$T = (\bm{s'_1}, \dots, \bm{s'_{n - 1}} \rightarrow \bm{s'_t}, T^{(1)}, \dots, T^{(n - 1)}).$$
\end{lemma}

\begin{proof}
    \small
    bla bla
\end{proof}