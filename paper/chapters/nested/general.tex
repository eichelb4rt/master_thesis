\section{General Nested Expressions}

The following is an example of an expression, which we cannot compress with the previous theorem:
$$(a,b,c,d,e,abbcde \rightarrow bc, v^{(1)}, v^{(2)}, v^{(3)}, v^{(4)}, v^{(5)}, (
    i,j,k,l \rightarrow iijkkl, v^{(6)}, v^{(7)}, v^{(8)}, v^{(9)}
    ))$$
for $v^{(i)} \in \R^{d_{vi}}$ with $i \in [9]$, where $d_{vi} \in \N$ are appropriate dimensions.
This is because the output string $\bm{s_u} = iijkkl$ and the input string $\bm{\hat{s}_u} = abbde$ are not the same.
In the following, we will explore how to compress such expressions.
Note that, for the theorem, we use disjoint sets of symbols for the inner and outer expression.
This helps in the proof, and is not a real constraint in practice,
because we can just rename the symbols in different scopes.
For example, we could write the matrix-matrix-vector multiplication of $A \in \R^{m \times r}, B \in \R^{r \times n}, v \in \R^n$ as
$$A \cdot (B \cdot v) = (ij, j \rightarrow i, A, (ab, b \rightarrow a, B, v))$$
or as
$$A \cdot (B \cdot v) = (ij, j \rightarrow i, A, (ij, j \rightarrow i, B, v)),$$
because the scope of each symbol does not reach into nested expressions,
and therefore the $i$ and $j$ used in the outer expression are treated as different symbols than the $i$ and $j$ used in the inner expression.

\begin{theorem}
    \label{thm:nested_einsum:general}
    For $i \in [m + n]$, let $T^{(i)}$ be an $n_i$-th order tensor with index string $\bm{s_i} \in S^{n_i}$.
    Let $\bm{s_u}$ be an index string for the $n_u$-th order tensor $U$, which is defined as follows:
    $$U = (\bm{s_{m + 1}},\dots,\bm{s_{m + n}} \rightarrow \bm{s_u}, T^{(m + 1)},\dots,T^{(m + n)})$$
    Also let $\bm{\hat{s}_u}$ be alternative index strings for $U$.

    Let $s_v$ be an index string and
    $$V = (\bm{s_1},\dots,\bm{s_m}, \bm{\hat{s}_u} \rightarrow \bm{s_v}, T^{(1)},\dots,T^{(m)}, U)$$
    where the first and second Einsum expression share no symbols.
    Then these nested Einsum expressions can also be compressed into a single Einsum expression.

    Let us clarify that the index string $\bm{s_u}$ corresponds to the output string of the inner expression,
    and the index string $\bm{\hat{s}_u}$ corresponds to the input string that is used for the input tensor $U$ in the outer expression.
    In our example, these are $\bm{s_u}$ and $\bm{\hat{s}_u}$:
    $$(a,b,c,d,e,\overbrace{abbcde}^{\bm{\hat{s}_u}} \rightarrow bc, v^{(1)}, v^{(2)}, v^{(3)}, v^{(4)}, v^{(5)}, (
        i,j,k,l \rightarrow \overbrace{iijkkl}^{\bm{s_u}}, v^{(6)}, v^{(7)}, v^{(8)}, v^{(9)}
        )).$$

    In contrast to \cref{thm:nested_einsum:simple}, we cannot just replace the input index string $\bm{\hat{s}_u}$ by all the input index strings in the inner Einsum expression $\bm{s_{m + 1}},\dots,\bm{s_{m + n}}$.
    Instead, we first need to apply a symbol map $\nu: S \rightarrow S$ to each of the index strings.
    This symbol map holds information about which symbols are effectively the used for the same index.

    For the definition of the map $\nu$, we first construct an undirected graph $G = (V, E)$ that we call \textit{symbol graph}.
    % $V = \sigma(\bm{s_v}) \cup \sigma(\bm{s_u}) \cup \sigma(\bm{\hat{s}_u}) \cup \bigcup_{i \in [m + n]} \sigma(\bm{s_i})$
    In the symbol graph, the nodes consist of all symbols from $\bm{s_u}$ and $\bm{\hat{s}_u}$.
    The edges are $E = \smallset{\smallset{s_{uj}, \hat{s}_{uj}} \mid j \in [n_u]}$,
    which connects all symbols from $\bm{s_u}$ and $\bm{\hat{s}_u}$ that share an index.
    The symbol graph for our example is illustrated in \cref{fig:nested_expressions:example_symbol_graph}.

    \begin{figure}[h]
        \centering
        \begin{tikzpicture}[node distance = 2cm, semithick]

            \node[state] (a)					{$a$};
            \node[state] (b) [right of=a]		{$b$};
            \node[state] (c) [right of=b] 		{$c$};
            \node[state] (d) [right of=c] 		{$d$};
            \node[state] (e) [right of=d] 		{$e$};
            \node (middle) at ($(a)!0.5!(b)$) {};
            \node[state] (i) [below of=middle] 		{$i$};
            \node[state] (j) [right of=i] 		{$j$};
            \node[state] (k) [right of=j] 		{$k$};
            \node[state] (l) [right of=k] 		{$l$};

            \node (x_center) at ($(a)!0.5!(j)$) {};
            \node (y_center) at ($(c)!0.5!(d|-k)$) {};
            \node (z_center) at ($(e)!0.5!(l)$) {};
            \node (x) [below of=x_center] {$x$};
            \node (y) [below of=y_center] {$y$};
            \node (z) [below of=z_center] {$z$};

            \path (a) edge (i);
            \path (i) edge (b);
            \path (b) edge (j);
            \path (c) edge (k);
            \path (k) edge (d);
            \path (e) edge (l);

            \begin{pgfonlayer}{background}
                \draw[gray!15, fill=gray!15, line width=12mm, line cap=round, line join=round] (a.center) -- (b.center) -- (j.center) -- (i.center) -- cycle;
                \draw[gray!15, fill=gray!15, line width=12mm, line cap=round, line join=round] (c.center) -- (d.center) -- (k.center) -- cycle;
                \draw[gray!15, fill=gray!15, line width=12mm, line cap=round, line join=round] (e.center) -- (l.center) -- cycle;
            \end{pgfonlayer}

        \end{tikzpicture}
        \caption{Symbol graph for the example}
        \label{fig:nested_expressions:example_symbol_graph}
    \end{figure}

    In the symbol graph, if two symbols are connected, then they will both be mapped to the same symbol.
    Therefore, it makes sense assigning a symbol $s_C \in S \setminus V$ to each of the graphs components $C$.
    Then we can define $\nu$ as follows:
    $$\nu(s) := \begin{cases}
            s_C & \text{if $\exists C: \text{$C$ is a component of $G$ and $s \in C$}$}, \\
            s   & \text{else}.
        \end{cases}$$
    In our example, the components are $\smallset{a,b,i,j}$, $\smallset{c,d,k}$, and $\smallset{e,l}$.
    Therefore we could use
    $$\nu(s) := \begin{cases}
            x & \text{if $s \in \smallset{a,b,i,j}$}, \\
            y & \text{if $s \in \smallset{c,d,k}$},   \\
            z & \text{if $s \in \smallset{e,l}$},     \\
            s & \text{else}.
        \end{cases}$$

    The symbol map $\nu$ can be extended, such that it maps entire index strings instead of just symbols, by setting $\nu(\bm{s_i}) \in S^{n_i}, \nu(\bm{s_i})_j := \nu(s_{ij})$.
    Then we can write the substituted index strings by setting $\bm{\hat{s}_i} := \nu(\bm{s_i})$ for $i \in [m + n]$ and $\bm{\hat{s}_t} = \nu(\bm{s_t})$.
    With these index strings, the compressed Einsum expression is the following:
    $$V = (\bm{\hat{s}_1}, \dots, \bm{\hat{s}_{m + n}} \rightarrow \bm{\hat{s}_v}, T^{(1)},\dots,T^{(m + n)})$$
    which helps us to compress the example:
    \begin{gather*}
        (a,b,c,d,e,abbde \rightarrow bc, v^{(1)}, v^{(2)}, v^{(3)}, v^{(4)}, v^{(5)}, (
        i,j,k,l \rightarrow iijkkl, v^{(6)}, v^{(7)}, v^{(8)}, v^{(9)}
        ))\\
        =(x,x,y,y,z,x,x,y,z \rightarrow xy, v^{(1)}, v^{(2)}, v^{(3)}, v^{(4)}, v^{(5)}, v^{(6)}, v^{(7)}, v^{(8)}, v^{(9)}).
    \end{gather*}
\end{theorem}

\bigskip
For the proof of this theorem, we first need three lemmata, which essentially boil down to one intuitive thought:
The effective equality of two symbols can be expressed by multiplication with the unity matrix $\1_d$:
$$\left(\1_d\right)_{ij} := \begin{cases}
        \1 & \text{if $i = j$}, \\
        \0 & \text{else}
    \end{cases}$$
for $i,j \in [d]$, where $\0$ and $\1$ indicate the neutral element of addition and multiplication in the given semiring respectively.

\begin{lemma}
    \label{lemma:nested_einsum:1}
    For $i \in [n]$, let $T^{(i)}$ be an $n_i$-th order tensor with index string $\bm{s_i} \in S^{n_i}$.
    Let $\bm{s_t}$ be the index string for $T$ with
    $$T = (\bm{s_1}, \dots, \bm{s_n} \rightarrow \bm{s_t}, T^{(1)}, \dots, T^{(n)}).$$
    Let $F$ and $B$ be the free and bound symbols of this expression.
    Let $k \in [n]$ and $j \in [n_k]$, then we can replace the $j$-th symbol of the $k$-th index string with a new symbol $s_{\text{new}} \in S \setminus (F \cup B)$ by adding the unity matrix $\1_{d_{kj}}$ as an input tensor in the following way:

    Let $\bm{s'_k}$ be a new index string such that
    $$s'_{ki} := \begin{cases}
            s_{\text{new}} & \text{if $i = j$}, \\
            s_{ki}         & \text{else}
        \end{cases}$$
    for $i \in [n_k]$.
    Let $\bm{s_\1} = (s_{kj}, s_{\text{new}})$.
    Then
    $$T = (\bm{s_1}, \dots, \bm{s'_k}, \dots, \bm{s_n}, \bm{s_\1} \rightarrow \bm{s_t}, T^{(1)}, \dots, T^{(n)}, \1_{d_{kj}}).$$
\end{lemma}

\begin{proof}
    \small
    Let $\mathcal{F}$ and $\mathcal{B}$ be the induces multi-index spaces for the free and bound symbols of the Einsum expression.
    Then
    \begin{align*}
        T                                                        & = (\bm{s_1}, \dots, \bm{s_n} \rightarrow \bm{s_t}, T^{(1)}, \dots, T^{(n)})                                                                                                                                                                        \\
        \iff \forall \bm{f} \in \mathcal{F}: T_{\bm{f}:\bm{s_t}} & = \bigoplus\limits_{\bm{b} \in \mathcal{B}} \bigodot\limits_{i \in [n]} T^{(i)}_{(\bm{f}, \bm{b}):\bm{s_i}}                                                                                                                                        \\
                                                                 & = \bigoplus\limits_{\bm{b} \in \mathcal{B} \times [d_{kj}]} \bigodot\limits_{1 \leq i < k} T^{(i)}_{(\bm{f}, \bm{b}):\bm{s_i}} \odot T^{(k)}_{(\bm{f}, \bm{b}):\bm{s'_k}} \odot \bigodot\limits_{k < i \leq n} T^{(i)}_{(\bm{f}, \bm{b}):\bm{s_i}} \\
                                                                 & \phantom{{}=\bigoplus\limits_{\bm{b} \in \mathcal{B} \times [d_{kj}]}} \odot \begin{cases}
            \1 & \text{if $(\bm{f}, \bm{b}): s_{kj} = (\bm{f}, \bm{b}): s_{\text{new}}$}, \\
            \0 & \text{else}
        \end{cases}                                                                                                                                             \\
                                                                 & = \bigoplus\limits_{\bm{b} \in \mathcal{B} \times [d_{kj}]} \bigodot\limits_{1 \leq i < k} T^{(i)}_{(\bm{f}, \bm{b}):\bm{s_i}} \odot T^{(k)}_{(\bm{f}, \bm{b}):\bm{s'_k}} \odot \bigodot\limits_{k < i \leq n} T^{(i)}_{(\bm{f}, \bm{b}):\bm{s_i}} \\
                                                                 & \phantom{{}=\bigoplus\limits_{\bm{b} \in \mathcal{B} \times [d_{kj}]}} \odot \left(\1_{d_{kj}}\right)_{(\bm{f}, \bm{b}):\bm{s_\1}}                                                                                                                 \\
        \iff T                                                   & = (\bm{s_1}, \dots, \bm{s'_k}, \dots, \bm{s_n}, \bm{s_\1} \rightarrow \bm{s_t}, T^{(1)}, \dots, T^{(n)}, \1_{d_{kj}})
    \end{align*}
    where the third equality holds because in the summation over $\mathcal{B} \times [d_{kj}]$, exactly those summands get selected by the condition, which are also valid summands in the previous summation over $\mathcal{B}$.
    All other summands are disregarded because they are multiplied by $\0$, which is the additive neutral element in the semiring and \textit{annihilates} every element, which means $a \odot \0 = \0$ for every $a \in M$.
\end{proof}
\bigskip

This lemma intuitively means that we can replace any symbol in an index string of an input tensor of our choice with a new symbol by introducing the unity matrix with an appropriate index string as a factor.
Now the same holds for the index string of the output tensor $\bm{s_t}$, which will be the content of the next lemma.

\begin{lemma}
    \label{lemma:nested_einsum:2}
    For $i \in [n]$, let $T^{(i)}$ be an $n_i$-th order tensor with index string $\bm{s_i} \in S^{n_i}$.
    Let $\bm{s_t}$ be the index string for $T$ with
    $$T = (\bm{s_1}, \dots, \bm{s_n} \rightarrow \bm{s_t}, T^{(1)}, \dots, T^{(n)}).$$
    Let $F$ and $B$ be the free and bound symbols of this expression.
    Let $n_t := \abs{\bm{s_t}}$, $j \in [n_t]$, and $d_{tj} := d_{s_{tj}}$, then we can replace the $j$-th symbol of the output string with a new symbol $s_{\text{new}} \in S \setminus (F \cup B)$ by adding the unity matrix $\1_{d_{tj}}$ as an input tensor in the following way:

    Let $\bm{s'_t}$ be a new index string such that
    $$s'_{ti} := \begin{cases}
            s_{\text{new}} & \text{if $i = j$}, \\
            s_{ti}         & \text{else}
        \end{cases}$$
    for $i \in [n_t]$.
    Let $\bm{s_\1} = (s_{tj}, s_{\text{new}})$.
    Then
    $$T = (\bm{s_1}, \dots, \bm{s_n}, \bm{s_\1} \rightarrow \bm{s'_t}, T^{(1)}, \dots, T^{(n)}, \1_{d_{kj}}).$$
\end{lemma}

\begin{proof}
    \small
    Let $\mathcal{F}$ and $\mathcal{B}$ be the induces multi-index spaces for the free and bound symbols of the Einsum expression.
    % TODO: Fallunterscheidung s_{tj} wird bound symbol / wird kein bound symbol
    If $s_{tj}$ occurs in $\bm{s_t}$ even after replacing it with $s_{\text{new}}$, then
    \begin{align*}
        T                                                                         & = (\bm{s_1}, \dots, \bm{s_n} \rightarrow \bm{s_t}, T^{(1)}, \dots, T^{(n)})                                                                                             \\
        \iff \forall \bm{f} \in \mathcal{F}: T_{\bm{f}:\bm{s_t}}                  & = \bigoplus\limits_{\bm{b} \in \mathcal{B}} \bigodot\limits_{i \in [n]} T^{(i)}_{(\bm{f}, \bm{b}):\bm{s_i}}                                                             \\
        \iff \forall \bm{f} \in \mathcal{F} \times [d_{tj}]: T_{\bm{f}:\bm{s'_t}} & = \bigoplus\limits_{\bm{b} \in \mathcal{B}} \bigodot\limits_{i \in [n]} T^{(i)}_{(\bm{f}, \bm{b}):\bm{s_i}} \odot \begin{cases}
            \1 & \text{if $\bm{f}: s_{tj} = \bm{f}: s_{\text{new}}$}, \\
            \0 & \text{else}
        \end{cases}                             \\
                                                                                  & = \bigoplus\limits_{\bm{b} \in \mathcal{B}} \bigodot\limits_{i \in [n]} T^{(i)}_{(\bm{f}, \bm{b}):\bm{s_i}} \odot \left(\1_{d_{tj}}\right)_{(\bm{f}, \bm{b}):\bm{s_\1}} \\
        \iff T                                                                    & = (\bm{s_1}, \dots, \bm{s_n}, \bm{s_\1} \rightarrow \bm{s'_t}, T^{(1)}, \dots, T^{(n)}, \1_{d_{tj}})
    \end{align*}
    where the third equality holds because exactly those indices get selected by the condition, where $T$ was originally defined.
    If $s_{tj}$ no longer occurs in $\bm{s_t}$ after replacing it with $s_{\text{new}}$, then $s_{tj}$ turns into a bound symbol.
    Therefore we have to define $\mathcal{F}' = \prod_{s \in \bm{s'_t}} [d_s]$.
    Then
    \begin{align*}
        T                                                          & = (\bm{s_1}, \dots, \bm{s_n} \rightarrow \bm{s_t}, T^{(1)}, \dots, T^{(n)})                                                                                                             \\
        \iff \forall \bm{f} \in \mathcal{F}: T_{\bm{f}:\bm{s_t}}   & = \bigoplus\limits_{\bm{b} \in \mathcal{B}} \bigodot\limits_{i \in [n]} T^{(i)}_{(\bm{f}, \bm{b}):\bm{s_i}}                                                                             \\
        \iff \forall \bm{f} \in \mathcal{F}': T_{\bm{f}:\bm{s'_t}} & = \bigoplus\limits_{\bm{b} \in \mathcal{B} \times [d_{tj}]} \bigodot\limits_{i \in [n]} T^{(i)}_{(\bm{f}, \bm{b}):\bm{s_i}} \odot \begin{cases}
            \1 & \text{if $(\bm{f}, \bm{b}): s_{tj} = (\bm{f}, \bm{b}): s_{\text{new}}$}, \\
            \0 & \text{else}
        \end{cases}                             \\
                                                                   & = \bigoplus\limits_{\bm{b} \in \mathcal{B} \times [d_{tj}]} \bigodot\limits_{i \in [n]} T^{(i)}_{(\bm{f}, \bm{b}):\bm{s_i}} \odot \left(\1_{d_{tj}}\right)_{(\bm{f}, \bm{b}):\bm{s_\1}} \\
        \iff T                                                     & = (\bm{s_1}, \dots, \bm{s_n}, \bm{s_\1} \rightarrow \bm{s'_t}, T^{(1)}, \dots, T^{(n)}, \1_{d_{tj}})
    \end{align*}
    where the third equality holds because exactly those summands get selected by the condition, where $(\bm{f}, \bm{b}):\bm{s_t}$ could also get used as an index for $T$.
    All others are annihilated.
\end{proof}
\bigskip

Now with these two lemmata, we can replace any symbol in any index string, regardless whether it is an input string or the output string, by introducing the unity matrix with an appropriate index string as a factor.
In the following lemma, we will show that any unity matrix factors can be removed again, by renaming certain symbols in all other index strings in the Einsum expression.

\begin{lemma}
    \label{lemma:nested_einsum:3}
    For $i \in [n]$, let $T^{(i)}$ be an $n_i$-th order tensor with index string $\bm{s_i} \in S^{n_i}$,
    where $T^{(n)} = \1_m$ for some $m \in \N$.
    Let $\bm{s_t}$ be the index string for $T$ with
    $$T = (\bm{s_1}, \dots, \bm{s_n} \rightarrow \bm{s_t}, T^{(1)}, \dots, T^{(n)}).$$

    Let $F$ and $B$ be the free and bound symbols of this expression.
    Then we can introduce a symbol map $\mu: S \rightarrow S$, which maps both symbols in $\bm{s_n} = s_{n1}s_{n2}$ to the same symbol $s_{\text{new}} \in S \setminus (F \cup B)$:
    $$\mu(s) := \begin{cases}
            s_{\text{new}} & \text{if $s \in \smallset{s_{n1}, s_{n2}}$}, \\
            s              & \text{else}.
        \end{cases}$$
    The symbol map $\mu$ can be extended, such that it maps entire index strings instead of just symbols, by setting $\mu(\bm{s_i}) \in S^{n_i}, \mu(\bm{s_i})_j := \mu(s_{ij})$.
    Then we can write the substituted index strings by setting $\bm{s'_i} := \mu(\bm{s_i})$ for $i \in [n]$ and $\bm{s'_t} = \mu(\bm{s_t})$.
    With these index strings, the following holds:
    $$T = (\bm{s'_1}, \dots, \bm{s'_{n - 1}} \rightarrow \bm{s'_t}, T^{(1)}, \dots, T^{(n - 1)}).$$
\end{lemma}

\begin{proof}
    \small
    For this proof, we provide the following example of an Einsum expression on which we demonstrate the given arguments for better understanding:
    $$(ij, kl, mn, ij, kl, mn \rightarrow imn, A, B, C, \1_a, \1_b, \1_c)$$
    for $A \in \R^{a \times a}, B \in \R^{b \times b}, C \in \R^{c \times c}$ and some $a,b,c \in \N$.

    We need to consider three cases for the symbols used in the index string $\bm{s_n} = (s_{n1}, s_{n2})$:
    \begin{itemize}
        \item $s_{n1}$ and $s_{n2}$ are both free symbols,
        \item $s_{n1}$ and $s_{n2}$ are both bound symbols,
        \item one symbol of $s_{n1}$ and $s_{n2}$ is a free symbol, the other is a bound symbol.
    \end{itemize}
    Every one of these cases leads to the same result, but in a slightly different way.

    First let us consider the case where both symbols are free.
    In this case, both symbols can be replaced by a single symbol,
    because $T$ is $\0$ for all entries with a multi-index,
    where the indices projected by the symbols are not equal.

    In our example, this is equivalent to the following:
    \begin{align*}
        \forall i,m,n: T_{imn}    & = \bigoplus\limits_{j,k,l} A_{ij} B_{kl} C_{mn} \left(\1_a\right)_{ij} \left(\1_b\right)_{kl} \left(\1_c\right)_{mn} \\
                                  & = \begin{cases}
            \bigoplus\limits_{j,k,l} A_{ij} B_{kl} C_{mn} \left(\1_a\right)_{ij} \left(\1_b\right)_{kl} & \text{if $m = n$}, \\
            \0                                                                                          & \text{else}
        \end{cases}                                                                                          \\
        \iff \forall i,z: T_{izz} & = \bigoplus\limits_{j,k,l} A_{ij} B_{kl} C_{zz} \left(\1_a\right)_{ij} \left(\1_b\right)_{kl}.
    \end{align*}

    Next let us consider the case where both symbols are bound.
    In this case, those summands are multiplied with $\0$,
    which have a multi-index where the projected indices are not equal.
    Therefore, those summands are annihilated and left out from the summation.
    This means that both symbols can be replaced by a single symbol.

    In our example, this is equivalent to the following:
    \begin{align*}
        \forall i,z: T_{izz} & = \bigoplus\limits_{j,k,l} A_{ij} B_{kl} C_{zz} \left(\1_a\right)_{ij} \left(\1_b\right)_{kl}          \\
                             & = \bigoplus\limits_{j,k,l} A_{ij} B_{kl} C_{zz} \left(\1_a\right)_{ij} \odot \begin{cases}
            \1 & \text{if $k = l$}, \\
            \0 & \text{else}
        \end{cases} \\
                             & = \bigoplus\limits_{j,y} A_{ij} B_{yy} C_{zz} \left(\1_a\right)_{ij}.
    \end{align*}

    Next let us consider the case where one symbol is free and one symbol is bound.
    W.l.o.g. we consider the case where $s_{n1}$ is free and $s_{n2}$ is bound.
    In this case, those summands are multiplied with $\0$,
    which have a multi-index where the index projected by the bound symbol $s_{n2}$ is not the same as the index projected by the free symbol $s_{n1}$.
    Therefore those summands are annihilated and left out from the summation, and the symbol $s_{n2}$ can be replaced by the symbol $s_{n1}$.
    Additionally, we can rename the $s_{n1}$ to some new symbol.

    In our example, this is equivalent to the following:
    \begin{align*}
        \forall i,z: T_{izz}      & = \bigoplus\limits_{j,y} A_{ij} B_{yy} C_{zz} \left(\1_a\right)_{ij}          \\
                                  & = \bigoplus\limits_{j,y} A_{ij} B_{yy} C_{zz} \odot \begin{cases}
            \1 & \text{if $i = j$}, \\
            \0 & \text{else}
        \end{cases} \\
                                  & = \bigoplus\limits_{y} A_{ii} B_{yy} C_{zz}                                   \\
        \iff \forall x,z: T_{xzz} & = \bigoplus\limits_{y} A_{xx} B_{yy} C_{zz}.
    \end{align*}

    Therefore, in all three cases, the symbols, that are used in an index string for a unity matrix, can simply be replaced by a single symbol.
\end{proof}

From these three lemmata, \cref{thm:nested_einsum:general} follows with the following procedure:
\begin{enumerate}[label={Step \arabic*:}, align=left]
    \item Apply \cref{lemma:nested_einsum:1} to all of the symbols in the input string $\bm{\hat{s}_u}$ and therefore replace it with a new index string $\bm{s'_u} \in S^{n_u}$ where $\bm{s'_u}$ contains no duplicate symbols.
    \item Apply \cref{lemma:nested_einsum:2} to all of the symbols in the output string $\bm{s_u}$ and therefore replace it with the same new index string $\bm{s'_u}$.
    \item Apply \cref{thm:nested_einsum:simple} to the nested expression and therefore compress the nested expression into a single expression with lots of unity matrices.
          This is possible because the input string and output string for $U$ are both $\bm{s'_u}$ now.
    \item Apply \cref{lemma:nested_einsum:3} to remove unity matrices from the compressed expression until there are no more of those unity matrices left, which were introduced in Step 1 and Step 2.
\end{enumerate}

\begin{proof}
    \small
    For the proof, we again demonstrate the arguments on the example used in \cref{thm:nested_einsum:general} for better understanding:
    $$(a,b,c,d,e,abbcde \rightarrow bc,v^{(1)}, v^{(2)}, v^{(3)}, v^{(4)}, v^{(5)}, (
        i,j,k,l \rightarrow iijkkl, v^{(6)}, v^{(7)}, v^{(8)}, v^{(9)}
        ))$$
    Applying Step 1 and Step 2 to this example results in the following Einsum expression if $\bm{s'_u} = s_1 s_2 s_3 s_4 s_5 s_6$:
    \begin{gather*}
        (a,b,c,d,e,s_1 s_2 s_3 s_4 s_5 s_6, a s_1, b s_2, b s_3, c s_4, d s_5, e s_6 \rightarrow bc,\\
        v^{(1)}, v^{(2)}, v^{(3)}, v^{(4)}, v^{(5)}, \1, \1, \1, \1, \1, \1,\\
        (i,j,k,l, i s_1, i s_2, j s_3, k s_4, k s_5, l s_6 \rightarrow s_1 s_2 s_3 s_4 s_5 s_6,\\
        v^{(6)}, v^{(7)}, v^{(8)}, v^{(9)}, \1, \1, \1, \1, \1, \1)),
    \end{gather*}
    where we used $\1$ to indicate unity matrices of different sizes, because the sizes can be derived from the context and are not important for better understanding.
    Applying Step 3 results in the following compressed expression:
    \begin{gather*}
        (a,b,c,d,e,i,j,k,l, i s_1, i s_2, j s_3, k s_4, k s_5, l s_6, a s_1, b s_2, b s_3, c s_4, d s_5, e s_6 \rightarrow bc,\\
        v^{(1)}, v^{(2)}, v^{(3)}, v^{(4)}, v^{(5)}, v^{(6)}, v^{(7)}, v^{(8)}, v^{(9)}, \1, \1, \1, \1, \1, \1, \1, \1, \1, \1, \1, \1)
    \end{gather*}

    The only fact that remains to be understood is why the removal of the unity matrices in Step 4 leads to the transformation described in \cref{thm:nested_einsum:general}.
    For this, we construct another undirected graph $G' = (V', E')$ that we call \textit{extended symbol graph}.
    In the extended symbol graph, the nodes consist of all symbols from $\bm{s_u}$, $\bm{s'_u}$, and $\bm{\hat{s}_u}$.
    An edge $\smallset{u,v}$ exists precisely when the compressed expression contains a unity matrix with index string $uv$, that was introduced in Step 1 or Step 2.
    The extended symbol graph for our example is illustrated in \cref{fig:nested_expressions:example_extended_symbol_graph}.

    \begin{figure}[h]
        \centering
        \begin{tikzpicture}[semithick, scale=0.6]

            \node[state] (a) at (0, 0)		{$a$};
            \node[state] (b) at (3, 0)		{$b$};
            \node[state] (c) at (6, 0) 		{$c$};
            \node[state] (d) at (9, 0) 		{$d$};
            \node[state] (e) at (12, 0) 	{$e$};

            \node[state] (s1) at (-0.25, -2)        {$s_1$};
            \node[state] (s2) at (2.25, -2)         {$s_2$};
            \node[state] (s3) at (4.75, -2)         {$s_3$};
            \node[state] (s4) at (7.25, -2)         {$s_4$};
            \node[state] (s5) at (9.75, -2)         {$s_5$};
            \node[state] (s6) at (12.25, -2)        {$s_6$};

            \node[state] (i) at (1.5, -4) 		{$i$};
            \node[state] (j) at (4.5, -4) 		{$j$};
            \node[state] (k) at (7.5, -4) 		{$k$};
            \node[state] (l) at (10.5, -4) 		{$l$};

            \path (a) edge (s1);
            \path (b) edge (s2);
            \path (b) edge (s3);
            \path (c) edge (s4);
            \path (d) edge (s5);
            \path (e) edge (s6);

            \path (i) edge (s1);
            \path (i) edge (s2);
            \path (j) edge (s3);
            \path (k) edge (s4);
            \path (k) edge (s5);
            \path (l) edge (s6);

            \begin{pgfonlayer}{background}
                \draw[gray!15, fill=gray!15, line width=12mm, line cap=round, line join=round] (a.center) -- (s1.center) -- (i.center) -- (j.center) -- (s3.center) -- (b.center) -- cycle;
            \end{pgfonlayer}

        \end{tikzpicture}
        \caption{Extended symbol graph for the example with the first component highlighted}
        \label{fig:nested_expressions:example_extended_symbol_graph}
    \end{figure}

    Now, every newly introduced unity matrix is represented by an edge in the extended symbol graph.
    If a unity matrix is removed with \cref{lemma:nested_einsum:3}, the symbols connected by the representing edge collapse into one symbol,
    and the rest of the extended symbol graph stays the same.
    An example of this collapse is illustrated in \cref{fig:nested_expressions:collapsed_extended_symbol_graph}.

    \begin{figure}[h]
        \centering
        \begin{tikzpicture}[semithick, scale=0.6]

            % graph where the unity matrix was not removed

            \node[state] (a) at (0, 0)		{$a$};
            \node[state] (b) at (3, 0)		{$b$};

            \node[state] (s1) at (-0.25, -2)        {$s_1$};
            \node[state] (s2) at (2.25, -2)         {$s_2$};
            \node[state] (s3) at (4.75, -2)         {$s_3$};

            \node[state] (i) at (1.5, -4) 		{$i$};
            \node[state] (j) at (4.5, -4) 		{$j$};

            \path (a) edge (s1);
            \path[dashed] (b) edge (s2);
            \path (b) edge (s3);

            \path (i) edge (s1);
            \path (i) edge (s2);
            \path (j) edge (s3);

            % graph where the unity matrix was removed

            \node[state] (removed_a) at (9, 0)		{$a$};
            \node (removed_b) at (12, 0)		{};

            \node[state] (removed_s1) at (8.75, -2)    {$s_1$};
            \node[state, dashed] (removed_x) at (11.25, -2)         {$x$};
            \node[state] (removed_s3) at (13.75, -2)         {$s_3$};

            \node[state] (removed_i) at (10.5, -4) 		{$i$};
            \node[state] (removed_j) at (13.5, -4) 		{$j$};

            \path (removed_a) edge (removed_s1);
            \path (removed_x) edge (removed_s3);

            \path (removed_i) edge (removed_s1);
            \path (removed_i) edge (removed_x);
            \path (removed_j) edge (removed_s3);

            \begin{pgfonlayer}{background}
                \draw[gray!15, fill=gray!15, line width=12mm, line cap=round, line join=round] (a.center) -- (s1.center) -- (i.center) -- (j.center) -- (s3.center) -- (b.center) -- cycle;
                \draw[gray!15, fill=gray!15, line width=12mm, line cap=round, line join=round] (removed_a.center) -- (removed_s1.center) -- (removed_i.center) -- (removed_j.center) -- (removed_s3.center) -- (removed_b.center) -- cycle;
            \end{pgfonlayer}

            % connect them
            \draw[->, thick] (6, -2) -- (7.5, -2);

        \end{tikzpicture}
        \caption{First component of the extended symbol graph after removing the unity matrix represented by the edge $\smallset{b, s_2}$}
        \label{fig:nested_expressions:collapsed_extended_symbol_graph}
    \end{figure}

    Therefore, after repeatedly applying \cref{lemma:nested_einsum:3},
    all nodes that are part of the same component in the extended symbol graph will collapse into one symbol.
    Now the only thing left to show is that two nodes are connected in $G$ exactly if they are also connected in $G'$.
    This can be seen by understanding the edges of $G$ and $G'$.
    In $G$, two symbols $s_{ui}$ and $\hat{s}_{uj}$ share an edge if they share a position $i = j$.
    In $G'$, two symbols $s_{ui}$ and $\hat{s}_{uj}$ share a neighbour $s'_{uk}$ if both were replaced by $s'_{uk}$ in Step one and Step two,
    which happens exactly if they share a position $i = j = k$.
    Therefore every edge in $G$ is represented by a shared neighbour in $G'$.
    Now, because every $s'_{uk}$ has exactly two neighbours, and because there are no direct edges between any symbols of $\bm{s_u}$ and $\bm{\hat{s}_u}$, there are no more edges in $G'$ other than the ones that contribute to a shared neighbour $s'_{uk}$.
    Then, because sharing an edge in $G$ is the same as sharing a neighbour in $G'$, two symbols are in the same component in $G$ precisely when they are also in the same component in $G'$.

    Therefore collapsing every edge in the extended symbol graph leads to the symbol map defined in \cref{thm:nested_einsum:general}.
\end{proof}