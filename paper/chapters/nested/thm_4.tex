\section{General Nested Expressions}

In the final generalisation of compressing nested Einsum expressions, all duplication breaking is allowed.
This has no application for linear algebra, as all the previous theorems had, because this first comes into play with third order tensors.
% This is because, with two or less axes, there is no possibility of simultaniously breaking a duplication and introducing a new one.
% Nevertheless, it serves as a useful tool for compressing all kinds of nested Einsum expressions.
But it still serves as a useful tool, because with it, we are able to compress every nested Einsum expression,
if the nested expressions are over the same semiring.

The following example is an expression, which we cannot compress with the previous theorems:
$$(a,b,c,d,e,abbde \rightarrow bc, v^{(1)}, v^{(2)}, v^{(3)}, v^{(4)}, v^{(5)}, (
    i,j,k,l \rightarrow iijkkl, v^{(6)}, v^{(7)}, v^{(8)}, v^{(9)}
    ))$$
for $v^{(i)} \in \R^{d_{vi}}$ with $i \in [9]$, $\bm{d_v} = (f, f, g, g, h, f, f, g, h) \in \N^{9}$.
In the following theorem, we will explore how to compress expressions such as this one.
Again, we use disjoined sets of symbols for the inner and outer expression to help us in the formulation and the proof.

\begin{theorem}
    \label{thm:nested_einsum:4}
    For $i \in [m + n]$, let $T^{(i)}$ be an $n_i$-th order tensor with index strings $\bm{s_i} \in S^{n_i}$.
    Let $\bm{s_u}$ be an index string for the $n_u$-th order tensor $U$, which is defined as follows:
    $$U := (\bm{s_{m + 1}},\dots,\bm{s_{m + n}} \rightarrow \bm{s_u}, T^{(m + 1)},\dots,T^{(m + n)})$$
    Also let $\bm{\hat{s}_u}$ be alternative index strings for $U$.

    Let $s_v$ be an index string and
    $$V := (\bm{s_1},\dots,\bm{s_m}, \bm{\hat{s}_u} \rightarrow \bm{s_v}, T^{(1)},\dots,T^{(m)}, U)$$
    where the first and second Einsum expression share no symbols.
    Then these nested Einsum expressions can also be compressed into a single Einsum expression.

    Once again, a map $\nu: S \rightarrow S$ has to be applied to the index strings before substituting index strings.
    The definition of the map this time is somewhat more complex.
    As in the previous two theorems, this map holds information about which symbols are essentially used together as one index.

    For the definition of the map $\nu$, we first construct an undirected graph $G = (V, E)$ that we call \textit{symbol graph}.
    % $V = \sigma(\bm{s_v}) \cup \sigma(\bm{s_u}) \cup \sigma(\bm{\hat{s}_u}) \cup \bigcup_{i \in [m + n]} \sigma(\bm{s_i})$
    In the symbol graph, the nodes consist of all symbols from both expressions.
    The edges are $E = \smallset{\smallset{s_{uj}, \hat{s}_{uj}} \mid j \in [n_u]}$,
    which connects all symbols from $\bm{s_u}$ and $\bm{\hat{s}_u}$ that share an index.
    The symbol graph for our example is displayed in \autoref{fig:nested_expressions:example_symbol_graph}.

    In the symbol graph, if two symbols are connected, then they need to be iterated over at the same time in the compressed expression, because they are essentially the same index.
    Therefore, it makes sense assigning a symbol $s_C \in S \setminus V$ to each of the graphs components $C$.
    Then we can define $\nu$ as follows:
    $$\nu(s) := \begin{cases}
            s_C & \text{if } s \in C \\
            s   & \text{else}
        \end{cases},$$
    which can be extended to map entire index strings as in \autoref{thm:nested_einsum:2}.
    In our example, the components are $\smallset{a,b,i,j}$, $\smallset{c,d,k}$, and $\smallset{e,l}$.
    Therefore we could use
    $$\nu(s) := \begin{cases}
            x & \text{if } s \in \smallset{a,b,i,j} \\
            y & \text{if } s \in \smallset{c,d,k}   \\
            z & \text{if } s \in \smallset{e,l}     \\
            s & \text{else}
        \end{cases}.$$

    \begin{figure}[h]
        \centering
        \begin{tikzpicture}[node distance = 3cm, semithick]

            \node[state] (a)					{$a$};
            \node[state] (b) [right of=a]		{$b$};
            \node[state] (c) [right of=b] 		{$c$};
            \node[state] (d) [right of=c] 		{$d$};
            \node[state] (e) [right of=d] 		{$e$};
            \node (middle) at ($(a)!0.5!(b)$) {};
            \node[state] (i) [below of=middle] 		{$i$};
            \node[state] (j) [right of=i] 		{$j$};
            \node[state] (k) [right of=j] 		{$k$};
            \node[state] (l) [right of=k] 		{$l$};

            \node (x_center) at ($(a)!0.5!(j)$) {};
            \node (y_center) at ($(c)!0.5!(d|-k)$) {};
            \node (z_center) at ($(e)!0.5!(l)$) {};
            \node (x) [below of=x_center] {$x$};
            \node (y) [below of=y_center] {$y$};
            \node (z) [below of=z_center] {$z$};

            \path (a) edge (i);
            \path (i) edge (b);
            \path (b) edge (j);
            \path (c) edge (k);
            \path (k) edge (d);
            \path (e) edge (l);

            \begin{pgfonlayer}{background}
                \draw[gray!15, fill=gray!15, line width=12mm, line cap=round, line join=round] (a.center) -- (b.center) -- (j.center) -- (i.center) -- cycle;
                \draw[gray!15, fill=gray!15, line width=12mm, line cap=round, line join=round] (c.center) -- (d.center) -- (k.center) -- cycle;
                \draw[gray!15, fill=gray!15, line width=12mm, line cap=round, line join=round] (e.center) -- (l.center) -- cycle;
            \end{pgfonlayer}

        \end{tikzpicture}
        \caption{Symbol graph for the example}
        \label{fig:nested_expressions:example_symbol_graph}
    \end{figure}

    In this general form, this map is applied to all index strings from both expressions before the substitution.
    Let $\bm{\hat{s}_i} := \nu(\bm{s_i})$ for $i \in [m + n]$, $\bm{\hat{s}_v} := \nu(\bm{s_v})$, then the compressed Einsum expression is the following:
    $$V = (\bm{\hat{s}_1}, \dots, \bm{\hat{s}_{m + n}} \rightarrow \bm{\hat{s}_v}, T^{(1)},\dots,T^{(m + n)})$$
    which helps us to compress the example:
    \begin{gather*}
        (a,b,c,d,e,abbde \rightarrow bc, v^{(1)}, v^{(2)}, v^{(3)}, v^{(4)}, v^{(5)}, (
        i,j,k,l \rightarrow iijkkl, v^{(6)}, v^{(7)}, v^{(8)}, v^{(9)}
        ))\\
        =(x,x,y,y,z,x,x,y,z \rightarrow xy, v^{(1)}, v^{(2)}, v^{(3)}, v^{(4)}, v^{(5)}, v^{(6)}, v^{(7)}, v^{(8)}, v^{(9)})
    \end{gather*}
\end{theorem}

\begin{proof}
    \small
    Wird spannend.
\end{proof}