\section{General Nested Expressions}

In the final generalisation of compressing nested Einsum expressions, all duplication breaking is allowed.
This has no application for linear algebra, as all the previous theorems had, because this first comes into play with third order tensors.
This is because, with two or less axes, there is no possibility of simultaniously breaking a duplication and introducing a new one.
Nevertheless, it serves as a useful tool for compressing all kinds of nested Einsum expressions.

% TODO: example
% In the following theorem, we explore a way of compressing the expression
% $$(ijk, jk \rightarrow ijk, A, (l \rightarrow ll, v))$$
% Again, we use disjoined sets of symbols for the inner and outer expression to help us in the formulation and the proof.

\begin{theorem}
    \label{thm:nested_einsum:4}
    For $i \in [m + n]$, let $T^{(i)}$ be an $n_i$-th order tensor with index strings $\bm{s_i} \in S^{n_i}$.
    Let $\bm{s_u}$ be an index string for the $n_u$-th order tensor $U$, which is defined as follows:
    $$U := (\bm{s_{m + 1}},\dots,\bm{s_{m + n}} \rightarrow \bm{s_u}, T^{(m + 1)},\dots,T^{(m + n)})$$
    Also let $\bm{\hat{s}_u}$ be alternative index strings for $U$.

    Let $s_v$ be an index string and
    $$V := (\bm{s_1},\dots,\bm{s_m}, \bm{\hat{s}_u} \rightarrow \bm{s_v}, T^{(1)},\dots,T^{(m)}, U)$$
    where the first and second Einsum expression share no symbols.
    Then these nested Einsum expressions can also be compressed into a single Einsum expression.

    Once again, a map $\omega: S \rightarrow S$ has to be applied to the index strings before substituting index strings.
    The definition of the map this time is somewhat more complex.
    As in the previous two theorems, this map holds information about which symbols are essentially used together as one index.

    For the definition of the map $\omega$, we first construct an undirected graph $G = (V, E)$,
    in which the nodes $V = \sigma(\bm{s_v}) \cup \sigma(\bm{s_u}) \cup \sigma(\bm{\hat{s}_u}) \cup \bigcup_{i \in [m + n]} \sigma(\bm{s_i})$ consist of all symbols from both expressions.
    Now the edges of this graph are $E = \smallset{\smallset{s_{uj}, \hat{s}_{uj}} \mid \exists j \in [n_u]: s_{uj} \neq \hat{s}_{uj}}$,
    which connects all symbols from $\bm{s_u}$ and $\bm{\hat{s}_u}$ that share an index.

    In this graph, if two symbols are connected, then they to be iterated over at the same time in the compressed expression, because they are essentially the same index.
    Therefore, it makes sense assigning a symbol $s_C \in S \setminus V$ to each of the graphs components $C$.
    Then we can define $\omega$ as follows:
    $$\omega(s) := \begin{cases}
            s_C & \text{if } s \in C \\
            s   & \text{else}
        \end{cases}$$

    $\omega$ can be extended in a similar way as $\mu, \nu$ to map entire index strings.
    In this general form, the this map is applied to all index strings from both expressions before the substitution.
    Let $\bm{\hat{s}_i} := \omega(\bm{s_i})$ for $i \in [m + n]$, $\bm{\hat{s}_v} := \omega(\bm{s_v})$, then the compressed Einsum expression is the following:
    $$V = (\bm{\hat{s}_1}, \dots, \bm{\hat{s}_{m + n}} \rightarrow \bm{\hat{s}_v}, T^{(1)},\dots,T^{(m + n)})$$
\end{theorem}

\begin{proof}
    \small
    Wird spannend.
\end{proof}