\section{Einsum}
\begin{definition}
    Let $S$ be a collection of symbols that correspond to non-empty sets of tensor axes.
    For $i \in [n]$, let $T^{(i)}$ be a $d^{(i)}$-th order tensor with axis symbols $s^{(i)} \in S^{d^{(i)}}$ for their unique axes $a^{(i)}_j \in \mathcal{N}$.
    The actual value of $a^{(i)}_j$ does not matter. These variables are just used as a unique identifier for the axes so that the definition is more rigorous.

    The size of the tensor $T^{(i)}$ on the axis $a^{(i)}_j$ is denoted as $d^{(i)}_j$ for $j \in [d_i]$.
    For this definition it must hold that $s^{(i)}_j = s^{(i')}_{j'} \implies d^{(i)}_j = d^{(i')}_{j'}$ for all $i,i' \in [n], j \in [d^{(i)}], j' \in [d^{(i')}]$.
    This means that the set of axes that a symbol $s \in S$ corresponds to, must all have the same size.

    Therefore we can also denote the size of the axes that a symbol $s \in S$ corresponds to as $d_s := d^{(i)}_j$ for all $i \in [n], j \in [d^{(i)}]$ with $s = s^{(i)}_j$.
    Note that not all same size axes have to assigned the same symbol. E.g. with $S = \set{i,j}$, a square matrix $T^{(1)}$ could have axes $s^{(1)} = (i, i)$ or $s^{(1)} = (i, j)$.

    Let $T^{(0)}$ be a $d^{(0)}$-th order tensor with axis symbols $s^{(0)} \in S^{d^{(0)}}$.
    This tensor will be the result of the computation.
    Its axis symbols decide which axes are kept and which axes will be summed over.
    The symbols $S$ are partitioned into bound symbols $B = \smallset{s^{(0)}_j \mid j \in [d^{(0)}]}$ and  free symbols $F = S \setminus B$.
    We will keep all axes corresponding to symbols in $B$ and sum over all axes corresponding to symbols in $F$.

    Let $\mathcal{I} := \prod_{s \in B} [d_s]$ and $\mathcal{J} := \prod_{s \in F} [d_s]$.
    These will be the space of multi-indices which we iterate over.
    For $I \in \mathcal{I}, J \in \mathcal{J}$, the concatenation of the indices is denoted as $IJ$.
    The projection of the multi-index over all tensors $IJ$ on the multi-index for tensor $T^{(i)}$ is denoted as $IJ:s^{(i)}$.
    $I:s^{(0)}$ is defined analogously.

    Given a semiring $R = (M, \oplus, \odot)$. Then the einsum-notation denotes the following:
    \begin{align*}
        T^{(0)}                                               & = (s^{(1)},\dots,s^{(n)} \rightarrow s^{(0)}, T^{(1)},\dots,T^{(n)})_R                  \\
        :\iff \forall I \in \mathcal{I}: T^{(0)}_{I: s^{(0)}} & = \bigoplus\limits_{J \in \mathcal{J}} \bigodot\limits_{i = 1}^{n} T^{(i)}_{IJ:s^{(i)}}
    \end{align*}
    if $B,F$ are non-empty. However, if there are no free symbols, the summation is removed:
    \begin{align*}
        T^{(0)}                                               & = (s^{(1)},\dots,s^{(n)} \rightarrow s^{(0)}, T^{(1)},\dots,T^{(n)})_R \\
        :\iff \forall I \in \mathcal{I}: T^{(0)}_{I: s^{(0)}} & = \bigodot\limits_{i = 1}^{n} T^{(i)}_{IJ:s^{(i)}}
    \end{align*}
    If there are no bound symbols, then $s^{(0)}$ is empty and $T^{(0)}$ is a scalar:
    \begin{align*}
        T^{(0)}       & = (s^{(1)},\dots,s^{(n)} \rightarrow s^{(0)}, T^{(1)},\dots,T^{(n)})_R                 \\
        :\iff T^{(0)} & = \bigoplus\limits_{J \in \mathcal{J}} \bigodot\limits_{i = 1}^{n} T^{(i)}_{J:s^{(i)}}
    \end{align*}

    Note that duplicate entries in $s^{(i)}, i \in [n]$ will result in the expression iterating over the input tensors in a sort of \textit{diagonal} way.
    Duplicate entries in $s^{(0)}$ will result in some entries of $T^{(0)}$ not being defined. These can be set to an arbitrary value like 0.

    In case the semiring can be derived from the context, or if it is irrelevant, it can be left out from the expression.
\end{definition}

\subsection{Examples}
All following examples use the standard semiring $R = (\R, +, \cdot)$.
\begin{itemize}
    \item matrix-vector multiplication: Let $A \in \R^{m \times n}, v \in \R^{n}$. Then
          $$A \cdot v = (ij, j \rightarrow i, A, v)$$
    \item matrix-matrix multiplication: Let $A \in \R^{m \times r}, B \in \R^{r \times n}$. Then
          $$A \cdot B = (ik, kj \rightarrow ij, A, B)$$
    \item trace: Let $A \in \R^{n \times n}$. Then
          $$\text{trace}(A) = (ii \rightarrow, A)$$
    \item squared frobenius norm: Let $A \in \R^{n \times n}$. Then
          $$\abs{A}_2^2 = (ij, ij \rightarrow,A,A)$$
    \item diagonal matrix: Let $v \in \R^{n}$. Then
          $$\text{diag}(v) = (i \rightarrow ii, v)$$
\end{itemize}

\subsection{Nested Einsum Expressions}
\begin{theorem}
    For $i \in [m + n + 1]$, let $T^{(i)}$ be a $d^{(i)}$-th order tensor with axis symbols $s^{(i)} \in S^{d^{(i)}}$, $o := m + n + 1$.
    Let
    $$T^{(0)} := (s^{(1)},\dots,s^{(m)}, s^{(x)} \rightarrow s^{(0)}, T^{(1)},\dots,T^{(m)}, T^{(o)})$$
    and
    $$T^{(o)} = (s^{(m + 1)},\dots,s^{(m + n)} \rightarrow s^{(o)}, T^{(m + 1)},\dots,T^{(m + n)})$$
    where the free symbols of the second einsum expression share no symbols with the first einsum expression.

    Then
    $$T^{(0)} = (s^{(1)}, \dots, s^{(m + n)} \rightarrow s^{(0)}, T^{(1)}, \dots, T^{(m + n)})$$
\end{theorem}
\begin{proof}
    Let $B, B', F, F'$ be the bound and free symbols of the first and second einsum expression respectively.
    W.l.o.g. they are all non-empty.
    From them we can derive $\mathcal{I}, \mathcal{I}', \mathcal{J}, \mathcal{J}'$ as above.
    Then
    \begin{align*}
        T^{(0)}                                              & = (s^{(1)},\dots,s^{(m)}, s^{(o)} \rightarrow s^{(0)}, T^{(1)},\dots,T^{(m)}, T^{(o)})                                                                                                             \\
        \iff \forall I \in \mathcal{I}: T^{(0)}_{I: s^{(0)}} & = \bigoplus\limits_{J \in \mathcal{J}} \bigodot\limits_{i = 1}^{m} T^{(i)}_{IJ:s^{(i)}} \odot T^{(o)}_{IJ:s^{(o)}}                                                                                 \\
                                                             & = \bigoplus\limits_{J \in \mathcal{J}} \bigodot\limits_{i = 1}^{m} T^{(i)}_{IJ:s^{(i)}} \odot \bigoplus\limits_{J' \in \mathcal{J}'} \bigodot\limits_{i' = m + 1}^{m + n} T^{(i')}_{IJJ':s^{(i')}} \\
                                                             & = \bigoplus\limits_{J \in \mathcal{J}} \bigoplus\limits_{J' \in \mathcal{J}'} \bigodot\limits_{i = 1}^{m} T^{(i)}_{IJ:s^{(i)}} \odot \bigodot\limits_{i = m + 1}^{m + n} T^{(i)}_{IJJ':s^{(i)}}    \\
                                                             & = \bigoplus\limits_{J \in \mathcal{J} \times \mathcal{J}'} \bigodot\limits_{i = 1}^{m + n} T^{(i)}_{IJ:s^{(i)}}                                                                                    \\
        \iff T^{(0)}                                         & = (s^{(1)}, \dots, s^{(m + n)} \rightarrow s^{(0)}, T^{(1)}, \dots, T^{(m + n)})
    \end{align*}
    where the third equality follows from
    $$\forall I' \in \mathcal{I}': T^{(o)}_{I': s^{(o)}} = \bigoplus\limits_{J' \in \mathcal{J}'} \bigodot\limits_{i' = m + 1}^{m + n} T^{(i')}_{I'J':s^{(i')}},$$
    $B' \subseteq B \cup F$, and $(B \cup F) \cap F' = \emptyset$. The last two facts are required so that $IJJ':s^{(i')}$ is well-defined and projects on the same indices as $I'J':s^{(i')}$.
    The fourth equality follows from the distributivity in a semiring.
\end{proof}

\subsection{A More General Result}
\begin{theorem}
    For $i \in [m + n + 1]$, let $T^{(i)}$ be a $d^{(i)}$-th order tensor with axis symbols $s^{(i)} \in S^{d^{(i)}}$, $o := m + n + 1$.
    Also let $\hat{s}^{(o)}$ be alternative axis symbols for $T^{(o)}$ with $s^{(o)}_j = s^{(o)}_{j'} \implies \hat{s}^{(o)}_j = \hat{s}^{(o)}_{j'}$ for all $j, j' \in [d^{(o)}]$.
    Let
    $$T^{(0)} := (s^{(1)},\dots,s^{(m)}, \hat{s}^{(o)} \rightarrow s^{(0)}, T^{(1)},\dots,T^{(m)}, T^{(o)})$$
    and
    $$T^{(o)} = (s^{(m + 1)},\dots,s^{(m + n)} \rightarrow s^{(o)}, T^{(m + 1)},\dots,T^{(m + n)})$$
    where the free symbols of the second einsum expression share no symbols with the first einsum expression.
    Let $\nu: S \rightarrow S$ such that
    $$\nu(s) = \begin{cases}
            \hat{s}^{(o)}_j & \text{if }\exists j \in [d^{(o)}]: s^{(o)}_j = s \\
            s               & \text{else}
        \end{cases}$$
    which maps symbols in $s^{(o)}$ to the symbol at the same index in $\hat{s}^{(o)}$ and all other symbols to themselves.
    $\nu$ can be extended to map from axis symbol tuples by setting $\nu(s^{(i)}) \in S^{d^{(i)}}, \nu(s^{(i)})_j := \nu(s^{(i)}_j)$.

    Let $\hat{s}^{(i)} := \nu(s^{(i)})$
    Then
    $$T^{(0)} = (s^{(1)},\dots,s^{(m)}, \hat{s}^{(m + 1)}, \dots, \hat{s}^{(m + n)} \rightarrow s^{(0)}, T^{(1)},\dots,T^{(m + n)})$$
\end{theorem}
\begin{proof}
    Let \dots
    \begin{align*}
        T^{(0)}                                              & = (s^{(1)},\dots,s^{(m)}, s^{(o)} \rightarrow \hat{s}^{(0)}, T^{(1)},\dots,T^{(m)}, T^{(o)})                                                                                                                       \\
        \iff \forall I \in \mathcal{I}: T^{(0)}_{I: s^{(0)}} & = \bigoplus\limits_{J \in \mathcal{J}} \bigodot\limits_{i = 1}^{m} T^{(i)}_{IJ:s^{(i)}} \odot T^{(o)}_{IJ:\hat{s}^{(o)}}                                                                                           \\
                                                             & = \bigoplus\limits_{J \in \mathcal{J}} \bigodot\limits_{i = 1}^{m} T^{(i)}_{IJ:s^{(i)}} \odot \mu\left(\bigoplus\limits_{J' \in \mathcal{J}'} \bigodot\limits_{i' = m + 1}^{m + n} T^{(i')}_{IJJ':s^{(i')}}\right) \\
                                                             & = \bigoplus\limits_{J \in \mathcal{J}} \bigodot\limits_{i = 1}^{m} T^{(i)}_{IJ:s^{(i)}} \odot \bigoplus\limits_{J' \in \mathcal{J}'} \bigodot\limits_{i' = m + 1}^{m + n} \mu\left(T^{(i')}_{IJJ':s^{(i')}}\right) \\
                                                             & = \bigoplus\limits_{J \in \mathcal{J}} \bigodot\limits_{i = 1}^{m} T^{(i)}_{IJ:s^{(i)}} \odot \bigoplus\limits_{J' \in \mathcal{J}'} \bigodot\limits_{i' = m + 1}^{m + n} T^{(i')}_{IJJ':\hat{s}^{(i')}}           \\
                                                             & = \bigoplus\limits_{J \in \mathcal{J} \times \mathcal{J}'} \bigodot\limits_{i = 1}^{m} T^{(i)}_{IJ:s^{(i)}} \odot \bigodot\limits_{i = m + 1}^{m + n} T^{(i)}_{IJ:\hat{s}^{(i)}}                                   \\
        \iff T^{(0)}                                         & = (s^{(1)},\dots,s^{(m)}, \hat{s}^{(m + 1)}, \dots, \hat{s}^{(m + n)} \rightarrow s^{(0)}, T^{(1)},\dots,T^{(m + n)})
    \end{align*}
\end{proof}
