\section{Einsum}
Given two third-order tensor $A \in \R^{3 \times 4 \times 5}, B \in \R^{3 \times 3 \times 5}$ and a vector $v \in \R^4$.
Consider the following computation for a matrix $C \in \R^{3 \times 3}$:
$$\forall i \in [3]: \forall j \in [4]: C_{ij} = \sum\limits_{k = 1}^{5} A_{ijk} B_{iik} v_j$$
The original Einstein-notation for summation removes redundant formalism ("boilerplate") from this expression:
$$C_{ij} = A_{ijk} B_{iik} v_j$$
where it is assumed that $C$ is defined for all possible $i,j$.
We sum over all indices that are not used to index the output.
In this example, we therefore have to sum over all possible values of $k$, because it is not used to index $C_{ij}$.
Note how it is clear what the shape of $C$ is, because $i$ and $j$ were used to index the tensors $A$, $B$, and $v$,
for which we defined the dimensions on every axis.

This notation is essentially the inspiration for einsum, which might be apparent given the name einsum.
Einsum is just an adaptation of this style, which makes it easier to use in programming.
With it, we can write the above expression like this:
$$C := (ijk, iik, j \rightarrow ij, A, B, v)$$
Through the following definition, we hope to clear up why this einsum expression results in the computation above,
and what computation a general einsum expression results in.

\begin{definition}
    Einsum expressions generally evaluate to some computation over tensors, so let $T^{(1)},\dots,T^{(n)}$ be our input tensors,
    where $T^{(i)}$ is a $n_i$-th order tensor for $i \in [n]$.
    The core of the einsum expression are index strings. For this, we first need a collection of symbols $S$.
    The respective index string for a tensor $T^{(i)}$ is then just a tuple $\bm{s_i} \in S^{n_i}$,
    composed of symbols $s_{ij} \in S$ for $j \in [n_i]$.
    We refer to the index string that is right of the arrow ($\rightarrow$) as the output string $\bm{s_t}$.

    In our example this could be $S = \smallset{i,j,k}$ with respective index strings
    $\bm{s_1} = ijk$ for $T^{(1)} = A$,
    $\bm{s_2} = iik$ for $T^{(2)} = B$,
    $\bm{s_3} = j$ for $T^{(3)} = v$,
    and $\bm{s_t} = ij$.
    The individual symbols are $s_{11} = i$, $s_{12} = j$, $s_{13} = k$, $s_{12} = i$, $s_{22} = i$, $s_{23} = k$, $s_{31} = j$, $s_{t1} = i$, $s_{t2} = j$.

    In order to refer to individual tensor axes, let us numerate them with $a_{ij} \in \mathbb{N}$,
    where $a_{ij}$ denotes the $j$-th axis of the tensor $i$-th tensor $T^{(i)}$.
    The actual value of $a_{ij}$ does not matter.
    These variables are just used as unique identifiers for the axes.

    The next step in the definition is to speak about the axis sizes.
    If we want to iterate over shared indices, it would be nice if the axes, that these indices are used for, share the same size.
    In our example, $A_{ijk}$ and $v_j$ share the symbol $s_{12} = s_{31} = j$.
    This means that the respective axes $a_{12}$ and $a_{31}$ have to have the same size, which happens to be 4 (four?).
    Let us express this formally.

    Let $d_{ij}$ denote the size of the axis $a_{ij}$ for $i \in [n], j \in [n_i]$.
    Then it must hold that $s_{ij} = s_{i'j'} \implies d_{ij} = d_{i'j'}$ for all $i,i' \in [n], j \in [n_i], j' \in [n_{i'}]$.

    Therefore we can also denote the size of all axes that a symbol $s \in S$ corresponds to as $d_s := d_{ij}$ for all $i \in [n], j \in [n_i]$ with $s = s_{ij}$.
    Note that not all same size axes have to assigned the same symbol. E.g. a square matrix could have index strings $\bm{s} = (i, i)$ or $\bm{s} = (i, j)$.

    The next step of the definition is figuring out which symbols are used for summation and which symbols are used for saving the result of the computation.
    In order to do this, it is useful to know which symbols are in an index string, because symbols can occur more than once in just one index string (as seen in $B_{iik}$ in our example).
    Therefore, let $\sigma(\bm{s})$ denote the set with all symbols used in an index string $\bm{s}$.
    E.g. $\sigma(\bm{s_2}) = \sigma(iik) = \smallset{ik}$.

    All symbols to the right of the arrow ($\rightarrow$) are used as an index for the result of the computation.
    These symbols are called \textit{bound} symbols $B = \sigma(\bm{s_t})$.
    All other symbols used in the expression are called \textit{free} symbols $F = \bigcup_{i \in [n]} \sigma(\bm{s_i}) \setminus \sigma(\bm{s_t})$.
    In einsum, we sum over all axes that belong to free symbols.
    It follows that the multi-index space we iterate over is $\mathcal{I} = \prod_{s \in B} [d_s]$ and the multi-index space we sum over is $\mathcal{J} = \prod_{s \in F} [d_s]$.
    In our example, the bound symbols are $B = \smallset{ij}$ and the free symbols are $F = \smallset{k}$.
    The multi-index space we iterate over is $d_i \times d_j = [3] \times [4]$.
    The multi-index space we sum over is $d_k = [5]$.

    From the definition of $\mathcal{I}$, it follows that $d_s$ has to be defined for all symbols $s \in B$.
    This means we have to add the constraint $\sigma(\bm{s_t}) \subseteq \bigcup_{i \in [n]} \sigma(\bm{s_i})$.

    However, we do not use every symbol in the multi-index spaces to index every input tensor.
    Instead, we use the index strings $\bm(s)$ to index the tensor.
    To formally express this, we need a projection from a multi-index $IJ \in \mathcal{I} \times \mathcal{J}$ to another multi-index, which includes only the symbols used in $\bm{s}$,
    in the same order as present in $\bm{s}$.
    We denote this as $IJ:\bm{s}$.
    Notice how this still allows duplication of indices given in $IJ$.
    This is needed, as can be seen in our example for $B_{iik}$,
    where a multi-index, e.g. $(1,4,2) \in \mathcal{I} \times \mathcal{J}$, is projected on the index string $iik$,
    which results in the multi-index $(1,4,2):iik = (1,1,2)$.

    In our example, we used the standard sum and multiplication as operators for computing our result.
    But with einsum, we allow the more general use of any semiring $R = (M, \oplus, \odot)$.
    With this, we can finally write down what computation a general einsum expression
    $$T := (\bm{s_1},\dots,\bm{s_n} \rightarrow \bm{s_t}, T^{(1)},\dots,T^{(n)})_R$$
    results in. It means that $T$ is a $\abs{\bm{s_t}}$-th order tensor with
    $$\forall I \in \mathcal{I}: T_{I: \bm{s_t}} = \bigoplus\limits_{J \in \mathcal{J}} \bigodot\limits_{i = 1}^{n} T^{(i)}_{IJ:\bm{s_t}}$$

    Because we also project the indices $I$ on the output string $\bm{s_t}$, we allow to iterate over duplicate indices,
    e.g. $\text{diag}(v) = (j \rightarrow jj, v)$.
    This leaves some entries of the result undefined.
    We define these entries to be the additive neutral element $\0$ in the given semiring $R$.
    This may sound arbitrary at first, but will be useful for later theorems.

    There are still some special case which need to be considered.
    If there are no free symbols in the expression, then the sum will be empty.
    But we still want the result of the computation of the product.
    Therefore, if $F = \emptyset$, then
    $$T := (\bm{s_1},\dots,\bm{s_n} \rightarrow \bm{s_t}, T^{(1)},\dots,T^{(n)})_R$$
    results in the computation of a $\abs{\bm{s_t}}$-th order tensor $T$ with
    $$\forall I \in \mathcal{I}: T_{I: \bm{s_t}} = \bigodot\limits_{i = 1}^{n} T^{(i)}_{I:\bm{s_t}}$$
    If there are no bound symbols, we will sum over all axes given by the symbols in the expression.
    Therefore, if $B = \emptyset$, then
    $$T := (\bm{s_1},\dots,\bm{s_n} \rightarrow , T^{(1)},\dots,T^{(n)})_R$$
    results in the computation of a scalar $T$ with
    $$T = \bigoplus\limits_{J \in \mathcal{J}} \bigodot\limits_{i = 1}^{n} T^{(i)}_{J:\bm{s_t}}$$

    In case the semiring can be derived from the context, or if it is irrelevant, it can be left out from the expression.
\end{definition}

\subsection{Examples}
All following examples use the standard semiring $R = (\R, +, \cdot)$.
\begin{itemize}
    \item matrix-vector multiplication: Let $A \in \R^{m \times n}, v \in \R^{n}$. Then
          $$A \cdot v = (ij, j \rightarrow i, A, v)$$
    \item matrix-matrix multiplication: Let $A \in \R^{m \times r}, B \in \R^{r \times n}$. Then
          $$A \cdot B = (ik, kj \rightarrow ij, A, B)$$
    \item trace: Let $A \in \R^{n \times n}$. Then
          $$\text{trace}(A) = (ii \rightarrow, A)$$
    \item squared Frobenius norm: Let $A \in \R^{n \times n}$. Then
          $$\abs{A}_2^2 = (ij, ij \rightarrow,A,A)$$
    \item diagonal matrix: Let $v \in \R^{n}$. Then
          $$\text{diag}(v) = (i \rightarrow ii, v)$$
\end{itemize}

\subsection{Nested Einsum Expressions}

In practice, concatenations of operations come naturally, e.g. computing the squared norm of a matrix-vector product $\abs{A \cdot v}_2^2$
for $A \in \R^{m \times n}, v \in \R^n$.
This would lead to a nested einsum expression $\abs{A \cdot v}_2^2 = (i,i \rightarrow , (ij, j \rightarrow i, A, v), (ij, j \rightarrow i, A, v))$.
This expression dictates the order of evaluating the expression.
In the example of the norm, the expression $(ij, j \rightarrow i, A, v)$ has to be evaluated before squaring and summing over the results of this computation.

This is limiting, because the order of evaluation might not yield optimal runtime.
This can be seen with a simple matrix-matrix-vector multiplication, which can be written as follows:
$$(A \cdot B) \cdot v = (ij, j \rightarrow i, (ik, kj \rightarrow ij, A, B), v)$$
which is clearly worse than the optimal contraction order
$$A \cdot (B \cdot v) = (ij, j \rightarrow i, A, (ij, j \rightarrow i, B, v))$$
for $A \in \R^{m \times r}, B \in \R^{r \times n}, v \in \R^r$.

But fortunately, all nested einsum expressions can be compressed into a single einsum expression, if they are computed over the same semiring.
This leaves the path of contraction up to the implementation.
In the following theorems, we assume that the computations are all over the same semiring $R = (M, \oplus, \odot)$.

\begin{theorem}
    \label{thm:nested_einsum:1}

    For $i \in [m + n]$, let $T^{(i)}$ be a $n_i$-th order tensor with index strings $\bm{s_i} \in S^{n_i}$.
    Let $\bm{s_u}, \bm{s_v}$ be index strings.
    Let
    $$U := (\bm{s_{m + 1}},\dots,\bm{s_{m + n}} \rightarrow \bm{s_u}, T^{(m + 1)},\dots,T^{(m + n)})$$
    and
    $$V := (\bm{s_1},\dots,\bm{s_m}, \bm{s_u} \rightarrow \bm{s_v}, T^{(1)},\dots,T^{(m)}, U)$$
    where the free symbols of the second einsum expression share no symbols with the first einsum expression.
    Then
    $$V = (\bm{s_1}, \dots, \bm{s_{m + n}} \rightarrow \bm{s_v}, T^{(1)}, \dots, T^{(m + n)})$$
\end{theorem}
\begin{proof}
    Let $B, B', F, F'$ be the bound and free symbols of the second and first einsum expression respectively.
    W.l.o.g. they are all non-empty.
    From them we can derive $\mathcal{I}, \mathcal{I}', \mathcal{J}, \mathcal{J}'$ as above.
    Then
    \begin{align*}
        V                                               & = (\bm{s_1},\dots,\bm{s_m}, \bm{s_u} \rightarrow \bm{s_v}, T^{(1)},\dots,T^{(m)}, U)                                                                                                                   \\
        \iff \forall I \in \mathcal{I}: V_{I: \bm{s_v}} & = \bigoplus\limits_{J \in \mathcal{J}} \bigodot\limits_{i = 1}^{m} T^{(i)}_{IJ:\bm{s_i}} \odot U_{IJ:\bm{s_u}}                                                                                         \\
                                                        & = \bigoplus\limits_{J \in \mathcal{J}} \bigodot\limits_{i = 1}^{m} T^{(i)}_{IJ:\bm{s_i}} \odot \bigoplus\limits_{J' \in \mathcal{J}'} \bigodot\limits_{i' = m + 1}^{m + n} T^{(i')}_{IJJ':\bm{s_{i'}}} \\
                                                        & = \bigoplus\limits_{J \in \mathcal{J}} \bigoplus\limits_{J' \in \mathcal{J}'} \bigodot\limits_{i = 1}^{m} T^{(i)}_{IJ:\bm{s_i}} \odot \bigodot\limits_{i = m + 1}^{m + n} T^{(i)}_{IJJ':\bm{s_{i'}}}   \\
                                                        & = \bigoplus\limits_{J \in \mathcal{J} \times \mathcal{J}'} \bigodot\limits_{i = 1}^{m + n} T^{(i)}_{IJ:\bm{s_i}}                                                                                       \\
        \iff V                                          & = (\bm{s_1}, \dots, \bm{s_{m + n}} \rightarrow \bm{s_v}, T^{(1)}, \dots, T^{(m + n)})
    \end{align*}
    where the third equality follows from
    $$\forall I' \in \mathcal{I}': U_{I': \bm{s_u}} = \bigoplus\limits_{J' \in \mathcal{J}'} \bigodot\limits_{i' = m + 1}^{m + n} T^{(i')}_{I'J':\bm{s_{i'}}},$$
    $B' \subseteq B \cup F$, and $(B \cup F) \cap F' = \emptyset$. The last two facts are required so that $IJJ':\bm{s_{i'}}$ is well-defined and projects on the same indices as $I'J':\bm{s_{i'}}$.
    The fourth equality follows from the distributivity in a semiring.
\end{proof}

\subsection{More General Results}

Compressing these nested einsum expressions is already helpful when the output string of the inner expression is exactly the same as the input string of the outer expression,
e.g. for $A \in \R^{m \times n}, v \in \R^{n}$:
\begin{align*}
    \abs{A \cdot v}_2^2 & = (i,i\rightarrow,(ij, j \rightarrow i, A, v),(ij, j \rightarrow i, A, v)) \\
                        & = (ij,j,ij,j\rightarrow,A,v,A,v)
\end{align*}
But sometimes, we need to access a different multi-index set than the one we computed, e.g. for $A \in \R^{m \times n}, B \in \R^{n \times m}, v \in \R^{n}$:
$$\text{trace}(A \cdot B) = (ii \rightarrow, (ik, kj \rightarrow ij, A, B))$$
or
$$A \cdot \text{diag}(v) = (ik, kj \rightarrow ij, A, (i \rightarrow ii, v))$$
For this, we need more general ways of compressing nested einsum expressions.

In the following theorem, we explore a way of compressing the expression
$$(ij, jjj \rightarrow i, A, (kl, lo \rightarrow kko, B, C))$$
Note that, for the theorem, we use disjoined sets of symbols for the inner and outer expression.
This helps in the proof, and is not a real constraint in practice,
because we can just rename the symbols in different scopes.
E.g. we could also write the above expression as
$$(ij, jjj \rightarrow i, A, (ik, kj \rightarrow iij, B, C))$$

\begin{theorem}
    \label{thm:nested_einsum:2}

    For $i \in [m + n]$, let $T^{(i)}$ be a $n_i$-th order tensor with index strings $\bm{s_i} \in S^{n_i}$.
    Let $\bm{s_u}$ be an index string for the $n_u$-th order tensor $U$, which is defined as follows:
    $$U := (\bm{s_{m + 1}},\dots,\bm{s_{m + n}} \rightarrow \bm{s_u}, T^{(m + 1)},\dots,T^{(m + n)})$$
    Also let $\bm{\hat{s}_u}$ be alternative index strings for $U$ with $s_{uj} = s_{uj'} \implies \hat{s}_{uj} = \hat{s}_{uj'}$ for all $j, j' \in [n_u]$,
    which means that $\bm{\hat{s}_u}$ can only introduce new symbol duplications, and cannot remove any.

    In our example, $\bm{s_u} = kko$ and $\bm{\hat{s}_u} = jjj$.
    This does not break the symbol duplication of the first and second index,
    and introduces a new duplication on the third index.

    Let $s_v$ be an index string and
    $$V := (\bm{s_1},\dots,\bm{s_m}, \bm{\hat{s}_u} \rightarrow \bm{s_v}, T^{(1)},\dots,T^{(m)}, U)$$
    such that the first and second einsum expression share no symbols.
    Then these nested einsum expressions can also be compressed into a single einsum expression.

    In contrast to \autoref{thm:nested_einsum:1}, we cannot just replace the input index string $\bm{\hat{s}_u}$ by all the input index strings in the inner einsum expression $\bm{s_{m + 1}},\dots,\bm{s_{m + n}}$.
    Instead, we first need to apply a symbol map.
    Let $\nu: S \rightarrow S$ such that
    $$\nu(s) := \begin{cases}
            \hat{s}_{uj} & \text{if }\exists j \in [n_u]: s_{uj} = s \\
            s            & \text{else}
        \end{cases}$$
    which maps symbols in $\bm{s_u}$ to the symbol at the same index in $\bm{\hat{s}_u}$ and all other symbols to themselves.

    This symbol map holds information about which symbols will be iterated over at the same time in the outer expression.
    In our example, the interesting parts of the map are $\nu(k) = j$ and $\nu(o) = j$, which means that $k$ and $j$ will be iterated over at the same time.

    $\nu$ can be extended, such that it maps entire index strings instead of just symbols, by setting $\nu(\bm{s_i}) \in S^{n_i}, \nu(\bm{s_i})_j := \nu(s_{ij})$.
    Then we can write the substituted index strings by setting $\bm{\hat{s}_i} := \nu(\bm{s_i})$ for $i \in [m + 1, m + n]$.

    Then the compressed einsum expression is the following:
    $$V = (\bm{s_1},\dots,\bm{s_m}, \bm{\hat{s}_{m + 1}}, \dots, \bm{\hat{s}_{m + n}} \rightarrow \bm{s_v}, T^{(1)},\dots,T^{(m + n)})$$
    which helps us to compress the example:
    $$(ij, jjj \rightarrow i, A, (kl, lo \rightarrow kko, B, C)) = (ij, jl, lj \rightarrow i, A, B, C)$$
\end{theorem}
\begin{proof}
    The fundamental idea behind this theorem is, that by using the index string $\bm{\hat{s}_u}$, we only iterate over a sub-space of the indices that we defined for the computation of $U$.
    To formulate this, we need some idea of which multi-indices we iterate over.
    Therefore, let $\mathcal{M}:\bm{s} := \smallset{M: \bm{s} \mid M \in \mathcal{M}}$ for an index string $\bm{s}$ and a multi-index space $\mathcal{M}$.

    Let $\mathcal{K} = \prod_{s \in \sigma(\hat{s}_u)} [d_s]$ be the multi-index space in the input for the computation of $V$,
    and let $\mathcal{K}' = \prod_{s \in \sigma(s_u)} [d_s]$ be the multi-index space in the output for the computation of $U$.
    Then $\mathcal{K}:\bm{\hat{s}_u} \subseteq \mathcal{K}':\bm{s_u}$, because $d_{s_{uj}} = d_{\hat{s}_{uj}}$ per the definition of einsum,
    and because the amount of axes contributing to $\mathcal{K}$ ($\abs{\sigma(\bm{\hat{s}_u})}$) has to be smaller or equal to the amount of axes contributing to $\mathcal{K}'$ ($\abs{\sigma(\bm{s_u})}$).
    This last fact follows from the constraint $s_{uj} = s_{uj'} \implies \hat{s}_{uj} = \hat{s}_{uj'}$.

    Then
    $$\forall K' \in \mathcal{K}': U_{K': \bm{s_u}} = \bigoplus\limits_{J' \in \mathcal{J}'}\bigodot\limits_{i = m + 1}^{m + n} T^{(i)}_{K'J':\bm{s_{i}}}$$
    and therefore
    $$\forall K \in \mathcal{I}': U_{K: \bm{\hat{s}_u}} = \bigoplus\limits_{J' \in \mathcal{J}'}\bigodot\limits_{i = m + 1}^{m + n} T^{(i)}_{KJ':\bm{\hat{s}_{i}}}$$
    because of the previous observation,
    and because the free symbols of the expression, which are used in $J'$, are not changed by the symbol map $\nu$.

    Therefore
    \begin{align*}
        V                                               & = (\bm{s_1},\dots,\bm{s_m}, \bm{\hat{s}_u} \rightarrow \bm{s_v}, T^{(1)},\dots,T^{(m)}, U)                                                                                                                   \\
        \iff \forall I \in \mathcal{I}: V_{I: \bm{s_v}} & = \bigoplus\limits_{J \in \mathcal{J}} \bigodot\limits_{i = 1}^{m} T^{(i)}_{IJ:\bm{s_i}} \odot U_{IJ:\bm{\hat{s}_u}}                                                                                         \\
                                                        & = \bigoplus\limits_{J \in \mathcal{J}} \bigodot\limits_{i = 1}^{m} T^{(i)}_{IJ:\bm{s_i}} \odot \bigoplus\limits_{J' \in \mathcal{J}'} \bigodot\limits_{i' = m + 1}^{m + n} T^{(i')}_{IJJ':\bm{\hat{s}_{i'}}} \\
                                                        & = \bigoplus\limits_{J \in \mathcal{J} \times \mathcal{J}'} \bigodot\limits_{i = 1}^{m} T^{(i)}_{IJ:\bm{s_i}} \odot \bigodot\limits_{i = m + 1}^{m + n} T^{(i)}_{IJ:\bm{\hat{s}_i}}                           \\
        \iff V                                          & = (\bm{s_1}, \dots, \bm{s_m}, \bm{\hat{s}_{m + 1}}, \dots, \bm{\hat{s}_{m + n}} \rightarrow \bm{s_v}, T^{(1)}, \dots, T^{(m + n)})
    \end{align*}
    where the third equality holds because we only iterate over a sub-space of the indices that we defined for the computation of $U$,
    and because the first and second einsum expression share no symbols.
    The rest of the steps are the same as in \autoref{thm:nested_einsum:1}.
\end{proof}

With this theorem we can prove a property of the the trace in a relatively simple manner, namely that for $A \in \R^{m \times n}, B \in \R^{n \times m}$,
it holds that
$$\text{trace}(A \cdot B) = \text{trace}(B \cdot A)$$

\begin{proof}
    \begin{align*}
        \text{trace}(A \cdot B) & = (ii \rightarrow , (ik,kj \rightarrow ij, A, B)) \\
                                & = (ik, ki \rightarrow ,A, B)                      \\
                                & = (ki, ik \rightarrow ,A, B)                      \\
                                & = (ik, ki \rightarrow ,B, A)                      \\
                                & = (ii \rightarrow , (ik,kj \rightarrow ij, B, A)) \\
                                & = \text{trace}(B \cdot A)
    \end{align*}
    where the second equality holds because of \autoref{thm:nested_einsum:2},
    the third equality is just a renaming of the indices,
    and the fourth equality holds because of the commutivity in the used semiring.
\end{proof}

This is already a useful tool for compressing nested expressions, but there are still some naturally occuring expressions we cannot compress with this,
e.g. for $A \in \R^{m \times n}, v \in \R^{n}$:
$$A \cdot \text{diag}(v) = (ik, kj \rightarrow ij, A, (i \rightarrow ii, v))$$
This is because we use more indices than the ones we computed by breaking the symbol duplication $ii$ with the index string $kj$.

In the following theorem, we explore a way of compressing the expression
$$(ijk, jk \rightarrow ijk, A, (l \rightarrow ll, v))$$
Again, we use disjoined sets of symbols for the inner and outer expression to help us in the formulation and the proof.

\begin{theorem}
    \label{thm:nested_einsum:3}

    For $i \in [m + n]$, let $T^{(i)}$ be a $n_i$-th order tensor with index strings $\bm{s_i} \in S^{n_i}$.
    Let $\bm{s_u}$ be an index string for the $n_u$-th order tensor $U$, which is defined as follows:
    $$U := (\bm{s_{m + 1}},\dots,\bm{s_{m + n}} \rightarrow \bm{s_u}, T^{(m + 1)},\dots,T^{(m + n)})$$
    Also let $\bm{\hat{s}_u}$ be alternative index strings for $U$ with $s_{uj} \neq s_{uj'} \implies \hat{s}_{uj} \neq \hat{s}_{uj'}$ for all $j, j' \in [n_u]$,
    which means that $\bm{\hat{s}_u}$ can only remove symbol duplications, and cannot introduce any.
    Note that this is the converse of the constraint in \autoref{thm:nested_einsum:2}.

    In our example, $\bm{s_u} = ll$ and $\bm{\hat{s}_u} = jk$.
    This removes the symbol duplication of the first and second index.

    Let $s_v$ be an index string and
    $$V := (\bm{s_1},\dots,\bm{s_m}, \bm{\hat{s}_u} \rightarrow \bm{s_v}, T^{(1)},\dots,T^{(m)}, U)$$
    where the first and second einsum expression share no symbols.
    Then these nested einsum expressions can also be compressed into a single einsum expression.

    As in \autoref{thm:nested_einsum:3}, we need to apply a symbol map before substituting $\bm{\hat{s}_u}$.
    Interestingly, the symbol map is not applied to the index strings in the computation of $U$ ($\bm{s_{m + 1}},\dots,\bm{s_{m + n}}$),
    but to the index strings in the computation of $V$ ($\bm{s_1},\dots,\bm{s_m}$).
    Similarly, it does not map $\bm{s_u}$ to $\bm{\hat{s}_u}$, but $\bm{\hat{s}_u}$ to $\bm{s_u}$.

    Let $\mu: S \rightarrow S$ such that
    $$\mu(s) := \begin{cases}
            s_{uj} & \text{if }\exists j \in [n_u]: \hat{s}_{uj} = s \\
            s      & \text{else}
        \end{cases}$$
    $\mu$ can be extended in a similar way as $\nu$ to map entire index strings.

    Let $\bm{\hat{s}_i} := \mu(\bm{s_i})$ for $i \in [m]$, $\bm{\hat{s}_v} := \mu(\bm{s_v})$, then the compressed einsum expression is the following:
    $$V = (\bm{\hat{s}_1},\dots,\bm{\hat{s}_m}, \bm{s_{m + 1}}, \dots, \bm{s_{m + n}} \rightarrow \bm{\hat{s}_v}, T^{(1)},\dots,T^{(m + n)})$$
    which helps us to compress the example:
    $$(ijk, jk \rightarrow ijk, A, (l \rightarrow ll, v)) = (ill, l \rightarrow ill, A, v)$$
    Note how even the index string for the output $\bm{s_v}$ was changed into $\bm{\hat{s}_v}$.
    This will become apparent in the proof.

\end{theorem}
\begin{proof}
    The key idea behind this proof, is that the entries of $U$, which were not defined in the computation, are set to the additive neutral element $\0$.
    This is useful, because in a semiring over some set $M$, the additive neutral element \textit{annihilates} $M$.
    This means, that for any $a \in M$, $a \cdot \0 = \0 \cdot a = \0$.
    Therefore, for any multiindex where $U$ is set to $\0$, $V$ is also set to $\0$.
    This means, that in the computation of $V$, only the indices which follow the duplications in $\bm{s_u}$ are defined.
    % Let us prove this formally.

    Let $\mathcal{I}, \mathcal{J}$ be the appropriate multi-index spaces for the computation of $V$. Then
    $$U_{IJ:\bm{\hat{s}_u}} = \0$$
    for all $IJ \in \mathcal{I} \times \mathcal{J}$ with
\end{proof}

\subsection{More Examples}
With these theorems, we can write some more complex expressions as einsum.
\begin{itemize}
    \item squared norm of matrix-vector multiplication: Let $A \in \R^{m \times n}, v \in \R^{n}$. Then
          \begin{align*}
              \abs{A \cdot v}_2^2 & = (i,i\rightarrow,(ij, j \rightarrow i, A, v),(ij, j \rightarrow i, A, v)) \\
                                  & = (ij,j,ij,j\rightarrow,A,v,A,v)
          \end{align*}
    \item trace of matrix-matrix multiplication: Let $A \in \R^{m \times n}, B \in \R^{n \times m}$. Then
          \begin{align*}
              \text{trace}(A \cdot B) & = (ii \rightarrow, (ik, kj \rightarrow ij, A, B)) \\
                                      & = (ik, ki \rightarrow, A, B)
          \end{align*}
    \item The theorem for this still has to be shown \dots:
          matrix multiplication with a diagonal matrix: Let $A \in \R^{m \times n}, v \in \R^{n}$. Then
          \begin{align*}
              A \cdot \text{diag}(v) & = (ik, kj \rightarrow ij, A, (i \rightarrow ii, v)) \\
                                     & = (ij, j \rightarrow ij, A, v)                      \\
          \end{align*}
\end{itemize}

In the final generalisation of compressing nested einsum expressions, all duplication breaking is allowed.
This has no application for linear algebra, as all the previous theorems had, because this first comes into play with third order tensors.
This is because, with two or less axes, there is no possibility of simultaniously breaking a duplication and introducing a new one.
Nevertheless, it serves as a useful tool for compressing all kinds of nested einsum expressions.

% TODO: example
% In the following theorem, we explore a way of compressing the expression
% $$(ijk, jk \rightarrow ijk, A, (l \rightarrow ll, v))$$
% Again, we use disjoined sets of symbols for the inner and outer expression to help us in the formulation and the proof.

\begin{theorem}
    \label{thm:nested_einsum:4}
    For $i \in [m + n]$, let $T^{(i)}$ be a $n_i$-th order tensor with index strings $\bm{s_i} \in S^{n_i}$.
    Let $\bm{s_u}$ be an index string for the $n_u$-th order tensor $U$, which is defined as follows:
    $$U := (\bm{s_{m + 1}},\dots,\bm{s_{m + n}} \rightarrow \bm{s_u}, T^{(m + 1)},\dots,T^{(m + n)})$$
    Also let $\bm{\hat{s}_u}$ be alternative index strings for $U$.

    Let $s_v$ be an index string and
    $$V := (\bm{s_1},\dots,\bm{s_m}, \bm{\hat{s}_u} \rightarrow \bm{s_v}, T^{(1)},\dots,T^{(m)}, U)$$
    where the first and second einsum expression share no symbols.
    Then these nested einsum expressions can also be compressed into a single einsum expression.

    Once again, a map $\omega: S \rightarrow S$ has to be applied to the index strings before substituting index strings.
    The definition of the map this time is somewhat more complex.
    As in the previous two theorems, this map holds information about which symbols are essentially used together as one index.

    For the definition of the map $\omega$, we first construct an undirected graph $G = (V, E)$,
    in which the nodes $V = \sigma(\bm{s_v}) \cup \sigma(\bm{s_u}) \cup \sigma(\bm{\hat{s}_u}) \cup \bigcup_{i \in [m + n]} \sigma(\bm{s_i})$ consist of all symbols from both expressions.
    Now the edges of this graph are $E = \smallset{\smallset{s_{uj}, \hat{s}_{uj}} \mid \exists j \in [n_u]: s_{uj} \neq \hat{s}_{uj}}$,
    which connects all symbols from $\bm{s_u}$ and $\bm{\hat{s}_u}$ that share an index.

    In this graph, if two symbols are connected, then they to be iterated over at the same time in the compressed expression, because they are essentially the same index.
    Therefore, it makes sense assigning a symbol $s_C \in S \setminus V$ to each of the graphs components $C$.
    Then we can define $\omega$ as follows:
    $$\omega(s) := \begin{cases}
            s_C & \text{if } s \in C \\
            s   & \text{else}
        \end{cases}$$

    $\omega$ can be extended in a similar way as $\mu, \nu$ to map entire index strings.
    In this general form, the this map is applied to all index strings from both expressions before the substitution.
    Let $\bm{\hat{s}_i} := \omega(\bm{s_i})$ for $i \in [m + n]$, $\bm{\hat{s}_v} := \omega(\bm{s_v})$, then the compressed einsum expression is the following:
    $$V = (\bm{\hat{s}_1}, \dots, \bm{\hat{s}_{m + n}} \rightarrow \bm{\hat{s}_v}, T^{(1)},\dots,T^{(m + n)})$$
\end{theorem}
\begin{proof}
    Wird spannend.
\end{proof}
