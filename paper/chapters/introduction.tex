\chapter{Introduction}

% explain on the example of SAT and ILPs and stuff why frameworks are important?
% Frameworks like SAT and ILP
% Naturally occuring problems in the industry can often be reduced to frameworks like SAT or an ILP, which are well studied and optimized.
% Naturally occuring problems in the industry are often solved in frameworks like SAT or an ILP, because they are well understood, well optimized, and mostly intuitive.
% With the use of framworks like SAT and ILP,
% With the use of such frameworks, we get a mostly efficient solution to naturally occuring problems by reformulating them in this framwork and applying a general solver that was built for problems of this framework.

% Opening
In the process of building software, it can often happen that one encounters optimization problems.
When such a problem occurs in this natural way, it can lead to vast amounts of time spent on implementing a solution to the specific problem, which can severely slow down production time.
To help with this, we can use frameworks like SAT and ILP.
The benefit of such frameworks is that we do not need to spend much time building an efficient solver for each individual problem.
Instead, we can translate the individual problems to such a framework, which is often much easier.
Then a general solver can be applied to the now embedded problem, which often yields efficient solutions.
This can make the process of coming up with good solutions to naturally occurring problems much faster.

% Challenge
Inference problems are a subset of naturally occurring problems for which no such framework is known.
Such a framework would have to be powerful enough to solve modern inference problems and specific enough to find good general optimization techniques.
% Action
Einsum seems like a promising candidate for that role.
% do they occur frequently in inference problems? another/different reason why Einsum is promising?
% TODO: maybe don't quote the SQL thing here?
It is powerful enough to support SAT \cite{Biamonte2014} and ILP (see \cref{chap:appendix:map_ilp}), and is a natural framework for operations on tensors, which occur frequently in inference problems.
It is also specific enough so that efficient solvers can be built.
Einsum can include multiple reductions, which allows us to optimize the order in which these reductions are applied.
An example of this is matrix-matrix-vector multiplication $w = A B v$ for matrices $A$ and $B$ and a vector $v$.
Here it is more efficient to first compute the right matrix-vector product $u = Bv$, and then compute the product of the remaining matrix with the resulting vector $w = Au$,
instead of first computing the left matrix-matrix product $C = AB$, and then computing the product of the resulting matrix with the remaining vector $w = Cv$.
Additionally, Einsum's restriction on tensor operations allows the use of specific hardware which is optimized for exactly those operations.
% Resolution?
To explore if Einsum is a good fit for a universal inference language,
we try to translate a small set of modern inference techniques to Einsum.

% another short opening + challenge + resolution
Another problem we investigate is the nesting of Einsum expressions, which naturally occurs when trying to translate chained operations to Einsum.
Nested expressions partially force the order in which reductions are applied, meaning the order cannot be optimized freely.
To avoid this constraint, we explore a way of compressing nested Einsum expressions into flat Einsum expressions.