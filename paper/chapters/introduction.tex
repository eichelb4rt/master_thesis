\chapter{Introduction}

% explain on the example of SAT and ILPs and stuff why frameworks are important?
% Frameworks like SAT and ILP
% Naturally occuring problems in the industry can often be reduced to frameworks like SAT or an ILP, which are well studied and optimized.
% Naturally occuring problems in the industry are often solved in frameworks like SAT or an ILP, because they are well understood, well optimized, and mostly intuitive.
% With the use of framworks like SAT and ILP,
% With the use of such frameworks, we get a mostly efficient solution to naturally occuring problems by reformulating them in this framwork and applying a general solver that was built for problems of this framework.

% Opening
Frameworks like SAT and ILP are essential tools for coming up with quick solutions to natural problems.
The benefit of such framworks is that we do not need to spend much time building an efficient solver for each individual problem.
Instead we can translate the individual problems to such a framework, which is often much easier.
Then a general solver can be applied to the now embedded problem, which often yields efficient solutions.
This can make the process of coming up with good solutions to naturally occuring problems much faster.

% Challenge
We have not yet found such a framework that is powerful enough to solve modern inference problems
and specific enough to find good general optimization techniques.
% Action
Einsum seems like a promising candidate for that role.
% do they occur frequently in inference problems? another/different reason why Einsum is promising?
It is powerful enough to support SAT and ILP (TODO: cite), and is a natural framework for operations on tensors, which occur frequently in inference problems.
Because of its form as some reduction over some combination of inputs, it is specific enough for us to build relatively efficient solvers, which can optimize the contraction path.
Additionally, its restriction on tensor operations enables us to use specific hardware that optimizes these.

% Resolution?
To find out if Einsum is a good fit for a universal inference language,
we try to translate a small set of modern inference techniques to Einsum.
Additionally, we explore a way of embedding nested Einsum expressions in the framework of Einsum,
which usually only includes flat expressions,
because nested expressions occur naturally when trying to translate chained operations.